%auto-ignore
%      this ensures the arxiv doesn't try to start TeXing here.
%!TEX root = super_lattice_models_draft.tex
%      prev line helps TeXShop do the right thing

%%%%%%%%%%%%%%%%%%%%%%%%%%%
\section{$\halfesix$ data}
%%%%%%%%%%%%%%%%%%%%%%%%%%%

%%%%%%%%%%%%%%%%
\subsection{Associators}   \label{E6Fsymbols}
%%%%%%%%%%%%%%%%

There are four solutions to the pentagon equation for $\frac{1}{2}$E6 fusion rules \cite{Hagge2007}.
They split into two sets, one pair has all positive quantum dimensions, while the other has negative quantum dimension on the $x$ particle. 
The solutions in each set are related by complex conjugation. 
Here we present one of the solutions with positive quantum dimensions on all particles (they have been extracted from \cite{Wakui2002}) . 
Several of the $F$ symbols are trivial,
\begin{align}
F^{yyy}_y = F^{xyy}_x = F^{yyx}_x = F^{xxy}_y  = F^{yxx}_y = F^{xyx}_{\mathds{1}} = F^{xxy}_{\mathds{1}} = F^{yxx}_{\mathds{1}}= 1
\end{align}
Let $v_1, v_2$ be orthogonal unit vectors for the two-dimensional splitting space $V^{xx}_x$. 
We define the $F$ symbols acting on these vectors by $V^{xy}_x \tp V^{xx}_x  = F^{xyx}_x V^{xx}_x \tp V^{yx}_x$ in the basis $(v_1, v_2)^{T}$.
Explicitly we have,
\begin{align}
&F^{xyx}_x = \left(\begin{matrix}
0&1\\
1&0
\end{matrix} \right) \quad
F^{xxy}_x = \left(\begin{matrix}
0&-i\\
i&0
\end{matrix} \right) \quad
F^{yxx}_x = \left(\begin{matrix}
1&0\\
0&-1
\end{matrix} \right) 
\\
\nonumber \\
&F^{xxx}_{\mathds{1}} = c_2^* \left(\begin{matrix}
1&-i\\
1&i
\end{matrix} \right) \quad
F^{xxx}_y = c_2^*\left(\begin{matrix}
1&-i\\
-1&-i
\end{matrix} \right), \quad \quad c_2  = \frac{e^{7 i \pi/12}}{\sqrt{2}}.
\end{align}

Lastly we have the $F$ symbol whose four external legs are all labeled by $x$.
We write this $F$ symbol down in the basis $(0,y, v_1\tp v_1,v_1 \tp v_2, v_2 \tp v_1, v_2 \tp v_2)$, where we have 
\begin{align}
F^{xxx}_x  = 
 \left(\begin{matrix}
\frac{1}{d} & \frac{1}{d} & \frac{c_1^*}{\sqrt{2}} & \frac{c_1^*}{\sqrt{2}} & \frac{c_1^*}{\sqrt{2}} & - \frac{c_1^*}{\sqrt{2}} \\
\frac{1}{d} & - \frac{1}{d} & \frac{c_1^*}{\sqrt{2}} & -\frac{c_1^*}{\sqrt{2}} & \frac{c_1^*}{\sqrt{2}} & \frac{c_1^*}{\sqrt{2}} \\
\frac{e^{- i \pi/4}}{\sqrt{d}} & 0 & -\frac{1}{d} & 0 & -i c_4^* & 0 \\
0 & \frac{e^{i \pi /4}}{\sqrt{d}} & 0 &c_4^* & 0 & \frac{ i}{d} \\
0 & \frac{e^{- i \pi /4}}{\sqrt{d}} & 0 & -\frac{1}{d} & 0 & i c_4^* \\ 
\frac{e^{i \pi /4}}{\sqrt{d}} & 0 & c_4^* & 0 & - \frac{i}{d} & 0
\end{matrix} \right) \\
d = 1 + \sqrt{3} \quad \quad c_1 = \frac{e^{-5 i \pi/6} }{\sqrt{d}} \quad \quad c_4 = \frac{e^{-i \pi/4}}{\sqrt{2}}.
\end{align}


\subsection{Idempotents}
\label{IdempotentsHalfESix}

In this appendix we provide the minimal idempotents of $\tube(\halfesix)$, and give there images under condensation of $y$.
The inclusion is performed by simply condensing fermions off the tubes in $\tube(\halfesix)$ to get tubes in $\tube(\halfesix / y)$. 
Special care must be taken with respect to spin structure issues, 
since removing $y$ lines may force a pair of fermions to traverse a cycle of the tube. 

The minimal idempotents of $\tube(\halfesix)$ are listed in tables \ref{mor0to0}, \ref{mor1to1}, and \ref{mor2to2}.
They were found by brute force on a computer.
We also identify the minimal idempotents of $\tube(\halfesix)$ with simple objects of the Drinfeld center $\mcz(\halfesix)$ listed in \cite{Hong2008}.%
\footnote{Disclaimer: 
At the level of fusion rules and spins,
idempotents $X_2$ and $X_3$ are identical and so there is an ambiguity in identifying these minimal idempotents with the simple objects in $Z(\mcc)$ of \cite{Hong2008}.
The spins here differ from those in \cite{Hong2008} by complex conjugation. } 
Under condensation of $y$ described in Section \ref{halfesix} we find the following maps from idempotents in $\tube(\halfesix)$ to those in in $\tube(\halfesix / y)$:
\begin{align}
\xymatrix @!0 @M=1mm @C=10mm{
& &m_1 && m_2 && &&& m_3^+ &&  \\
 \\
 \quad&\mathds{1} \ar[ruu] \quad & Y \ar[uu] & W\ar[dd] & U\ar[uu] & V \ar[uul] & X_1 \ar[dd] & X_{2}\ar[ddr] &X_{3} \ar[dd] & X_4\ar[uu] & X_5\ar[uul] &  \\
\\
&&&q_1& &&q_2 &&m_4^+& &&  \\
},
\end{align}
where the center line lists the idempotents in $\tube(\halfesix)$ and the upper and lower objects are the objects in $\tube(\halfesix / y)$.
The identifications are made by taking a minimal idempotent in $\tube(\halfesix)$ and using the inclusion to $\tube(\halfesix/y)$, 
as discussed in Section \ref{double_fermionic_quotient}.

\begin{table}
{\tabulinesep=1.2mm
\begin{tabu}{ c c | c c c c }
particle & spin &$e$&$l_x$&$l_y$& \\ \hline
$\unit$ & $
       1       

$ &
$
\frac{1}{2+d^2}
$
 &
$
\frac{1}{2 \sqrt3}
$
 &
$
\frac{1}{2+d^2}
$
 &
\\
$W$ & $
       1       

$ &
$
\frac{1}{2}
$
 &
$
$
 &
$ -
\frac{1}{2}
$
 &
\\
$U$ & $
       1       

$ &
$
\frac{d}{4 \sqrt{3}}
$
 &
$ -
\frac{1}{2 \sqrt3}
$
 &
$
\frac{d}{4 \sqrt{3}}
$
 &
\end{tabu}
\caption{
Minimal idempotents for $\tube_{\unit \to \unit}$.
 We have used the notation $e$ = empty diagram, $l_x = \cl(x) $, and $l_y = \cl(y)$}
\label{mor0to0}}
{\tabulinesep=1.2mm
\begin{tabu}{ c c | c c c c }
particle & spin &$t_y$&$t_y l_x$&$v_y$& \\ \hline
%2
$W$ 
& $       1       $ 
&$\frac{1}{2}$
 &$$
 &$\frac{1}{2}$
 &
\\
%4
$Y$ & $
       1/2     

$ &
$ -
\frac{1}{2+d^2}
$
 &
$ e^{\frac{6i\pi}{12}}
\frac{1}{2 \sqrt3}
$
 &
$
\frac{1}{2+d^2}
$
 &
\\
%5
$V$ & $
       1/2     

$ &
$ -
\frac{d}{4 \sqrt{3}}
$
 &
$ e^{\frac{-6i\pi}{12}}
\frac{1}{2 \sqrt3}
$
 &
$
\frac{d}{4 \sqrt{3}}
$
 &
\end{tabu}
}
\caption{
Minimal idempotents for $\tube_{y \to y}$.
We have used the notation $v_y = \text{id}_y \in \tube_{y \to y}$, $t_y = t_{y\unit y y; 11}$, and $t_y l_x = t_{yxyx; 11}$.
\label{mor2to2}}
\end{table}



\begin{table}
\resizebox{\textwidth}{!}{
{\tabulinesep=1.2mm
\begin{tabu}{ c c | c c c c c c c c c }
particle & spin &$t_x$&$v_x$&$X_{11}$ & $X_{12}$ &$X_{21}$ & $X_{22}$ & $v_x l_y$ & $t_x h_y$ \\ \hline
%2
$W$ & $
       1       
$ &
$
\frac{1}{2d}
$
 &
$
\frac{1}{2d}
$
 &
$ e^{\frac{-9i\pi}{12}}
\frac{\sqrt{3d-8}}{4}
$
 &
$ e^{\frac{-9i\pi}{12}}
\frac{\sqrt{3d-8}}{4}
$
 &
$ e^{\frac{9i\pi}{12}}
\frac{\sqrt{3d-8}}{4}
$
 &
$ e^{\frac{-3i\pi}{12}}
\frac{\sqrt{3d-8}}{4}
$
 &
$ e^{\frac{6i\pi}{12}}
\frac{1}{2d}
$
 &
$ e^{\frac{-6i\pi}{12}}
\frac{1}{2d}
$
 &
\\
%3
$U$ & $
       1       

$ &
$
\frac{1}{4 \sqrt{3}}
$
 &
$
\frac{1}{4 \sqrt{3}}
$
 &
$ e^{\frac{3i\pi}{12}}
\frac{1}{4 \sqrt{d} }
$
 &
$ e^{\frac{9i\pi}{12}}
\beta
$
 &
$ e^{\frac{-9i\pi}{12}}
\beta
$
 &
$ e^{\frac{-3i\pi}{12}}
\frac{1}{4 \sqrt{d} }
$
 &
$ e^{\frac{-6i\pi}{12}}
\frac{1}{4 \sqrt{3}}
$
 &
$ e^{\frac{6i\pi}{12}}
\frac{1}{4 \sqrt{3}}
$
 &
\\
%5
$V$ & $
       1/2     

$ &
$ -
\frac{1}{4 \sqrt{3}}
$
 &
$
\frac{1}{4 \sqrt{3}}
$
 &
$ e^{\frac{-3i\pi}{12}}
\beta
$
 &
$ e^{\frac{-9i\pi}{12}}
\frac{1}{4 \sqrt{d} }
$
 &
$ e^{\frac{-3i\pi}{12}}
\frac{1}{4 \sqrt{d} }
$
 &
$ e^{\frac{-9i\pi}{12}}
\beta
$
 &
$ e^{\frac{-6i\pi}{12}}
\frac{1}{4 \sqrt{3}}
$
 &
$ e^{\frac{-6i\pi}{12}}
\frac{1}{4 \sqrt{3}}
$
 &
\\
%6
$X_2 $ & $
      -5/12    

$ &
$ e^{\frac{-10i\pi}{12}}
\frac{1}{2+d^2}
$
 &
$
\frac{1}{2+d^2}
$
 &
$ e^{\frac{-2i\pi}{12}}
\gamma
$
 &
$ e^{\frac{10i\pi}{12}}
\alpha
$
 &
$ e^{\frac{4i\pi}{12}}
\alpha
$
 &
$ e^{\frac{4i\pi}{12}}
\gamma
$
 &
$ e^{\frac{6i\pi}{12}}
\frac{1}{2+d^2}
$
 &
$ e^{\frac{8i\pi}{12}}
\frac{1}{2+d^2}
$
 &
\\
%7
$X_1$ & $
       1/4     

$ &
$ e^{\frac{6i\pi}{12}}
\frac{1}{2+d^2}
$
 &
$
\frac{1}{2+d^2}
$
 &
$ e^{\frac{6i\pi}{12}}
\frac{1}{2 \sqrt{6d}}
$
 &
$ e^{\frac{6i\pi}{12}}
\frac{1}{2 \sqrt{6d}}
$
 &
$
\frac{1}{2 \sqrt{6d}}
$
 &
$ -
\frac{1}{2 \sqrt{6d}}
$
 &
$ e^{\frac{6i\pi}{12}}
\frac{1}{2+d^2}
$
 &
$
\frac{1}{2+d^2}
$
 &
\\
%8
$X_{3}$ & $
      -5/12    

$ &
$ e^{\frac{-10i\pi}{12}}
\frac{1}{2+d^2}
$
 &
$
\frac{1}{2+d^2}
$
 &
$ e^{\frac{10i\pi}{12}}
\alpha
$
 &
$ e^{\frac{-2i\pi}{12}}
\gamma
$
 &
$ e^{\frac{-8i\pi}{12}}
\gamma
$
 &
$ e^{\frac{-8i\pi}{12}}
\alpha
$
 &
$ e^{\frac{6i\pi}{12}}
\frac{1}{2+d^2}
$
 &
$ e^{\frac{8i\pi}{12}}
\frac{1}{2+d^2}
$
 &
\\
%9
$X_5$ & $
       1/3     

$ &
$ e^{\frac{8i\pi}{12}}
\frac{1}{2+d^2}
$
 &
$
\frac{1}{2+d^2}
$
 &
$
$
 &
$ e^{\frac{1i\pi}{12}}
\frac{1}{2 \sqrt{3d} }
$
 &
$ e^{\frac{7i\pi}{12}}
\frac{1}{2 \sqrt{3d} }
$
 &
$
$
 &
$ e^{\frac{-6i\pi}{12}}
\frac{1}{2+d^2}
$
 &
$ e^{\frac{-10i\pi}{12}}
\frac{1}{2+d^2}
$
 &
\\
%10
$X_4$ & $
      -1/6     

$ &
$ e^{\frac{-4i\pi}{12}}
\frac{1}{2+d^2}
$
 &
$
\frac{1}{2+d^2}
$
 &
$ e^{\frac{-11i\pi}{12}}
\frac{1}{2 \sqrt{3d} }
$
 &
$
$
 &
$
$
 &
$ e^{\frac{7i\pi}{12}}
\frac{1}{2 \sqrt{3d} }
$
 &
$ e^{\frac{-6i\pi}{12}}
\frac{1}{2+d^2}
$
 &
$ e^{\frac{2i\pi}{12}}
\frac{1}{2+d^2}
$
 &
\end{tabu}
}
}
\caption{Minimal idempotents for $\tube_{x \to x}$. 
We have used the notation $t_x =t_{x\unit xx; 11}$, $v_x = \text{id}_x \in \tube_{x \to x}$, $X_{ij} = t_{xxx; ij}$, $v_x l_y = t_{xxxy; 11}$, and $t_x h_y = t_{xyxx; 11}$. 
Where $\alpha = \frac{1}{2} \left( 1+ 1/\sqrt{2d+1} \right)$, and $\beta = \frac{1}{2} \left( 1- 1/\sqrt{2d+1} \right)$, $\gamma/\alpha =1/(2\sqrt{d} 3^{1/4})$, and $d = 1+ \sqrt{3} $. 
\label{mor1to1}}
\end{table}


Some of the idempotents are isomorphic. 
For example, $W$ appears in all three tables \ref{mor0to0}, \ref{mor1to1} and \ref{mor2to2} (with its boundary condition $\unit$, $y$ or $x$ implicit in each table).
As usual, if $e$ and $e'$ are isomorphic idempotents, then we can find morphisms $u$, $v$ such that $e = u\cdot v$ and $e' = v \cdot u$.
In the following we denote the boundary condition of each idempotent by a subscript and similarly for the morphisms, so that, e.g., $W_x = w_{x \unit} \cdot w_{\unit x} = w_{xy} \cdot w_{yx}$, and $W_{\unit} = w_{\unit x} \cdot w_{x \unit}  = w_{\unit y}\cdot w_{y \unit} $ and so on. 
We have:
\begin{align}
w_{\unit x} &= \frac{i}{\sqrt{2 \sqrt{2d}}}(t_{\unit xxx; 11} - t_{\unit xxx;12}) \\
w_{x \unit}  &= \frac{-i}{\sqrt{2\sqrt{2d}}} (t_{xx \unit x;11}-t_{xx\unit x ;21}) \\
w_{xy} & = \frac{1}{(8d)^{\frac14}}(t_{xx y x ;11} + t_{xxyx;21}) \\
w_{yx} & = \frac{1}{(8d)^{\frac14}}(t_{yxxx;11} + t_{yxxx;12}) \\
w_{\unit y } & \frac{-e^{-i\pi/4}}{\sqrt{2}} t_{\unit x y x;11}\\
w_{y \unit} & \frac{-e^{i\pi/4}}{\sqrt{2}} t_{yx \unit x ;11}
\end{align}
\begin{align}
u_{\unit x}  & = \frac{1}{2}\sqrt{\frac{d}{6}}( t_{\unit xxx;11}+t_{\unit xxx;12})\\
u_{x \unit} & = \frac{1}{2}\sqrt{\frac{d}{6}}(t_{xx\unit x;11}+t_{xx \unit x;21})\\
v_{xy } & = \frac{i}{2}\left( \frac{d}{6} \right)^{\frac14} (t_{xxy x;11}-t_{xxyx;21})\\
v_{yx} & = \frac{i}{2}\left( \frac{d}{6} \right)^{\frac14} (t_{yxxx;11}-t_{yxxx;21})
\end{align}
with
\begin{align}
t_{abcd; \mu \nu } = \AnnularTubexp{\AnnularTubeNoIndex}{a}{b}{c}{d}{\mu}{\nu}{}{}.
\end{align}
In terms of diagrams, we have
\begin{align}
\xymatrix @!0 @M=2mm @R=22mm @C=19mm{
W_{\unit} \ar@/^2pc/[rr]^{w_{\unit y }} \ar@<.5ex>[r]^{w_{\unit x}}
& W_x \ar@<.5ex>[l]^{w_{x \unit }} \ar@<.5ex>[r]^{w_{xy}}
& W_y\ar@<.5ex>[l]^{w_{yx}} \ar@/^2pc/[ll]^{w_{y\unit}}
}\qquad
\xymatrix @!0 @M=2mm @R=22mm @C=19mm{
U_\unit\ar@<.5ex>[r]^{u_{\unit x}}
& U_x \ar@<.5ex>[l]^{u_{x \unit}} 
}\qquad
\xymatrix @!0 @M=2mm @R=22mm @C=19mm{
V_x\ar@<.5ex>[r]^{v_{xy}}
& V_y \ar@<.5ex>[l]^{v_{yx}} 
}.
\end{align}
Composing the morphisms in various ways constructs all isomorphic idempotents.

