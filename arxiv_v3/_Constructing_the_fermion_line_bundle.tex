%auto-ignore
%      this ensures the arxiv doesn't try to start TeXing here.
%!TEX root = super_lattice_models_draft.tex
%      prev line helps TeXShop do the right thing

%%%%%%%%%%%%%%%%%%%%%%%
\section{Constructing the fermion line bundle}
\label{fermion_line_bundle} \label{flb_appendix}
%%%%%%%%%%%%%%%%%%%%%%%

Recall our set-up from the end of Section \ref{condensing_psi}: 
We have a back wall $B$, which is a spin (and therefore oriented) 2-manifold.
Associated to $B$ is the configuration space of $\psi$-ribbon endpoints, $\mcr(B)$.
This configuration space is a disjoint union of pieces $\mcr(B)_k$, where $k$ is number of ribbon endpoints
in a configuration.

Our goal in this subsection is to construct a complex line-bundle-with-flat-connection $F(B)$ over $\mcr(B)$,
satisfying the following six conditions alluded to in Section \ref{condensing_psi}:
\begin{enumerate}
\item $F$ is functorial with respect to spin diffeomorphisms. That is,
if $f: B \to B'$ is a spin diffeomorphism, then there is a corresponding bundle isomorphism $F(B) \to F(B')$ which
preserves the flat connections and complex structure.
\item $F$ is functorial with respect to orientation-reversing $\mbox{pin}_+$ diffeomorphisms. That is, 
if $f: B \to B'$ is an orientation-reversing $\mbox{pin}_+$ diffeomorphism, then there is a corresponding map
$F(B) \to F(B')$ which preserves the flat connections and is complex antilinear on the fibers.
(Recall that any spin manifold has an associated $\mbox{pin}_+$ structure.
By ``orientation-reversing $\mbox{pin}_+$ diffeomorphism", we mean a $\mbox{pin}_+$ diffeomorphism
of the associated $\mbox{pin}_+$ manifolds which reverses the orientations of the underlying oriented manifolds.)
This condition is needed in order to define Hermitian/unitary structures.
\item The holonomy around a loop in $\mcr(B)$ corresponding to a $2\pi$ rotation of a ribbon endpoint is $-1$.
This condition is needed to compensate for fermionic twist of $\psi$.
\item The holonomy around a loop in $\mcr(B)$ corresponding to an exchange of two ribbon endpoints 
(inside a fixed disk) is $-1$.
This condition is needed to compensate for the fermionic statistics of $\psi$.
\item $F$ is local in the following sense.
Given a decomposition $B = B'\cup B''$, there is an obvious map $u: \mcr(B')\times\mcr(B'')\to\mcr(B)$, 
and a corresponding pull-back bundle $u^*(F(B))$ over $\mcr(B')\times\mcr(B'')$.
We require an isomorphism $u^*(F(B)) \cong F(B')\otimes F(B'')$ which is natural with respect to spin diffeomorphisms.
\item $F$ satisfies a cancellation property.
Given a configuration $r \in \mcr(B)_k$ and a point $x\in B$ distinct from the ribbon endpoints of $c$,
we can create a new configuration $c_+ \in \mcr(B)_{k+2}$ by inserting a pair of endpoints in a standard configuration
near $x$.
We require an isomorphism of fibers $F(B)_c \cong F(B)_{c_+}$ which is compatible with the flat connection as explained below.
This condition is needed to allow for well-defined creation and annihilation of pairs of $\psi$ 
endpoints in line with the fusion rule $\psi \tp \psi \cong \unit$.
\end{enumerate}

\medskip

To represent an element of $F(B)$, we will choose spin framings at each ribbon endpoint and also assign
an ordering to the ribbon endpoints.
The main idea is fairly simple, but making this construction compatible with orientation reversal
and ribbon endpoint cancellations requires a bit of fussiness with the details.

\medskip

Recall the group $\pin_+(1) \cong \zz/2 \times \zz/2$.
We will call the non-identity elements of $\pin_+(1)$ the ``spin flip" 
(the non-trivial element of the kernel of the covering map $\pin_+(1) \to O(1)$), 
the ``reflection", and the ``other reflection"
(the latter two reflections map to the single reflection in $O(1)$).

To construct $F(B)$, we will first construct a principal $\pin_+(1)$ bundle $P(B)$ over $\mcr(B)$.
The construction of $P(B)$ will be independent of reversing the orientation of $B$.
We define the action of $\pin_+(1)$ on $\cc$ as follows:
the spin flip sends $z\in \cc$ to $-z$;
the reflection sends $z$ to the complex conjugate $\bar z$;
and the other reflection sends $z$ to $-\bar z$.
Using this action, we can now define $F(B)$ to be
\be \label{fermbundef}
	F(B) = P(B) \times_{\pin_+(1)} \cc ,
\ee
the associated $\cc$ bundle over $\mcr(B)$.
(Recall that this means that elements of $F(B)$ are represented by pairs $(f, z) \in P(B) \times \cc$,
and that for each element $a \in \pin_+(1)$, we identify $(f\cdot a, z)$ with $(f, a \cdot z)$.)
Since $P(B)$ is a bundle with discrete fibers, it has a canonical flat connection.
This induces a flat connection on $F(B)$.

Note that $F(B)$, as defined above, has two different complex-linear structures, one the conjugate of the other.
We will see below that the orientation of $B$ picks out one of the two possible complex structures.

\medskip

We are now ready, finally, to construct $P(B)$.
Let $r \in \mcr(B)_k$ be a configuration of $k$ ribbon endpoints.
At each endpoint of $r$, there is a distinguished unit tangent vector $v \in TB$
pointing in the direction of the front of the ribbon.
There are two unit tangent vectors $w_1$ and $w_2$ in $TB$ orthogonal to the distinguished vector.
The orientation of $B$ allows us to designate one of these two orthogonal vectors as ``positive"
and the other as ``negative".
(We call $w_i$ positive if $(v, w_i)$ is a positively oriented frame with respect to the orientation of $B$.)
We will call a collection of framings $(v, w_i)$ at each endpoint ``consistent" if they are all positive
or all negative.
Note that there are exactly two possible consistent collections of framings for each fixed configuration $r$.
We will denote this set of two elements by $cf(r)$.
Note that reversing the orientation of $B$ does not change $cf(r)$.\footnote{It is tempting 
to say that $cf(r)$ depends only on the underlying unoriented manifold of $B$, but this
is not true if $B$ has more than one path component.
Reversing the orientation of some but not all of the path components of $B$ would change $cf(r)$.}

\newcommand{\tcf}{\widetilde{cf}}

Now define $\tcf(r)$ to be the set of all collections of $\pin_+(2)$-framings (one at each endpoint)
which cover an element of $cf(r)$.
At each endpoint, there are two possible lifts of an $O(2)$ framing, 
so $\tcf(r)$ is a set with $2^{k+1}$ elements.
Again, $\tcf(r)$ does not change if we reverse the spin structure on $B$.\footnote{
Reversing a spin structure is analogous to reversing the orientation of an oriented manifold.
One way to define it is to extend the Spin bundle $E$ to a Pin+ bundle $E_+$;
the reversed Spin bundle is $E_+ \setminus E$.
}

Let $S(r)$ denote the set of all orderings of the endpoints of $r$.
It is a set with $k!$ elements.
Now consider $\tcf(r) \times S(r)$, yet another set associated to a ribbon endpoint configuration $r$.
This set has a group of symmetries $G$ which is generated by (a) spin flips acting on any single endpoint, 
and (b) the symmetric group $S_k$, acting on the $S(r)$ component.
Let $G_e \subset G$ be the subgroup with even total parity, where 
parity in this context is defined by the homomorphism from $G$ to $\zz/2$
characterized by the condition that is odd for single spin flips and transpositions.
Now, finally, define
\be
	P(B)_r = (\tcf(r) \times S(r))/G_e .
\ee

We claim that $P(B)_r$ is naturally a torsor for $\pin_+(1)$.
First, let's check the cardinality:
$\tcf(r) \times S(r)$ has $2^{k+1} k!$ elements,
$G_e$ has $2^{k-1} k!$ elements, and $G$ acts freely on $\tcf(r) \times S(r)$.
Therefore $P(B)_r$ has four elements.

Now we define the action of $\pin_+(1)$ on $P(B)_r$.
The spin flip acts by changing (any) one of the spin framings of $\tcf(r)$ by a spin flip.
The reflection acts diagonally on all of the spin framings of $\tcf(r)$
(i.e.\ each spin framing is reflected).
These two actions are well-defined and commute, so we have
an action of $\pin_+(1) \cong \zz/2\times\zz/2$.
(If $k = 0$ and $r \in \mcr(B)_0$ is the unique configuration of zero ribbon endpoints,
we define $P(B)_r = \pin_+(1)$ and let $\pin_+(1)$ act in the obvious way.)

\medskip

In summary, an element of $F(B)_r$ is represented by a triple $(f, o, z) \in \tcf(r) \times S(r) \times \cc$.
Spin flips, permutations of ribbon endpoints, and reflections act as
\begin{itemize}
\item If $f$ and $f'$ differ by a spin flip at a single ribbon endpoint, then $(f, o, z) = (f', o, -z)$.
\item If $o$ and $o'$ differ by an odd permutation, then $(f, o, z) = (f, o', -z)$.
\item If $f$ and $f'$ differ by reflecting all of the spin framings, then $(f, o, z) = (f', o, \bar z)$.
\end{itemize}

To define complex multiplication by $a\in\cc$ on $F(B)_r$, we choose a collection of framings $f$ which is positive
with respect to the orientation of $B$ and then define $a\cdot (f, o, z) = (f, o, az)$.
If we were to reverse the orientation of $B$, then we would get the conjugate complex structure on $F(B)$.
In other words, $F(B)_r = F(-B)_r$ as sets (and even as vector spaces over $\rr$), but the identity map from
$F(B)_r$ to $F(-B)_r$ is complex antilinear.

\medskip

We began this subsection with a list of several desiderata for $F(B)$.
It is more or less obvious that $F(B)$ has the right sort of functoriality for both orientation-preserving
and orientation-reversing spin/pin maps.
It should also be clear that $F(B)$ has the desired holonomies for rotations and exchanges.
So all that's left to discuss is locality (gluing) and ribbon endpoint cancellation.

\medskip

We consider locality first.
Let $B = B_1 \cup B_2$.
Let $r_i \in \mcr(B_i)$ and let $(f_i, o_i, z_i)$ represent an element of $F(B_i)_{r_i}$.
If the spin framing collections $f_1$ and $f_2$ are either both positive or both negative, then
$f_1 \cup f_2$ is a consistent spin framing in $\tcf(r_1 \cup r_2)$, and the triple
$(f_1 \cup f_2, o_1\cup o_2, z_1 z_2)$ represents an element of $F(B)_{r_1\cup r_2}$.
It is easy to check that this map gives a well-defined  isomorphism between $u^*(F(B_1)$ 
and $F(B')\otimes F(B_2)$, both thought of as line bundles over $\mcr(B_1)\times\mcr(B_2)$.

\medskip

Now for cancelations.
We want a relation of the type 
\be \label{eval_psi_semicirc}
\PsiEnd  = \lambda \times \text{(vaccuum)},
 \ee
for some $\lambda \in \cc$. 
On the left hand side we have two ribbon endpoints in a disk $D \subset B$ connected by a ribbon in $D\times I$.
We have chosen coordinates so that the front of the ribbon always points in the same direction.
The spin framings at the two endpoints are chosen to be related by a translation in these coordinates
and to both be positive.
We have chosen the ordering so that the second vector at endpoint 1 points toward endpoint 2;
we will call this a ``standard configuration".
As indicated, we want this standard picture to be equal to $\lambda$ times the empty picture.

We will show that in order for this relation to be compatible with reflections, 
we must have that $\lambda = -\bar\lambda$, i.e.\ $\lambda$ must be pure imaginary.
Note that in order to define the action of a reflection on string-nets, it is essential that we have defined an 
action of $\pin_+(1)$ on $P(B)_r$ (rather than merely an action of $\spin(1)$). 
(A $\pin_-(1)$ structure would also work, but our examples happen to have $\pin_+$ rather than $\pin_-$ structures.)
The existence of a reflection structure also allows us to define inner products of diagrams. 


Consider first the RHS of \eqref{eval_psi_semicirc}. 
Reflections take the empty picture to the empty picture, and so, since reflections act antilinearly on $F(B)$,
the RHS of \eqref{eval_psi_semicirc} maps under reflection to $\bar\lambda$ times the empty picture.

Now for the LHS of \eqref{eval_psi_semicirc}.
After a reflection (by which we mean an orientation-reversing map), 
the framings are no longer positive, and so we must reflect them in order to compare to a standard configuration
in the target manifold.
After the framings are reflected, the second vector at endpoint 1 points away from endpoint 2,
so we must swap the ordering in order for it to be a standard configuration.
This change of orderings means that under a reflection, a standard configuration map to $-1$ times a standard
configuration.
It follows that we must have $\lambda = -\bar\lambda$, and so $\lambda$ must be purely imaginary.

One can show that this cancellation relation satisfies the necessary coherence relations.

