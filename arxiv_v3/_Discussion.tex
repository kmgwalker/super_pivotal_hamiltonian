%auto-ignore
%      this ensures the arxiv doesn't try to start TeXing here.
%!TEX root = super_lattice_models_draft.tex
%      prev line helps TeXShop do the right thing

%%%%%%%%%%%%%%%%%% 
\section{Outlook} \label{discussion}
%%%%%%%%%%%%%%%%%%


One potentially interesting aspect of the fermionic topological orders we have studied in this paper is their possible quantum information 
applications, which we now briefly speculate on. 
We consider a hybrid system with spin structure defects 
and deconfined anyonic excitations.
Each spin structure defect harboring a q-type vortex admits an action by $\text{End}(q)$, 
and so $n$ such defects admit an action of $\End(q_1) \tp \End(q_2) \tp \cdots \tp \End(q_n) \cong \cliff_1^{\tp n} \cong \cliff_n$.
One could then imagine utilizing this action to perform quantum computations. 
Physically, the action of $\cliff_n$ is implemented by choosing pairs of vortices $q_i$ and $q_j$ and pumping a charge into $q_i$ and out of $q_j$.
A natural platform for pumping charge through the q-type vortices is a Kitaev chain.
Of course, in addition to the action of $\cliff_n$, computations can also be performed with the conventional braiding of the deconfined quasiparticles appearing in $q_1 \tp q_2 \tp \cdots \tp q_n$.

It would be useful to make more precise connections between physical entities and some of the mathematical 
devices we have used to construct the fermionic theories 
we have studied. 
In particular, it would be useful to clarify the precise physical meaning of the complex line bundle 
and the ``back wall'' that we use to perform fermion condensation. 
In simple examples like the $C_2$ theory, the most natural interpretation for these constructs 
seems to be that they constitute a topological p-wave superconductor.
Indeed, the way we treat the physical fermions we use to perform condensation 
in our models is identical to the behavior of 
superconductors: they are 
free fermion states, 
where wavefunctions that differ by pairs of 
fermions are related by a phase. 
The specialization to the p-wave pairing channel is made because of the spinlessness
of the fermions we use to induce the condensation, which we assumed from the very 
beginning.  
The superconducting nature of the devices we use to perform condensation is forced on us
by our assumption that the emergent fermion $\psi$ we condense possesses $\zt$
fusion rules, and that in the complex line bundle we construct, pairs of
$\psi$ worldline endpoints can be created or destroyed in pairs. 
Evidence for the presence of a p-wave superconductor is clearly seen in the $C_2$ theory: 
restricting our attention to the non-bounding torus with $NN$ spin structure, both the modular $S$ and $T$ matrices 
factorize as $S=S_{Ising}\tp S_{p\pm ip}, T = T_{Ising} \tp T_{p\pm ip}$ (where the 
choice of $\pm$ is determined by the ``angular momentum'' of the fermionic dots 
in our graphical calculus, i.e. the choice of $\pm A^4$ when removing a semicircular fermion
worldline), suggesting a possible interpretation of this sector 
as a stack of the original Ising theory with a topological superconductor. 
Indeed, this was noticed recently in \cite{ware2016}.
Furthermore, the fact that the parity of the ground states on the torus with $(N,N)$ spin structure is always 
odd agrees with this interpretation, since the fermion parity of a topological 
p-wave superconductor on such a torus is always odd \cite{you2015}.

In our discussion of the modular $S$ and $T$ matrices in each of the examples we've worked out, 
we have focused on the modular transformation perspective, rather than on the braiding statistics perspective. 
For example, we have computed the $S$-matrix by considering the way it acts to exchange the two 
cycles of the torus and 
we have not focused on the statistical picture behind the $S$-matrix, in which matrix elements $S_{ab}$ correspond to double braids between $a$ and $b$ particles. 
While we have checked that the computation of double braids reproduces the correct $S$-matrix for $\tube(C_2)$,
some subtleties involving relative spin structures rear their heads when trying to compute particular braiding data in more general settings. 
We plan to address these subtleties in future work. 

 \paragraph{Acknowledgements}

Ethan Lake and Dave Aasen are grateful to Nick Bultinck, Nicolas Tarantino, Ryan Thorngren, Brayden Ware, and Dominic Williamson for helpful discussions.
Dave Aasen thanks Parsa Bonderson for explaining his unpublished work at early stages of this project. 
Dave Aasen gratefully acknowledges support from the KITP Graduate Fellows Program, the National Science Foundation through grant DMR- 1723367
and the Caltech Institute for Quantum Information and Matter, 
an NSF Physics Frontiers Center with support of the Gordon and Betty Moore Foundation through Grant GBMF1250.
Ethan Lake is supported by the Fannie and John Hertz Foundation.
Dave Aasen and Ethan Lake acknowledge support by the 2016 Boulder Summer School for Condensed
Matter and Materials Physics through NSF grant DMR-13001648.
Kevin Walker thanks Zhenghan Wang and Scott Morrison for helpful conversations, and
thanks the Aspen Center for Physics and  the Mathematisches Forschungsinstitut Oberwolfach
for providing stimulating research environments.

