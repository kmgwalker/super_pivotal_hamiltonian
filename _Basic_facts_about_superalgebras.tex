%auto-ignore
%      this ensures the arxiv doesn't try to start TeXing here.
%!TEX root = super_lattice_models_draft.tex
%      prev line helps TeXShop do the right thing

%%%%%%%%%%%%%%%%%%%%%%%
\section{Basic facts about super algebras} \label{superstuff}
%%%%%%%%%%%%%%%%%%%%%%%

In this appendix we briefly recall some of the key mathematical facts about semisimple super algebras; 
details can be found in \cite{jozefiak1988}.

There are two distinct classes of simple super algebras over $\cc$. One class is the set of 
super algebras $M(r|s)$ for $r,s\in\zz_{\ge 0}$, which are $(r+s)\times(r+s)$ matrices whose even and odd subspaces take the form
\be 
	M(r|s)_0 = \text{matrices of the form\ } \begin{pmatrix} A & 0 \\ 
	0 & B \end{pmatrix},\qquad M(r|s)_1 = \text{matrices of the form\ } \begin{pmatrix} 0 & C \\ D & 0 \end{pmatrix}.
\ee
In these expressions, $A$ is an $r\times r$ matrix, $B$ an $s\times s$ matrix, $C$ an $r\times s$ matrix, and $D$ an $s\times r$ matrix. 

The other class of simple super algebras are denoted by $Q(n)$ for $n\in \zz_{>0}$, 
which are $(2n)\times(2n)$ matrices with even and odd subspaces of the form
\be 
	Q(n)_0 = \text{matrices of the form\ } \begin{pmatrix} A & 0 \\ 0 & A \end{pmatrix},
	\qquad Q(n)_1 = \text{matrices of the form\ } \begin{pmatrix} 0 & B \\ B & 0 \end{pmatrix},
\ee
where both $A$ and $B$ are $n\times n$ matrices. 
In particular, $Q(1) = \langle \unit, \sigma^x\rangle$ is the first complex Clifford algebra $\cliff_1$. 

Note that all of the $M(r|s)$ are Morita equivalent to the trivial algebra $M(1|0) \cong \cc$.
All of the $Q(n)$ are Morita equivalent to $Q(1) \cong\cliff_1$.

If $x$ is an object in a super pivotal category, then $\End(x)$ will be isomorphic to a direct sum of instances of $M(r|s)$ and $Q(n)$.

The form of a general simple super algebra $A$ can be deduced by computing the center $Z(A)$.
(An element $a$ is in $Z(A)$ if it super commutes with everything in $A$, 
i.e.\ if $ax = (-1)^{|a||x|}xa$ for all $x\in A$, with $|x|$ denoting the parity of $x$.)
Since the super algebras $Q(n)$ treat even and odd vectors symmetrically, 
they expect that they will have odd elements in their center, while this will not be true for the super algebras $M(r|s)$. 
Indeed, we have that if $Z(A) \cong \cc^{1|0}$ then $A \cong M(r|s)$ for some $r,s$, while if $Z(A) \cong \cc^{1|1}$ then $A\cong Q(n)$ for some $n$. 

If $A,B$ are super algebras, their tensor product $C=A\tp B$ is defined as the super algebra such that $C_0 = A_0\tp B_0 + A_1\tp B_1$, $C_1 = A_0\tp B_1 + B_1\tp A_0$. 
The simple super algebras presented above can be tensored together by using the following rules:
\be \begin{aligned} 
	M(r|s) \tp M(p|q) & \cong M(rp+sq|rq+sp) \\ 
	M(r|s) \tp Q(n) & \cong Q(rn+sn) \\ 
	Q(n) \tp Q(m) & \cong M(nm | nm). 
\end{aligned} \ee
Note that all the $Q(n)$ can be generated from $M(n|0)$ and $Q(1)$ according to $Q(n) \cong M(n|0)\tp Q(1)$. 

