


%%%%%%%%%%%%%%%%%%%%%%%
\section{Basic facts about superalgebras} \label{superstuff}
%%%%%%%%%%%%%%%%%%%%%%%

In this appendix we briefly recall some of the key mathematical facts about superalgebras; details can be found in \cite{jozefiak1988}.

There are two distinct classes of simple superalgebras over $\cc$. One class is the set of superalgebras $M(r|s)$ for $r,s\in\zz$, which are $(r+s)\times(r+s)$ matrices whose even and odd subspaces take the form
\be M(r|s)_0 = \text{matrices of the form\ } \begin{pmatrix} A & 0 \\ 0 & B \end{pmatrix},\qquad M(r|s)_1 = \text{matrices of the form\ } \begin{pmatrix} 0 & C \\ D & 0 \end{pmatrix}.\ee
In these expressions, $A$ is an $r\times r$ matrix, $B$ an $s\times s$ matrix, $C$ an $r\times s$ matrix, and $D$ an $s\times r$ matrix. 

The other class of simple superalgebras are denoted by $Q(n)$ for $n\in \zz$, which are $(2n)\times(2n)$ matrices with even and odd subspaces of the form
\be Q(n)_0 = \text{matrices of the form\ } \begin{pmatrix} A & 0 \\ 0 & A \end{pmatrix},\qquad Q(n)_1 = \text{matrices of the form\ } \begin{pmatrix} 0 & B \\ B & 0 \end{pmatrix},\ee
where both $A$ and $B$ are $n\times n$ matrices. In particular, $Q(1) = \langle \unit, \sigma^x\rangle$ is the first complex Clifford algebra $\cl_1$. 
If $s$ is a simple object in a superfusion category, then $\End(x)$ will be a simple superalgbera. If $\End(x)\cong Q(n)$ for some $n$, then since the superalgebra $Q(n)$ treats even and odd vectors 
on the same footing, $x$ will be invariant under odd morphisms. 
For a fermionic theory in which $\psi$ has been condensed, this translates to the fusion rule $x\tp\psi \cong x$ (such a fusion rule is not possible if $\End(x)\cong M(r|s)$ instead). 
Furthermore, since $\End(x)$ must actually be a division algebra, then either $\End(x) \cong M(1|0) \cong \cc$ or $\End(x) \cong Q(1)$. 
This is our reason for calling simple objects in super fusion categories m-type if their endomorphism algebra is $\cc$, and q-type if their endomorphism algebra is $Q(1)\cong\cl_1$. 

The form of a general simple superalgebra $A$ can be diagnosed by computing the odd subspace $Z(A)_1$ of the center $Z(A)$ (an element $a$ is in $Z(A)$ if it supercommutes with everything in $A$, i.e. if $ax = (-1)^{|a||x|}xa$ for all $x\in A$, with $|x|$ denoting the parity of $x$).
Since the superalgebras $Q(n)$ treat even and odd vectors symmetrically, they expect that they will have odd elements in their center, while this will not be true for the superalgebras $M(r|s)$. 
Indeed, we have that if $Z(A)_1 = 0$ then $A \cong M(r|s)$ for some $r,s$, while if $Z(A)_1 \neq 0$ then $A\cong Q(n)$ for some $n$. 

If $A,B$ are superalgebras, their tensor product $C=A\tp B$ is defined as the superalgebra such that $C_0 = A_0\tp B_0 + A_1\tp B_1$, $C_1 = A_0\tp B_1 + B_1\tp A_0$. 
The simple superalgebras presented above can be tensored together by using the following rules:
\be \begin{aligned} M(r|s) \tp M(p|q) & \cong M(rp+sq|rq+sp) \\ 
M(r|s) \tp Q(n) & \cong Q(rn+sn) \\ 
Q(n) \tp Q(m) & \cong M(nm | nm). \end{aligned}
\ee
Note that all the $Q(n)$ can be generated from $M(n|0)$ and $Q(1)$ according to $Q(n) \cong M(n|0)\tp Q(1)$. 

%\dave{I liked this comment.}
%Then we have an isomorphism 
%\be \End(A) \tp \End(B) \ra \End(A\tp B),\ee
%which is helpful when computing fusion rules. 
