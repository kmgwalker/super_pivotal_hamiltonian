%auto-ignore
%      this ensures the arxiv doesn't try to start TeXing here.
%!TEX root = super_lattice_models_draft.tex
%      prev line helps TeXShop do the right thing

%%%%%%%%%%%%%%%%%%%%%%%
\section{Introduction}
%%%%%%%%%%%%%%%%%%%%%%%

%\dave{Lets not forget to cite \cite{Walker2014,Walker2015} in the intro.}
%\ethan{done}

A large program in condensed matter physics in recent years has been to classify 
topological phases of matter which host emergent quasiparticle excitations 
with topological properties like exotic braiding statistics. 
The mathematical framework of category theory has proven to provide the right 
framework needed to formally develop classification efforts \cite{kitaev2006}. 
Thus far, the bulk of these efforts have focused on understanding bosonic topological phases.
These phases are ``bosonic'' because the topological excitations emerge from bosonic 
local degrees of freedom.

\dave{Should probably get the historical order correct here. 
Something like: \cite{gu2015,gu2014} first, then KW's talks, Kapustin crowd, other intermediate fermion papers (e.g., \cite{Lan2016b}), then \cite{tarantino2016,ware2016},
and then math papers \cite{usher2016,brundan2016,bruillard2017,bonderson2017},
and maybe the old math papers (I'm not so familiar with what they were doing) e.g., Kirby-Melvin/Sawin/\cite{beliakova1998}.
I tried doing some re-writes but was failing so decided to move on.
}
By contrast, much less is known about how to complete a classification program for {\it fermionic} topological phases, in which fermions constitute 
the underlying degrees of freedom. 
Progress on understanding the coherence relations used to classify fermionic topological phases and the identification of a class of examples of such phases was made in \cite{gu2015,gu2014,Lan2016b}. 
Recently, Majorana dimer lattice models have appeared \cite{tarantino2016,ware2016}, 
which give an explicit Hamiltonian construction of a 
non-trivial
fermionic phase closely related to the Ising theory. 
There has also been some recent work in the math community \cite{usher2016,brundan2016,bruillard2017,bonderson2017} 
devoted to studying the formal category-theoretic description of fermionic topological phases and spin-TQFTs. 

Intimately related to the description of fermionic topological phases is the concept of fermion condensation, 
whereby emergent (anyonic) fermions are driven through a topological phase transition to become 
local particles (meaning that they can be created and destroyed locally). 
\dave{This is a nice sentence. But subtle. 
Is the idea is that 'condensing' something is the same as saying it's a local excitation?
Also are you considering physical fermions as local?
We could add a footnote which describes this sentence in more detail....
Also for the phase transition bit we can cite \cite{Bais2009}.}
\ethan{Right. I'm saying it means you can create/destroy something in a condensate locally. Feel free to add whatever explanatory remarks you like}
\kw{We don't talk about phase transitions in the body of the paper, so perhaps it does not make sense to mention them here}
\dave{True, and I would prefer to keep it that way (meaning no discussion of phase transitions in the body of the paper). 
(i.e., they don't show that anyon condensation generically comes with a 2nd order phase transition in the conventional sense, but rather stipulate it from (as far as I can tell) well established folk lore, that seems to be true.
But this is also a point I never really get hung up on since Bais and Slingerland go on to say what they mean by `condensate induced transition'.)}
\dave{I think the vote is to remove any mention of phase transition.
Though we should keep some of the content: 
Anyon condensation/adding new local operators (e.g., interactions).
Fermion condensation same story.
Hamiltonian must preserve fermion parity (contrast to bosonic picture). 
Physical perturbation is not free fermion.
}
There have been several recent approaches related to understanding 
fermionic topological phases with field theoretic methods, fermion condensation, and bosonization \cite{gaiotto2016, bhardwaj2016, bhardwaj2016b,kapustin2017,putrov2016}, 
with a more algebraic take on fermion condensation given in \cite{wan2016}. 

In this work, we perform a systematic study of fermionic topological phases and fermion condensation from a category-theoretic
point of view, taking the approach of \cite{Walker2014,Walker2015}. 
\kw{Are these supposed to be two nonidentical citations?}
\dave{Mhmmm. One was supposed to be Walker2015 probably. 
The first is your Princeton talk, the second is IPAM. 
It's good to include the second since there is a link to the recording of your talk.
I should probably also fix the references to some more standard way citing talks.}
The strategy we will use to construct examples of fermionic topological phases will be to start with 
a bosonic phase described by a tensor category $\mcc$
which contains an emergent fermion $\psi$.
To obtain a fermionic theory, we will condense $\psi$.
That is, we will show how to introduce a phase transition 
\kw{??}
in the bosonic phase 
described by $\mcc$ to a phase where $\psi$ is a {\it local} (as opposed to an anyonic) excitation.
The resulting fermionic topological phase is described by a super tensor category, which we denote
as $\mcc / \psi$.
$\mcc/\psi$ is a fermionic phase because in order for a phase to support local fermionic excitations, 
its underlying degrees of freedom must be fermionic. 

Due to their nontrivial spin and statistics, condensing fermions is not a straightforward business. 
From a mathematical perspective, in order to perform the condensation it is necessary to equip the configuration 
space of $\psi$ worldline endpoints with a certain complex line bundle. 
Physically, the construction of this bundle amounts to attaching a phase of physical (not emergent) fermions $f$ 
to the parent bosonic theory: 
fermion condensation then schematically proceeds by coupling the $\psi$ fermions to 
the $f$ fermions, and condensing $\psi f$ bound states. 
In most cases (when $\psi$ is not transparent in $\mcc$), we will also need to perform 
the condensation with a construction we call the ``back wall'', which is a codimension-1 surface 
on which $\psi$ worldlines are allowed to end. 

\kw{I think we need to be clearer when talking about simple objects of $\mcc$ (or $\mcc/\psi$)
versus simple objects of Tube($\mcc$) (or Tube($\mcc/\psi$)).
Are we calling both of these ``anyons"?}
\ethan{we've waffled a bit on the terminology. I've been calling all of the stuff in the tube category as ``quasiparticles'' because this is unambiguous if a bit long to write out. I'm fine with calling things in $\mcc$ ``objects'' or ``string-net labels'' and things in $\tube(\mcc)$ as ``anyons'', though.}
\ethan{any recent thoughts on terminology? I'm fine with what we're doing currently}
\dave{Just saw this now. 
I think most of the time when we say `quasiparticle' in the paper we are referring to simple objects/minimal idempotents in $\tube(...)$.
I would lean toward continuing with this terminology and not referring to simple objects of $\mcc$ as quasiparticles (the only `physical' particles we talk about in this paper are those of the string-net models, which all live in $\tube(...)$).
}

An important difference between fermionic topological phases and their bosonic counterparts is that 
the former possesses two distinct classes of simple objects (or anyons). 
One class of simple objects, which we refer to as ``m-type'' objects, are identical in character
to the simple objects found in bosonic theories. 
The other class of objects, dubbed ``q-type'' objects, have no bosonic analogues. 
From a formal perspective, they are distinguished by their nontrivial endomorphism algebras---if 
$a$ is a q-type simple object, then $\End(a) \cong \cliff_1$, the first complex Clifford algebra. 
From a physical point of view, these can be thought of as ``Majorana objects'', which have the ability 
to ``absorb'' fermions. 
In string-net constructions, strings labeled by q-type objects behave like Kitaev wires in the 
topological phase: the fermion parity of a closed loop of such a string is determined by 
the spin structure inherited by the string, and is delocalized in the sense that fermions living on the q-type string are allowed to fluctuate freely along its length.
Thus, q-type strings provide contributions to the fermion parity operator that cannot be written down 
in a local way. 
\kw{``contribution to parity operator"?}

One rather trivial way to produce a fermionic phase is to simply form a non-interacting stack of a phase of physical fermions 
and a known bosonic topological phase. 
To obtain examples of fermionic phases that do not arise in this way (which are called ``primitive'' in \cite{Lan2016b}),
it is essential to examine theories that contain q-type objects, which are fundamentally fermionic in nature. 
\kw{do we want to say that q-tyope simple objects do not arise in this way?}

A categorical description of the condensed phase $\mcc / \psi$ cannot be obtained within the framework 
of regular tensor categories, and one needs to instead adopt the framework of ``super pivotal tensor categories''. 
The fermionic nature of $\mcc / \psi$ means that the fusion spaces of the theory become supervector spaces 
(as opposed to normal vector spaces),
the tensor product of morphisms is modified to become supercommutative rather than commutative\dave{`maybe modify to `the tensor product of morphisms satisfy the super interchange law \cite{Usher?}...'}, 
coherence relations like the pentagon identity are modified to incorporate Koszul signs, and so on.

We should stress that all of the constructions we employ in this paper are mathematically well-defined and 
self-contained, independent of their physical interpretations. 
Most of the paper uses techniques from category theory, TQFTs and string nets to construct what one
might call fermionic Turaev-Viro theory, and this paper is mostly about fermionic TQFTs from this perspective, 
rather than studying the ground states of Hamiltonians. 
Indeed, not until near the end of the paper do we define a fermionic Levin-Wen style Hamiltonian whose ground states coincide with 
the Hilbert spaces of this TQFT.
Foreshadowing the introduction of this Hamiltonian, we will, throughout the paper, talk about physical notions like ``excitations", "ground state degeneracies", 
etc, even though
strictly speaking this point of view does not make sense until after we have introduced the Hamiltonian.
We emphasize that the Hamiltonian point of view is optional; most of the paper can be viewed as taking place
in the self-contained world of (fermionic) string-net TQFTs.

In order to examine the quasiparticle excitation spectrum of fermionic phases, 
we use the tube category construction \cite{ocneanu1994}\footnote{Again, it is technically imprecise to call the objects 
of the tube category ``excitations'' until after we have carefully defined a Hamiltonian.}.
The construction of the tube category is modified for fermionic theories, and produces a theory
equivalent to a fermionic version of Drinfeld center of the input category.
For bosonic theories we have the isomorphism $\tube(\mcc)\cong\mcc\times\overline{\mcc}$ \cite{muger2003b}, where $\overline{\mcc}$ denotes the opposite category of $\mcc$. 
We prove fermionic version of this isomorphism, namely 
that the tube category $\tube(\mcc/\psi)$ for a fermionic theory can be obtained from $\mcc$ through
\be 
	\tube(\mcc / \psi) \cong \mcc \times (\overline{\mcc / \psi})
\ee
In terms of Drinfeld centers, this can be read as $\mcz(\mcc / \psi)\cong \mcz(\mcc)/\psi$, where on the right-hand side, 
$\psi$ represents an embedding of $\psi$ into $\mcz(\mcc)$.


Because quasiparticle excitations are supported on circular boundary components of the ambient 
manifold on which the theory is defined, there are two distinct classes of quasiparticles (a.k.a.\ objects in the tube category) in 
fermionic theories: those with anti-periodic fermion boundary conditions around the circle, and 
those with periodic boundary conditions. 
Thus, the tubes in the tube category come equipped with spin structures, which gives the tube category a $\zt$ grading
and separates the quasiparticle spectrum in to vortex quasiparticles (those which bind spin structure defects) 
and non-vortex quasiparticles (which are similar in character to the quasiparticles present in bosonic theories). 






The analysis of things related to the tube category like modular transformations, braiding statistics, and the computation of ground state degeneracies on various spin surfaces
is also modified in the fermionic setting.  
The behavior of the $S$ and $T$ modular transformations on the torus depends on the spin structure of the torus in question, and 
certain relations like $(ST)^3 = \id$ are modified in the fermionic setting, becoming 
$(ST)^3=(-1)^F\id$ where $(-1)^F$ is the fermion parity operator.
%Unlike in bosonic phases, the matrix of double braids (mutual statistics) is not identical to the 
%modular $S$-matrix, although complete information about the double braids 
%is enough to reconstruct the modular $S$-matrix. 

Next, \kw{??} we show how to construct a commuting projector lattice Hamiltonian 
for fermionic phases, whose excitation spectrum is given by the objects in the fermionic tube category.  
It is similar in character to the Levin-Wen Hamiltonian \cite{levin2005}, 
except that it includes an extra edge term which is responsible for moving fermions along
edges labeled by q-type objects. 
Depending on the form of the edge term, spin structure defect excitations 
are realized as violations of this edge term, and are linearly confined. 

Finally, we show how to construct a tensor network which produces the ground state 
wavefunction of our lattice Hamiltonian. We do this using a cut-and-glue approach to 
the construction of the Turaev-Viro-Barrett-Westbury state sum, which allows us to write the bosonic partition function
as a tensor contraction. We then show how to modify the tensors 
and the way we contract them in order to obtain a fermionic version of the TVBW state sum.  

We illustrate all \kw{??} of the above points in a series of several in-depth examples, which occupy a large portion 
of this paper.
Our presentation is characterized by an emphasis on examples, 
and we often save formal definitions and more general statements 
for later sections after relevant examples have been presented.  
Many general observations are mixed in with the example sections,
and we stress that the more general sections are intended as a supplement
the example sections; they are not stand-alone.
We also note that a more mathematical presentation of this work will appear in \cite{Kevin and scott's paper?}. 


Due to the length of this paper, we will quickly give a summary of the remainder \kw{??} of its contents.
Sections \ref{C2_condense_sect} and \ref{C2_quasiparticles} are devoted to a detailed study of 
fermion condensation in the Ising TQFT, which is a simple case study that allows us to build intuition for the condensation procedure.
This condensed theory has been examined before in \cite{bhardwaj2016, kapustin2017}; 
here we examine it in greater detail. 
In Section \ref{C2_condense_sect} we review the Ising TQFT, describe how to perform the 
fermion condensation, and write down the local string-net rules in the condensed theory, 
which we call the $C_2$ theory. 
In Section \ref{C2_quasiparticles} we use a fermionic generalization of the tube category 
construction to study the excitation spectrum of this model, compute its ground state 
degeneracy on the torus, and study the modular transformations %and braiding data 
of the theory. 

In Section \ref{generalities}, we present a collection of more general results
on the machinery of fermionic theories. 
We show how to perform fermion condensation in more general settings
and discuss our fermionic generalization 
of the tube category construction in more generality.
We also discuss how to compute quantum dimensions and fusion rules,
and we prove some general
relations between the total quantum dimensions of condensed theories, their 
tube parent theories, and their tube categories.
Additionally, we present a Verlinde-type
dimension formula for fermionic theories.

We devote Section \ref{more_on_tubes} to a detailed study of the tube category 
for fermionic theories which result from fermion condensation in a modular tensor category (MTC). 
We introduce several tools for performing calculations in tube categories, 
and use them to prove that if $\mcc$ is an MTC containing a fermion $\psi$, then
$\tube(\mcc/\psi)\cong \mcc\times (\overline{\mcc/\psi})$ as tensor 
categories. We also describe a way of easily computing the $S$-matrix for 
$\tube(\mcc/\psi)$. 

Section \ref{so36} is devoted to the example of performing condensation in the $SO(3)_6$ 
and $SU(2)_6$ theories, while Section \ref{halfesix} discusses fermion condensation in the $
\halfesix$ theory. 
Both of these examples are more involved than the $C_2$ theory, 
and illustrate some of the more interesting features of generic fermionic topological phases. 

In Section \ref{def_sect} we define and discuss super pivotal categories from a more formal point of view. 
We show how fermionic fusion spaces are constructed and tensored together, how 
coherence relations like the pentagon identity are modified in the fermionic case, 
and how the fermionic nature of the vector spaces used in the construction of super 
pivotal categories affects their construction. 

Section \ref{Super_pivotal_Hamiltonian} is devoted to a discussion of a lattice Hamiltonian for fermionic theories, which is a generalization of the Levin-Wen Hamiltonian. 
It differs most significantly from the Levin-Wen Hamiltonian its possession of an edge term, which 
is responsible for allowing fermions to fluctuate across q-type strings and the violation 
of which produces linearly confined defects. 

We show how to construct a tensor network realizing the ground state of our lattice Hamiltonian 
in Section \ref{state_sums}. We first review how to write the partition function of a bosonic theory 
as a tensor contraction in a way amenable to generalization, and then show how our construction 
can be extended to cover the fermionic case. 

In Section \ref{Kitaev_wire}, we discuss the Kitaev chain within the framework of super pivotal 
categories. 
We show how the Kitaev chain Hamiltonian and ground-state wavefunctions can be 
succinctly written down using the diagrammatic calculus developed earlier in the paper, and 
discuss connections between the Kitaev chain and the $C_2$ theory. 
This section also provides a connection between our work 
and recent work on fermionic topological phases in the physics community \cite{ware2016,tarantino2016,turzillo2016}. 
\ethan{do we want something like the last sentence?} 

Finally, we end with a conclusion and discussion in Section \ref{discussion}. 
Several appendices contain miscellaneous results and background information. 


\bigskip

\kw{to do: mention fermionic 2+1 d TQFT; talk citations}


\kw{Probably should mention Steve Sawin's work on spin TQFTs somewhere}
\dave{Is this the paper you were thinking of \cite{Sawin2002}. Looks like he has a couple of interesting papers. 
It reminded me of the Kirby-Melvin paper, should we also cite them somewhere?}

\kw{I have been putting footnotes before the period at the end of a sentence (because doing the reverse looked weird and wrong to me),
but a small amount of googling suggests that the usual convention is to put the footnote after the period.}






\subsection{Table of notation}

In partial compensation for the unwieldy length of this paper, we give a table summarizing some of the 
most frequently used and/or least standard notation found herein.

\dave{Do we have a name for a generic net configuration, 
not necessarily modulo local relations?}

\medskip

\tabulinesep = 2mm
\begin{longtabu}{|X[1,$c]|X[4,l]|}
	\hline
    \mcc 	& tensor category \\ \hline
    \mcc/\psi 	& fermionic quotient of $\mcc$ \\ \hline
    \spc 	& super pivotal tensor category, e.g.~$\mcc/\psi$ \ethan{are we totally sure that they always come from condensation? I am 99\% sure this is true but I didn't know if we had a proof or not} \dave{I don't think we are sure.} \kw{it's e.g., not i.e.  No implication that all arise this way.} \ethan{Ah yes, apologies for mis-reading} \\ \hline
    \scat 	& super linear category \\ \hline
    \sob(\mcc) 	& complete set of representatives of simple objects for $\mcc$ \\ \hline
	\mor(x \to y)	& morphisms from $x$ to $y$ \\ \hline
    \End(x)	& endomorphism algebra of the object $x$ \\ \hline
    \cliff_1	& first complex Clifford algebra \\ \hline
    \text{m-type}	& simple object with $\End(x)\cong \cc$ \\ \hline
    \text{q-type}	& simple object with $\End(x)\cong \cliff_1$ \\ \hline
    n_x		& $\dim\End(x)$, 1 if $x$ is m-type and 2 if $x$ is q-type \\ \hline
	\cl(x) & closure of $x$ in either an annulus or torus depending on context \\ \hline
	\cl_W(x) & closure of $x$ with spin structure $W$ assigned to new cycle \\ \hline
	f\cdot g	& the composition of two morphisms $x \xrightarrow{f} y \xrightarrow{g} z$ (arrow order) \\ \hline
	g \circ f	& the composition of two morphisms $x \xrightarrow{f} y \xrightarrow{g} z$ (function order) \\ \hline
	A(Y)	& string nets modulo local relations on a 2-manifold $Y$; predual Hilbert space \\ \hline
	\mch(Y)	& same as $A(Y)$ \\ \hline
	Z(Y)	& functions on string nets invariant under local relations on a 2-manifold $Y$; dual space $A(Y)^*$; Hilbert space \\ \hline
	Z(M)	& path integral of a 3-manifold $M$ \\ \hline
	Z(M)(c)	& path integral of a 3-manifold $M$, evaluated on a boundary condition $c\in A(\bd M)$ \\ \hline
    \tube(\mcc)	& tube category of $\mcc$ \\ \hline
    \tube^B(\mcc)	& bounding tube category \\ \hline
    \tube^N(\mcc)	& non-bounding tube category \\ \hline
	\tube_{x \to y}	& morphisms from $x$ to $y$ in the tube category \\ \hline
	B, N	& bounding, nonbounding spin structures on the circle \\ \hline
	S^1_B	& spin circle with bounding spin structure \\ \hline
	S^1_N	& spin circle with nonbounding spin structure \\ \hline
	\unit	& tensor unit in a tensor category; trivial object \\ \hline
	d_a		& quantum dimension of the object $a$; loop value of $a$ \\ \hline
	\theta_a		& twist eigenvalue of the simple object $a$ \\ \hline
	\mcd^2	& $\sum_{a\in \sob(\mcc)} d_a^2$ in bosonic case; $\sum_{a\in \sob(\mcc)} d_a^2/n_a$ in fermionic case \\ \hline
\end{longtabu}


%\begin{align}
%%\scat &= \text{super linear category}\\
%%\mcc &= \text{tensor category}\\
%%\mcc/\psi & = \text{fermionic quotient of $\mcc$}\\
%%\spc & = \text{super pivotal tensor category, e.g.~$\mcc/\psi$}\\
%%\sob(\mcc) & = 
%%\begin{cases} 
%%&\text{complete set of representatives of simple objects for $\mcc$} \\
%%& \text{or a complete set of simple objects for $\mcc$ (depends on context)}\\ 
%%\end{cases}\\
%%n_x = \dim \End(x)
%%\\
%& = \text{\dave{Do we have a name for a generic net configuration, 
%not necessarily modulo local relations?}}\\
%A(Y),\mch(Y) &= \text{string nets modulo local relations on a manifold $Y$}\\
%Z(Y) & = \text{Partition function on $Y$; the linear dual of $A(Y)$}\\
%\\
%x \xrightarrow{f} y \xrightarrow{g} z &\text{ is equivalent to } x \xrightarrow{f \cdot g} z  \text{ is equivalent to } x \xrightarrow{g \circ f} z 
%\\
%\nonumber\\
%%\tube(\mcc) &= \text{tube category of $\mcc$}\\ 
%%\tube^B(\mcc) & = \text{Bounding tube category}\\
%%\tube^N(\mcc) & = \text{Non-bounding tube category} \\
%%\tube_{x \to x} & = \text{a subcategory of $\tube(\mcc)$ with boundary condition $x$}\\
%%\nonumber\\
%%\mor^B (x\to y)&= \text{morphisms of $\tube^B(\mcc)$ with boundary conditions}\\
%%\mor^N (x\to y)&=\text{morphisms of $\tube^N(\mcc)$ with boundary conditions}\\
%%\mor(x \to y)&=\text{morphisms of $\tube(\mcc)$ or $\mcc$ (depends on context)}
%\end{align}

