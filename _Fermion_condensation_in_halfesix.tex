%auto-ignore
%      this ensures the arxiv doesn't try to start TeXing here.
%!TEX root = super_lattice_models_draft.tex
%      prev line helps TeXShop do the right thing



%%%%%%%%%%%%%%%%%%%%%%%%%%%%%%%%%%
\section{Fermion condensation in $\halfesix$} \label{halfesix}
%%%%%%%%%%%%%%%%%%%%%%%%%%%%%%%%%

\dave{Note to self: look at p. 14 of \cite{wan2016}}

In this section we perform fermion condensation in the category $\halfesix$.
This provides an example of a super pivotal category with fusion multiplicity.
After condensation we will obtain a theory with one non-trivial q-type particle, 
which we will denote by $\rho$.
$\rho$ obeys the fusion rule
\begin{align}
\rho \tp \rho = \mathbb{C}^{1|1} \mathds{1} \oplus \mathbb{C}^{1|1} \rho,
\end{align}
%\dave{I think the word `enriched' has some connotation.}
%which is a sort of ``fermion-enriched'' version of the Fibonacci fusion rule
which is similar to the Fibonacci fusion rule but with q-type objects.\footnote{A simpler generalization of the Fibonacci theory 
to the super pivotal case, with one non-trivial q-type particle and fusion space 
$V^{\tilde{\tau} \tilde{\tau}}_{\tilde {\tau}}\cong \mathbb{C}^{1|1}$, does not exist. 
Indeed, suppose $\text{End}(\tau) = \mathbb{C} \ell_1$ and $V^{\tau \tau \tau} \cong \mathbb{C}^{1|1}$.
$\gamma \in \text{End}(\tau)$ denote the odd endomorphism in $V^{\tau\tau\tau}$.
Then we can write down three anti-commuting operators $\gamma \tp \gamma \tp 1$, $\gamma \tp 1 \tp \gamma$ and $1\tp \gamma \tp \gamma$ 
which are each even and as such preserve the grading on $V^{\tau\tau\tau}$. 
However, the even and odd subspaces of $V^{\tau\tau\tau}$ are one dimensional, and so these operators cannot be represented.
Hence a theory with a single q-type particle with Fibonacci-like fusion rules must have nontrivial fusion multiplicity.
In the $\halfesix$ theory we have $V^{\tau\tau\tau} \cong \cc^{2|2}$, which is big enough to represent all three of the above operators.}
The nontrivial fusion spaces are $V^{\rho\rho}_\rho \cong \cc^{2|2}$ and $V^{\rho\unit}_\rho \cong \cc^{1|1}$, 
with the first telling us that the theory has nontrivial fusion multiplicity. 


This theory is richer than the examples we have considered previously, and serves as a good 
case study for the features that appear in phases described by more general super pivotal 
categories. 
These more general features include:
\begin{itemize}
%\dave{I don't think we use `center' in most of the paper.}
%\item The center has a quasiparticle excitation which has a non-bounding spin structure but which is m-type 
%(this also occurs in the $SO(3)_6/\psi$ theory discussed previously).
\item There is a quasiparticle excitation which has a non-bounding spin structure but which is m-type 
(this also occurs in the $SO(3)_6/\psi$ theory discussed previously) and is oddly self dual.
\item The ground state degeneracy on the three spin tori with a bounding cycle (with spin 
structures $BB$, $BN$, and $NB$) is $\mathbb{C}^{3|0}$, and the ground state degeneracy on the 
non-bounding torus (with $NN$ spin structure) is $\mathbb{C}^{1|2}$.
In particular, the fermion parity of a ground state on the torus is not uniquely determined by 
the torus' spin structure (this also occurs in the $SO(3)_6/\psi$ theory discussed previously). 
 \item As a fusion category, there is a fusion rule that takes two q-type particles to another q-type particle
%This provides another example of a theory in which 
\footnote{Note that fusion products of this type cannot appear in the tube category of a fermionic theory since two idempotents in the tube category with non-bounding spin structure must fuse to an idempotent with bounding spin structure, which can never be q-type.
 % However, q-type fusion products cannot appear in the fusion of two q-type particles in the tube
%category of a fermionic theory, since two idempotents in the tube category with non-bounding spin 
%structure must fuse to an idempotent with bounding spin structure,
%(the fusion rules are $\zt$-graded by spin structure), 
%which can never be q-type.
}. 
\end{itemize}

Performing the condensation requires one additional step that did not appear in the previous examples. 
This is because the category $\halfesix$ is not braided, and so doesn't have a fermion to condense.
However, as described in Section \ref{lift_and_condense}, 
it suffices to lift a particle in $\halfesix$ to a fermion in the Drinfeld center of $\halfesix$.
%all we need to perform fermion condensation 
%is a particle that lifts to a fermion in the Drinfeld center. 
%This is the case for $\halfesix$.

In what follows, we will first introduce the fusion theory of $\halfesix$ 
and its properties that are pertinent to the rest of the section.
We will then compute the half braid for the emergent fermion, and condense it in the same way we did in the previous examples.
Following this, we will compute the idempotents in the condensed theory, 
as well as the modular $S$ and $T$ matrices. 



%%%%%%%%%%%%%%%%%%%%%%%%
\subsection{Fusion theory of $\halfesix$}
\label{halfesixFusionTheory}
%%%%%%%%%%%%%%%%%%%%%%%%

The $E_6$ fusion category is the fusion category whose principle graph is given by the $\text{E}_6$ Dynkin diagram, 
shown to the left in Figure \ref{EsixDynkin}.
The $E_6$ fusion category has two sub-categories: one subcategory has the fusion rules of the Ising theory, 
while the other is known as $\halfesix$ \cite{Hong2008} and has more complicated fusion rules. 
\begin{figure}
\begin{align}
\vcenter{
\xymatrix @!0 @M=4mm @R=28mm @C=40mm{
%&\EsixDynkin  \ar[rr]<-3ex>^-{ \text{   condense $y$  }} &   &\EsixCondensePsi&  \\
&\EsixDynkin  &   &\EsixCondensePsi&  \\
%&\HalfEsixDynkin  \ar[rr]<-1.5ex>^-{ \text{   condense $y$  }} &   &\HalfEsixDynkinCondensed&  \\
&\HalfEsixDynkin &   &\HalfEsixDynkinCondensed&  \\
 	}} \nonumber
\end{align}
\caption{
On the upper left we have the $E_6$ Dynkin diagram. 
The $E_6$ fusion theory has two closed fusion subcategories whose simple objects are $\left \{ \mathds{1},\; \sigma,\; y \right\}$ and $\left \{ \mathds{1}, x,y \right \}$. 
The first satisfies the Ising fusion rules while the second satisfies those of $\halfesix$ given in (\ref{halfEsixFusionRules}).
The figure on the upper right denotes the principal graph of the theory after condensing $y$ (after first lifting $y$ to the Drinfeld center).
On the bottom left we have drawn the $\halfesix$ principal graph.
%In this section we describe how to quotient by $y$ in the sub-category $\halfesix$.
%$y$ is fermionic in the center of $\halfesix$, so this is tantamount to performing fermion condensation.
The figure on the bottom right is the principal graph of $\halfesix/y$ studied in this section.
The fermionic quotient of the $\halfesix$ fusion subcategory reduces the particle content to $\{\unit,\rho\}$.  
%\kw{I think we should also include the principal graphs of half-e6 and half-e6/psi.}
%\dave{The more pictures the better. 
%I'm uncertain of what they look like though.
%Does the $\halfesix$ Dynkin diagram have a double line for the loop in the middle?
%What about $\halfesix/\psi$? }
%\kw{Let's make the non-directed double edges non-parallel (as in fig 6.1.1), so that they are
%not confused with directed double edges.}
}
\label{EsixDynkin}
\end{figure}


The fusion category $\halfesix$ has three particles, $\mathds{1}$, $x$, and $y$. 
The non-trivial fusion rules are
\begin{align}
y \tp y \cong \mathds{1} \quad \quad y \tp x \cong x \tp y \cong x \quad \quad x\tp x \cong \mathds{1}\oplus 2x \oplus y,
\label{halfEsixFusionRules}
\end{align}
and the quantum dimensions are given by
\begin{align}
d_{\mathds{1}} = 1 \quad \quad d_x = 1 + \sqrt{3} \quad \quad d_y = 1.
\end{align}
Note that one of the fusion spaces ($V_x^{xx}$) has dimension greater than 1.
Note also that $x$ 
is invariant under fusion with $y$, and that $y$ has quantum dimension $1$. 
If the theory were braided, and $y$ were fermionic, then condensing $y$ would lead to a super pivotal 
fusion theory with only two objects, $\mathds{1}$ and $\rho$, the image of $x$ under condensation of $y$. 
This theory however is not braided, and so we will have to do more work to condense $y$, 
as discussed in the next subsection 
%\ref{condensey} 
(see also the discussion in Section \ref{lift_and_condense}).

\medskip

We now to lay out some of the basic data of $\halfesix$ which will be useful to us in the following sections. 
Specifically we will give all information required to manipulate the $y$ line, 
which will be useful knowledge to have on hand when we condense $y$. 

From looking at (\ref{halfEsixFusionRules}) we notice that all particles are self dual, 
and therefore we must specify their Frobenius-Schur indicators. 
In this case both Frobenius-Schur indicators are equal to 1.
%, meaning that 
%\begin{align}
%\Kappay= \Idy
%\qquad \qquad \qquad
%\Kappax =\Idx
%\end{align}
These can be found from the associators $\kappa_x = d_x \left[ F^{xxx}_x \right]_{\mathds{1}\mathds{1}}$, and similarly for $y$. 
We list all the $F$-symbols (as found in \cite{okazaki2013,Wakui2002}) in App.~\ref{E6Fsymbols}.
Using the $F$-symbols in App.~\ref{E6Fsymbols} we can check that, in this gauge, $y$ has nice pivotal properties:
\begin{align}
\begin{matrix}
&{\xxypivotadual  = \xxypivotbdual  =  \Vxxydual} \\
&\\
&{\xxypivota =  \xxypivotb = \Vxxy } \\
\end{matrix}
&\;\;
\begin{matrix}
&{\yxxpivotadual = \yxxpivotbdual = \Vyxxdual }\\
&\\
&{\yxxpivota = \yxxpivotb=  \Vyxx }\\
\end{matrix}
&\;\;
\begin{matrix}
&{\xyxpivotadual = \xyxpivotbdual = \Vxyxdual } \\
&\\
&{\xyxpivota =  \xyxpivotb =  \Vxyx}\\
\end{matrix}
\end{align}
The fact that these diagrams are trivially pivotal is a reflection of the gauge choice used for the splitting spaces.

Next, we look at what happens when we slide a $y$ line past a $V^{xx}_x$ fusion space.
Since $\text{dim} V^{xx}_x = 2$ the fusion space requires a multiplicity index labeling the independent vectors spanning this vector space. 
We denote them $v_1$ and $v_2$, and diagrammatically label them with an index at the fusion vertex:
\begin{align}
V^{xx}_x \cong \mathbb{C} \left[\;\; \Vxxxa \;\;  \right  ] \qquad \qquad a = 1, 2.
\end{align}
The next three relations show what happens when $y$ shifts past the fusion space $V^{xx}_x$:
\begin{align}
\Vxyxxa = \sigma^x_{ab} \; \Vxyxxb
\quad \quad 
\Vyxxxa = \sigma^z_{ab} \;\; \Vyxxxb
\quad \quad 
\Vxxyxa =  \sigma^y_{ab} \;\Vxxyxb,
\label{yslide}
\end{align}
where the $\sigma^w$, $w = x,y,z$ are the standard Pauli matrices.\footnote{
Explicitly $\sigma^x = \left( \begin{matrix} 0 &1\\ 1&0 \end{matrix} \right) \quad  \sigma^y = \left( \begin{matrix} 0 &-i\\ i&0 \end{matrix} \right)  \quad \sigma^z = \left( \begin{matrix} 1 &0\\ 0&-1 \end{matrix} \right)$}
%When we condense $y$, these sliding moves will correspond to even endomorphisms of the fusion space in the condensed theory.
When we condense $y$, these sliding moves will determine the action of $\End(\rho) \tp \End(\rho)$ on $V^{\rho \rho}_\rho$.
\kw{why not $\End(\rho) \tp \End(\rho) \tp \End(\rho)$?}
%\kw{I'm confused by this.  Endomorphisms of what object?}
%\dave{Endomorphisms of $V^{\rho \rho}_\rho$.}
%\kw{But $V^{\rho \rho}_\rho$ is a morphism space, not an object, so ``endomorphism" is not the right word.
%Maybe ``These sliding moves will determine the action of $\End(\rho)$ on $V^{\rho \rho}_\rho$"?
%\dave{Sounds good. I changed to ``action of $\End(\rho) \tp \End(\rho)$ on $V^{\rho \rho}_\rho$.}
%\kw{I would have thought that the sliding moves have something to do with relations between $\Gamma$ matrices (which are odd operators).}
%\dave{They are also a product of two $\Gamma$ matrices.}

Lastly, we have
\begin{align}
&\Vxxxva \; = \; \left( W_{\mathds{1}} \right)_{ab} \Vxxxvb \quad \quad 
W_{\mathds{1}} = \frac{e^{-7 i \pi /12}}{\sqrt{2}}\left( \begin{matrix} 1 &-i \\ 1 & i \end{matrix} \right) \\
\nonumber \\
&  \Vxxxya \; = \;  \left( W_{y} \right )_{ab} \; \Vxxxyb  \quad \quad   W_y = \frac{e^{-7 i \pi /12}}{\sqrt{2}}\left( \begin{matrix} 1 &-i \\ -1 & -i \end{matrix} \right)
\end{align}
These will be of use to us when we specify the pivotal properties of $\halfesix$ 
after condensing $y$; see Section \ref{ESixPivotal} for more details.
The only data left to specify is the associators for the $V^{xxx}_x$ fusion space, 
which we list in Appendix~\ref{E6Fsymbols}.



%%%%%%%%%%%%%%%%%%%%%%%%%%%%
\subsection{Fermion condensation in $\halfesix$}
\label{condensey}
%%%%%%%%%%%%%%%%%%%%%%%%%%%%

%\dave{Some references that may be fitting: \cite{majid1991}, and apparently \cite{joyal1991}. Also is this called a relative center? Notes on Modular Categories by Muger, or chapter 6 of \cite{heunen2013}.}
In this subsection we will describe the procedure for condensing the $y$ particle in $\halfesix$. 
As mentioned earlier, $\halfesix$ is not braided, and so when we say condense $y$, 
we actually mean that we lift $y$ to the Drinfeld center (where it is an emergent fermion), and condense the lift of $y$. 
Since the center of $\halfesix$ has been computed in several places \cite{Hong2008,Izumi2001, kawahigashi2001} we will not provide all details. 

The lift of $y$ to the Drinfeld center can be found by solving \eqref{halfbraid_resolution} 
subject to the constraints \eqref{identityBraid} and \eqref{junctionBraid}.
Using the fusion theory of $\halfesix$ defined above in Section~\ref{halfesixFusionTheory} one readily finds the unique solution,
\begin{align}
\halfbraid{\ebox}{y}{\unit +x+y}{} = \halfbraid{\eone}{y}{\mathds{1}}{} - \; i   \halfbraid{\ex}{y}{x}{x} -   \halfbraid{\ey}{y}{y}{\mathds{1}}.
\label{yhalfbraid}
\end{align}
The negative sign on the last term makes the statistics and twist
of (the lift of) $y$ fermionic. 

We are now in a position to condense $y$.
Since $x$ is invariant under fusion with $y$ after condensation it becomes a q-type simple object, which we will denote by $\rho$:
\begin{align}
x \xrightarrow{\;\; \text{condense $y$\;\;}} \rho \quad \quad \quad \text{End}(\rho) \cong \mathbb{C} \ell_1
\end{align}
$\rho$ is the only non-trivial simple object in the condensed theory. 
Furthermore, since $x$ has fusion multiplicity in the parent theory, $\rho$ has fusion multiplicity in the condensed theory.
This is captured by the fusion space
\begin{align}
V^{\rho \rho}_\rho \cong \mathbb{C}^{2|2}.
\end{align}
The nontrivial fusion rule of the condensed theory is
\begin{align}
\rho \tp \rho = \mathbb{C}^{1|1}\cdot \mathds{1} \oplus \mathbb{C}^{1|1} \cdot \rho.
\end{align}
The fusion rule coefficients $\Delta^{\rho\rho}_\rho = \Delta^{\rho\rho}_\unit \cong \cc^{1|1}$ 
appearing in the above formula are determined by the relation $V^{ab}_c \cong \Delta^{ab}_c \tp \End(c)$ and the 
knowledge of the fusion spaces $V^{\rho\rho}_\rho \cong \cc^{2|2}, V^{\rho \unit}_\rho \cong \cc^{1|1}$.\footnote{We could have 
also taken $\Delta^{\rho \rho}_\rho \cong \cc^{2|0}$ or $ \cc^{0|2}$
\dave{Reminder to revisit this in case we find more constraints that rule these out.}
\dave{Still unsure what to say about this.}
\kw{If we decide that $\cc^{2|0}$ is just as valid as $\cc^{1|1}$, 
then we should reconsider the coloring of the edges on the principal graph fig.}
\dave{Maybe it gives a good reason on why to select $\cc^{1|1}$ over $\cc^{2|0}$ or $\cc^{0|2}$.
Haven't had time to play play with it yet though.}
}.

%\dave{I think we should consider dropping this heuristic.}
%\ethan{Agreed.}
%Heuristically, one can think of the $\rho$ lines as carrying a p-wave superconductor.
%In this scenario the junction of three of these has four different ways to fuse together, two even and two odd. 
%However, unlike a p-wave superconductor the junction does not leave behind a vortex. 





%%%%%%%%%%%%%%%%%%%%
\subsubsection{Pivotal structure}
\label{ESixPivotal}
%%%%%%%%%%%%%%%%%%%%

Since $\text{End}(\rho) \cong \cl_1$, $\End(\rho)$ possesses an odd endomorphism which we will denote as $f$. 
We will denote the even basis vectors of $V^{\rho \rho}_\rho$ as $v_1$ and $v_2$, so that the odd basis 
vectors are $f {v_1}$ and $f {v_{2}}$, where the $f v_i$ are obtained by acting with $f$ on the bottom leg of the fusion space.
Diagrammatically we can denote this vector space by
\begin{align}
V^{\rho \rho }_\rho \; \cong \; \mathbb{C} \left[ \VSeven , \quad \VSa \right] \quad \quad \quad a = 1, 2,
\end{align}
where $f$ is represented graphically by the blue dot. 
We can then use our knowledge of local relations in the parent $\halfesix$ theory and the lift of $y$ to 
derive the following relations in the condensed theory:
\begin{align} 
\VSa = \sigma^z_{ab} \VSb \qquad \VSa = \sigma^y_{ab} \VSc \qquad \VSd = -i \sigma^x_{ab} \VSc,
\label{esixdotslide}
\end{align}
where the $\sigma^{w}$ are the standard Pauli matrices (compare \eqref{yslide}). 
We can also obtain the following pivoting moves:
 \begin{align}
 \PivotEsixEven  = 
X_{ab}\Vrhorhorho \quad \quad X = \frac{e^{7 i \pi/12}}{\sqrt{2}}\left( \begin{matrix}
 1& 1\\ 
 i &- i
 \end{matrix} \right)  \\
 \PivotEsixOdd =
Y_{ab} \Vrhorhorhoodd \quad \quad Y =  \frac{e^{7 i \pi/12}}{\sqrt{2}}\left( \begin{matrix} 
 i& -i\\ 
 -1 & -1
 \end{matrix} \right) 
 \end{align}
 Note that we have $X^3=\unit,Y^3=-\unit$, which is consistent with the fact that $X$ acts on even vectors 
 while $Y$ acts on odd ones (so that $Y^3$ rotates an odd vector by $2\pi$, which produces a minus sign). 
 

%%%%%%%%%%%%%%%%%%%%%%%%%%%%%%% 
\subsection{The tube category and the torus}
%%%%%%%%%%%%%%%%%%%%%%%%%%%%%%%


%%%%%%%%%%%%%%%%%%%%%%%%%%%%%%
\subsubsection{Tube category morphism spaces}
%\subsubsection{Relations on the annulus}
%\subsubsection{Local relations on the annulus}
%%KW not really local if it's on the annulus;  local = in a disk
%%%%%%%%%%%%%%%%%%%%%%%%%%%%%%

%\kw{I think we should give a basis (even and odd) of the morphism spaces of the tube cat; (B.N) x (empty-empty, empty-rho, rho-rho) = 6 separate morphism spaces.}
%\dave{Sounds good.
%Is a table like the following three what you have in mind?
%Once we've added all the data we need, 
%I'll format it so it fits in the page better (nothing fanct, just a resizebox or something).}
%\dave{referenced each table throughout this subsection.}

\begin{table}
\scalebox{0.86}{
\begin{tabu}{ r  c c c }
&$e$&$h$&$\stackrel{\bullet}{h}$\\
\cline{2-4} 
&      &     &  \\[-2ex]
$\mor^B(e \ra e) $ 
&$\AddDat{\eTube}{B}{} $
&$\AddDat{\hTube}{B}{} $&\\
\\
$\mor^N(e \ra e) $ 
&$\AddDat{\eTube}{N}{} $
&&$\AddDat{\hTube}{N}{\dotb} $\\
\end{tabu}
}
\caption{A complete basis of morphisms for $\text{mor}^B(e \ra e)$.
The labels above each tube are short-hand for that tube: $e$ -- empty tube; $h$ -- tube with horizontal $\rho$ line. 
The dot above $h$ denotes that the morphism is an odd morphism.}
\label{etoe}

\bigskip
\bigskip


%\begin{tabu}{ r  c c c c }
%&$k_1$&$k_2$&$\stackrel{\bullet}{k}_1$& $\stackrel{\bullet}{k}_2$\\
%\cline{2-5} 
%&      &     & &  \\[-2ex]
%$\mor^B(e \ra \rho) $ 
%&$\AddDat{\AddDat{\TadupTube{1}}{B}{}}{}{} $
%&$\AddDat{\AddDat{\TadupTube{2}}{B}{}}{}{} $
%&$\AddDat{\AddDat{\TadupTube{1}}{B}{\dotc}}{}{} $
%&$\AddDat{\AddDat{\TadupTube{2}}{B}{\dotc}}{}{} $\\
%\\
%$\mor^N(e \ra \rho) $ 
%&$\AddDat{\AddDat{\TadupTube{1}}{N}{}}{}{} $
%&$\AddDat{\AddDat{\TadupTube{2}}{N}{}}{}{} $
%&$\AddDat{\AddDat{\TadupTube{1}}{N}{\dotc}}{}{} $
%&$\AddDat{\AddDat{\TadupTube{2}}{N}{\dotc}}{}{} $
%\end{tabu}
%\caption{$\mor(e \ra \rho) \in  \tube(\halfesix) $}


\scalebox{0.86}{
\begin{tabu}{ r  c c c }
&$k_1$&$\stackrel{\bullet}{k}_1$\\
\cline{2-3} 
&      &     \\[-2ex]
$\mor^B(\rho \ra e) $ 
&$\AddDat{\AddDat{\TaddownTubeprime{1}}{B}{}}{}{} $
&$\AddDat{\AddDat{\TaddownTubeprime{1}}{B}{\dota}}{}{} $\\
\\
$\mor^N(\rho \ra e) $ 
&$\AddDat{\AddDat{\TaddownTubeprime{1}}{N}{}}{}{} $
&$\AddDat{\AddDat{\TaddownTubeprime{1}}{N}{\dota}}{}{} $
\end{tabu}
}
\caption{A complete basis of morphisms $\mor(\rho \ra e) \in  \tube(\halfesix) $.
We have denoted these by $k_1$ and $\stackrel{\bullet}{k}_1$.
%	\kw{shouldn't there be only half this number, because of the relations?  $1|1$ instead of $2|2$}
%	\dave{Fixed.}
}
\label{rhotoe}


\bigskip
\bigskip

\scalebox{0.86}{
\begin{tabu}{ r  c c c c c c c c }
&$v$&$t$&$X_{11}$&$X_{12}$&$\stackrel{\bullet}{v}$&$\stackrel{\bullet}{t}$&$\stackrel{\bullet}{X}_{11}$&$\stackrel{\bullet}{X}_{12}$ \\
\cline{2-9} 
&      &      &      &     &      &     &    &\\[-2ex]
$\mor^B(\rho \ra \rho) $ 
&$\AddDat{\vTube}{B}{} $
&$\AddDat{\tTube}{B}{} $
&$ \AddDat{\XTube{1}{1}}{B}{} $
&$ \AddDat{\XTube{1}{2}}{B}{}  $
&$\AddDat{\vTube}{B}{\dotb} $
&$\AddDat{\tTube}{B}{\dotc} $
&$ \AddDat{\XTube{1}{1}}{B}{\dotc} $
&$ \AddDat{\XTube{1}{2}}{B}{\dotc}$\\
\\
$\mor^N(\rho \ra \rho) $ 
&$\AddDat{\vTube}{N}{} $
&$\AddDat{\tTube}{N}{} $
&$ \AddDat{\XTube{1}{1}}{N}{} $
&$ \AddDat{\XTube{1}{2}}{N}{}  $
&$\AddDat{\vTube}{N}{\dotb} $
&$\AddDat{\tTube}{N}{\dotc} $
&$ \AddDat{\XTube{1}{1}}{N}{\dotc} $
&$ \AddDat{\XTube{1}{2}}{N}{\dotc}$\\
\end{tabu}
}
\caption{A complete basis of morphisms for $\mor(\rho \ra \rho) \in  \tube(\halfesix)$.
The labels above each tube are shorthand for that tube:
$v$ -- tube with vertical $\rho$ strand; 
$t$ -- tube with $\rho$ strand wrapping both cycles;
$X$ -- tube with all labels given by $\rho$. 
As before, a dot denotes an odd vector.}
\label{rhotorho}
\end{table}

\kw{Should we center the morphism basis tables?}

In this subsection we compute bases for tube category morphism spaces. 
We will make use of the notation $s(X)$ defined in \eqref{sX_defn}.
%\dave{Maybe change to $s(X) = \pm1$ as we did in other sections.}
%\ethan{agreed, changed}
%\ethan{edited this because $|X|$ suggests a relation to fermion parity which isn't there and since we used $s(X)$ at one point earlier in the paper. Can do a find+replace to revert if you don't dig it}
%\dave{Works for me.}
Using the relations
\begin{align}
& \AddDat{\hTube}{X}{} \; =  s(X)  \AddDat{\hTube}{X}{} \quad \quad \quad \AddDat{\hTube}{X}{\dotb} \; =  -s(X)  \AddDat{\hTube}{X}{\dotb} 
\label{hlocalrelation}
\end{align}
we see that
a complete basis of morphisms from trivial (empty) boundary to the empty boundary is listed in Table \ref{etoe}. 
(The relations are found by taking two fermions out of the vacuum and sliding them around the annulus.)

Similarly, we have
\begin{align}
\AddDat{\TaddownTube{\mu }}{X}{}\;= s(X) \sigma^{x}_{\mu \nu} \AddDat{\TaddownTube{\nu}}{X}{} \quad \quad \quad
\AddDat{\TadupTube{\mu}}{X}{}\;= s(X) \sigma^y_{\mu \nu} \AddDat{\TadupTube{\nu}}{X}{}.
\end{align}
Hence some of these states are linearly dependent, and so a complete basis of morphisms is found if we fix the fusion space, 
and also consider the action of the odd endomorphism. 
The basis we choose is listed in Table \ref{rhotoe}.

\begin{comment}
\begin{table}
\begin{tabu}{ r  c c c c c c }
&$e$&$h$&$v$&$t$&$X_{11}$&$X_{12}$ \\
\cline{2-7} 
&      &      &      &     &              &\\[-2ex]
$\text{Bounding}$ &$\AddDat{\eTube}{B}{} $&$\AddDat{\hTube}{B}{} $&$ \AddDat{\vTube}{B}{} $&$ \AddDat{\tTube}{B}{}  $&$ \AddDat{\XTube{1}{1}}{B}{} $&$ \AddDat{\XTube{1}{2}}{B}{}$\\
\\
$\text{Non-bounding}$ &$\AddDat{\eTube}{N}{}  $& &$ \AddDat{\vTube}{N}{}$&$ \AddDat{\tTube}{N}{} $& $\AddDat{\XTube{1}{1}}{N}{}$ &$ \AddDat{\XTube{1}{2}}{N}{} $ \\
\end{tabu}
\caption{\label{AnnulusBasis} The table above is the basis of tubes we use on the annulus for a 
bounding spin structure ($B$) and non-bounding spin structure ($N$). 
%We also list which tubes are oddly isomorphic by labeling an arrow between them with $\Pi$, and which tubes have odd endomorphisms by labeling them with $\Pi_e$. 
The labels above each tube are short-hand for that tube, $e$ -- empty, $h$ -- horizontal, 
$v$ -- vertical, $t$ -- twist, $X$ -- crossed.
The tubes $X_{11}$ and $X_{12}$ are oddly isomorphic.
\kw{What does it mean for vectors to be isomorphic?  This makes no sense to me.}
\kw{Why are there no odd tubes in the figure?}
}
\dave{I was probably only listing tubes appearing in the idempotents. 
I'm making the new tables now.}
\end{table}
\end{comment} 


\begin{comment}
\begin{figure}
\begin{align}
\xymatrix @!0 @M=1mm @C=20mm{
&e&h&v&t&X_{11}&X_{12} \\
&&&&&& \\
&&&&&\;  \ar@{<->} @/^1pc/[r]^{\Pi} &\; \\
\text{Bounding}\quad &\AddDat{\eTube}{B}{} &\AddDat{\hTube}{B}{} & \AddDat{\vTube}{B}{}\ar @`{p+(-15,20),p+(15,20)}^{\Pi_e} & \AddDat{\tTube}{B}{}  \ar @`{p+(-15,20),p+(15,20)}^{\Pi} & \AddDat{\XTube{1}{1}}{B}{} & \AddDat{\XTube{1}{2}}{B}{}\\
\\
\text{Non-bounding}\quad &\AddDat{\eTube}{N}{}  \ar @`{p+(-15,-20),p+(15,-20)}_{\Pi_e } & & \AddDat{\vTube}{N}{} \ar @`{p+(-15,-20),p+(15,-20)}_{\Pi_e } & \AddDat{\tTube}{N}{} \ar @`{p+(-15,-20),p+(15,-20)}_{\Pi_e } & \AddDat{\XTube{1}{1}}{N}{} & \AddDat{\XTube{1}{2}}{N}{}\\
&&&&&\;  \ar@{<->} @/_1pc/[r]_{\Pi} &\; \\
}
\nonumber
\end{align}
\caption{The table above table is the basis of tubes we use on the annulus for a bounding spin structure ($B$) and non-bounding spin structure ($N$). 
We also list which tubes are oddly isomorphic by labeling an arrow between them with $\Pi$, and which tubes have odd endomorphisms by labeling them with $\Pi_e$. 
The labels above each tube are short-hand for that tube, $e$-- empty, $h$-- horizontal, $v$-- vertical, $t$-- twist, $X$-- crossed.
\dave{I think we should remove the $\Pi$'s and just comment in the caption that $X_{11} \cong X_{12}$ oddly.}}
\end{figure}
\end{comment}

The morphisms of $\text{mor}(\rho \ra \rho)$ satisfy
\begin{align}
& \AddDat{\XTube{\mu}{\nu}}{X}{} \;\;  =s(X) (\sigma^x \tp \sigma^y )_{\mu \nu; \kappa \tau}\AddDat{\XTube{\kappa}{\tau}}{X}{} 
\label{XlocalRelationAnnulus}
\end{align}
and also
%the odd isomorphism
\begin{align}
\AddDat{\XTube{\mu}{\nu}}{X}{\dotc} \;= (\sigma^y \tp \sigma^z)_{\mu \nu; \kappa \tau} \; \AddDat{\XTube{\kappa}{\tau}}{X}{\dota}
\end{align}
We can use these results to obtain a basis for $\text{mor}(\rho \ra \rho)$ which is given in Table \ref{rhotorho}.
% for the vector space of string-nets on the annulus, which is given in Figure \ref{AnnulusBasis}.



%%%%%%%%%%%%%%%%%%%%%%%%%%%%%%%%%%%%%%%%
\subsubsection{Relations on the torus} \label{TorusLocalRelations}
%%%%%%%%%%%%%%%%%%%%%%%%%%%%%%%%%%%%%%%%%

In this subsection we compute bases for the Hilbert spaces of spin Tori.
We could of course do this using knowledge of tube category idempotents and \ref{torus_basis_theorem}, but
it's an instructive exercise to also compute bases using more elementary means.
In addition, when computing $S$ and $T$ matrices, it will be useful to have a topologically simple basis at our disposal.

\medskip

On the torus there are four spin structures, and extra local relations coming from the additional cycle. 
%On the torus we have four spin structures, and extra local relations coming from the additional cycle. 
%We label the spin structures by with either a $B$ for bounding, and $N$ for non-bounding. 
%The notation we use to denote the spin structure is \begin{align}
%\SmallTorus{\AnnulusBare}{X}{Y} \quad \quad \quad X,Y = N, B, 
%\end{align}
%where the inner label $X$ denotes the spin structure around the longitudinal cycle (the one around the circular boundaries of the annulus in the plane of the page), 
%and where the outer label $Y$ denotes the spin structure around the meridional cycle (the one that wraps out of the plane of the page). 
%Usually we will have pictures on the torus, and will move the label for the spin structure on the meridional cycle outside of the torus:
%\begin{align}
%\SmallTorus{\AnnulusBare}{X}{Y} \; \longrightarrow \; \AddDatTorus{\eTube}{X}{Y}{}.
%\end{align}
We will first investigate the local relations on the three bounding tori which have spin structure $(X,Y) = (B,B),(N,B),(B,N)$.
Then we will consider the $(N,N)$ torus separately.
We use the same notation as in Section~\ref{modulartforms}. 

Depending on the spin structure, some of the annular tubes become zero after identifying the boundaries to form tori.
Since there is always one cycle with bounding spin structure, an odd tube is 
identified with zero for the same reason as discussed in \ref{c2_stringnets_torus}.
For another example, due to (\ref{hlocalrelation}) we have
\begin{align}
 0 = \;\AddDatTorus{\hTube}{N}{B} \;=\; \AddDatTorus{\vTube}{B}{N}\; = \; \AddDatTorus{\tTube}{B}{B}
\end{align}

We also get another local relation from \eqref{XlocalRelationAnnulus} after closing up the annulus to a torus, namely
\begin{align}
\AddDatTorus{\XTube{a}{b}}{X}{Y}{} \; = M_{ab;\alpha \beta} \AddDatTorus{\XTube{\alpha}{\beta}}{X}{Y}{} \qquad M = s(X) \sigma^x \tp \sigma^y, \; \; s(Y) \sigma^y \tp \sigma^z 
\end{align}
The two relations above can be multiplied to find a third:
\begin{align}
\AddDatTorus{\XTube{1}{1}}{X}{Y}{} \; = -s(X)s(Y) \AddDatTorus{\XTube{1}{2}}{X}{Y}{}  \; = -is(Y) \AddDatTorus{\XTube{2}{1}}{X}{Y}{} \; = -is(X) \AddDatTorus{\XTube{2}{2}}{X}{Y}{} 
\label{XLocalRelation}
\end{align}
We take the state with $(a,b) = (1,1)$ as the representative of this set of linearly dependent vectors. 

There is one additional useful linear relation to be found. 
This relation 
comes form nucleating a $\rho$ loop and extending it around the torus before fusing it back into the canonical basis:
\begin{align}
d_\rho \AddDatTorus{\eTube}{X}{Y} \; = \;
\AddDatTorus{\TorusLocalRelationa}{X}{Y} \;=\;
\AddDatTorus{\TorusLocalRelationb}{X}{Y} \;=\; 
\AddDatTorus{\TorusLocalRelationc}{X}{Y} \;=\; 
\sum_\lambda \; c_\lambda \AddDatTorus{\TorusNoLabelsx{\lambda}}{X}{Y},
\label{rhoLoopRelation}
\end{align}
where $c_\lambda$ are coefficients that depend on the $F$-symbols.
The string of equalities gives an additional local relation, in particular, it allows us 
solve for the tubes in (\ref{XLocalRelation}) above in terms of the other non-zero tubes.
Explicitly in the three sectors $(B,B), (B,N)$, and $(N,B)$ we have,
\begin{align}
& \AddDatTorus{\XTube{1}{1}}{B}{B} \;= e^{-i \pi /12} \frac{d+1}{\sqrt{d}} \left( \;\AddDatTorus{\eTube}{B}{B}\; -  \frac{1}{d} \left(\AddDatTorus{\hTube}{B}{B}   +  \AddDatTorus{\vTube}{B}{B}  \right) \right)\\
&\\
 & \AddDatTorus{\XTube{1}{1}}{B}{N} \; = e^{-5i \pi /12} \frac{d+1}{\sqrt{d}} \left( \AddDatTorus{\eTube}{B}{N} -  \frac{1}{d} \left( \AddDatTorus{\hTube}{B}{N}   + \AddDatTorus{\tTube}{B}{N} \right) \right)\\
 &\\
& \AddDatTorus{\XTube{1}{1}}{N}{B}  \;= e^{3 i \pi /4} \frac{d+1}{\sqrt{d}} \left( \AddDatTorus{\eTube}{N}{B}-  \frac{1}{d} \left( \AddDatTorus{\vTube}{N}{B} + \AddDatTorus{\tTube}{N}{B} \right) \right)
\end{align}

We now move onto the torus where the spin structure is non-bounding along both cycles.
We first notice that
\begin{align}
 0 = \;\AddDatTorus{\hTube}{N}{N} \;=\; \AddDatTorus{\vTube}{N}{N}\; = \; \AddDatTorus{\tTube}{N}{N}\;,
\end{align}
which can be seen by nucleating two fermions out of the vacuum along each $\rho$ line and 
dragging one of them along the entire $\rho$ line before fusing them back to the vacuum.
Furthermore, the same calculation as \eqref{rhoLoopRelation} implies that the the empty tube is identified with zero:
\begin{align}
\AddDatTorus{\eTube}{N}{N} \; = 0
\end{align}
The only non-zero tube with even parity is given by
\begin{align}
\AddDatTorus{\XTube{1}{1}}{N}{N}
\end{align}
and those which are proportional to it are given by \eqref{XLocalRelation}. 
%\dave{Do we have a nice way to see that this tube is non-zero other than there is no local relation that makes it proportional to the other even tubes?}
%\kw{I always find proving things are non-zero a pain in the ass.
%Maybe I'll ask Scott if he knows of better ways.
%One general prescription (which is frequently easy/practical) is to use the inner product
%coming from the TQFT.
%(That's what I did in the MTC/psi notes.)}
%\dave{Ah yes. That's pretty easy to do in this case. 
%(found the even trace to be $d_\rho$ and the odd trace to be $0$)}

As one may expect from the Ising example, there are odd tubes which are non-zero.
Indeed we can find four of them,
\begin{align}
\AddDatTorusDot{\hTube}{N}{N}{\dotb}\quad \AddDatTorusDot{\vTube}{N}{N}{\dotb} \quad \AddDatTorusDot{\tTube}{N}{N}{\dotc} \quad \AddDatTorusDot{\XTube{1}{1}}{N}{N}{\dotc}.
\end{align}
%As before one can find linear relations between the fourth tube above and the ones with different multiplicity indices.
However, these four tubes are not linearly independent. 
There are two independent linear relations that can be found between them by multiplying the tadpole-like diagrams in two different ways,
\begin{align}
\AddDatTorusDot{\TaddownTube{\mu}}{N}{N}{} \cdot \AddDatTorusDot{\TadupTube{\nu}}{N}{N}{\dotc} \;=\;
\AddDatTorusDot{\TadupTube{\nu}}{N}{N}{\dotc} \cdot  \AddDatTorusDot{\TaddownTube{\mu}}{N}{N}{} .
\end{align}
This is an instance of the familiar relation $cl(a\cdot b) = cl(b\cdot a)$.
Despite the indices $\mu$ and $\nu$ varying over four distinct values, this yields only two linearly independent relations on the torus. 
They are given by
\begin{align}
\frac{de^{ -i \pi / 4 }}{\sqrt{2}}  \AddDatTorusDot{\hTube}{N}{N}{\dotb}\; =\;
 \AddDatTorusDot{\vTube}{N}{N}{\dotb}  \; + \; 
 \AddDatTorusDot{\tTube}{N}{N}{\dotc} \; - \frac{2 e^{i \pi /4}}{\sqrt{d}}  
 \AddDatTorusDot{\XTube{1}{1}}{N}{N}{\dotc}
\end{align}
and\footnote{One can check that the second relation follows from the first by performing an $S$ transformation}
\begin{align}
\frac{d e^{i \pi/4}}{\sqrt{2}}  \AddDatTorusDot{\vTube}{N}{N}{\dotb}\; = \;
\AddDatTorusDot{\hTube}{N}{N}{\dotb}\; + i 
 \AddDatTorusDot{\tTube}{N}{N}{\dotc}\; + \frac{2 e^{11i \pi /12}}{\sqrt{d}} 
 \AddDatTorusDot{\XTube{1}{1}}{N}{N}{\dotc}
\end{align}
We can now solve for any two of the above four states.
We choose
\begin{align}
 \AddDatTorusDot{\tTube}{N}{N}{\dotc}  \;&=\; \AddDatTorusDot{\hTube}{N}{N}{\dotb} \; -i\;  \AddDatTorusDot{\vTube}{N}{N}{\dotb}  \\
 &\\
 \AddDatTorusDot{\XTube{1}{1}}{N}{N}{\dotc} \;&=\;\sqrt{\frac{d}{2}}\left( e^{i \pi /3}\;  \AddDatTorusDot{\hTube}{N}{N}{\dotb} \; -i \; \AddDatTorusDot{\vTube}{N}{N}{\dotb} \;\right) 
\end{align}

In summary, the Hilbert spaces on each of the different spin tori are 
\begin{align}
\begin{split}
\mch^{T^2}_{(B,B)}  &\cong \mathbb{C}^{3|0} = \mathbb{C}\left[ \AddDatTorus{\eTube}{B}{B}{},\AddDatTorus{\hTube}{B}{B}{},\AddDatTorus{\vTube}{B}{B}{}\right]\\
&\\
\mch^{T^2}_{(B,N)}  &\cong \mathbb{C}^{3|0}  = \mathbb{C}\left[\AddDatTorus{\eTube}{B}{N}{}, \AddDatTorus{\hTube}{B}{N}{}, \AddDatTorus{\tTube}{B}{N}{} \right]\\
&\\
\mch^{T^2}_{(N,B)}  &\cong \mathbb{C}^{3|0}=   \mathbb{C}\left[\AddDatTorus{\eTube}{N}{B}{}, \AddDatTorus{\vTube}{N}{B}{}, \AddDatTorus{\tTube}{N}{B}{} \right] \\
&\\
\mch^{T^2}_{(N,N)}  &\cong \mathbb{C}^{1|2} = \mathbb{C} \left[ \AddDatTorusDot{\hTube}{N}{N}{\dotb}\;,\AddDatTorusDot{\tTube}{N}{N}{\dotc}\;,  \AddDatTorusDot{\XTube{1}{1}}{N}{N}{} \right]
\end{split}
\label{TorusBasis}
\end{align}





%%%%%%%%%%%%%%%%%%%%%%%%%%%%%
 \subsection{The tube category of $\halfesix / y$}
 %%%%%%%%%%%%%%%%%%%%%%%%%%%%%

\kw{I have not finished reviewing this subsection.  I'll discuss with Dave before proceeding.}

 In this subsection we compute the minimal idempotents of the tube category of $\halfesix / y$. 
The tube category is somewhat exotic, and highlights many of the non-trivial features which arise when studying fermionic theories. 

\dave{Should we remove the $\halfesix$ idempotents? 
I think they are correct, but I wasn't as careful when computing them.}
In Appendix \ref{IdempotentsHalfESix}, we compute the tube category of the parent (un-condensed) $\halfesix$ theory. 
Our plan of attack in this section will be to first find the tube category of $\halfesix$ and then perform 
fermion condensation, rather than first performing fermion condensation and then constructing the tube category. 
 

In order for this strategy to work, we need to construct a map from idempotents of the parent theory to idempotents of the condensed theory.
This mapping is performed by simply condensing fermions off the tubes in $\tube(\halfesix)$ to get tubes in $\tube(\halfesix / y)$. 
Special care must be taken with respect to spin structure issues, since removing $y$ lines may force a pair of fermions to traverse a cycle of the tube. 
We will see that there are six excitations in the condensed theory: three excitations 
with bounding spin structure, and three with non-bounding spin structure. 
We index them as $\mca = \{ m_1, m_2, m_3\}$ and $\mcv = \{ q_1, q_2, m_6 \}$, 
%\kw{I think we should switch to lower case notation for idempotents}
%\dave{sure.}
%\dave{done.}
where $\mca$ is the set of non-vortex anyons and $\mcv$ is the set of vortex defects. 
As the notation suggests, there are two q-type excitations and one m-type excitation 
(namely $m_6$) in $\mcv$, while all of the excitations in $\mca$ are (necessarily) m-type. 


\dave{Re-worded a bit. 
I'm not opposed to removing this bit. 
It may be useful to anyone who follows up on the work, 
but it is also prone to error.
For example in identifying with \cite{Hong2008} I couldn't figure out which particle corresponds to $X_2$ and which to $X_3$.
We could add a remark about that.
}

In the notation of \cite{Hong2008}, we find the following maps from idempotents in $\tube(\halfesix)$ to those in in $\tube(\halfesix / y)$:
\begin{align}
\xymatrix @!0 @M=1mm @C=10mm{
& &m_1 && m_2 && &&& m_3 &&  \\
 \\
 \quad&\mathds{1} \ar[ruu] \quad & Y \ar[uu] & W\ar[dd] & U\ar[uu] & V \ar[uul] & X_1 \ar[dd] & X_{2}\ar[ddr] &X_{3} \ar[dd] & X_4\ar[uu] & X_5\ar[uul] &  \\
\\
&&&q_1& &&q_2 &&m_6& &&  \\
},
\end{align}
where the center line lists the idempotents in $\tube(\halfesix)$ and the upper and lower objects are the objects in $\tube(\halfesix / y)$.
The identifications are made by taking a minimal idempotent in $\tube(\halfesix)$ and using the inclusion to $\tube(\halfesix/y)$, 
as discussed in Section \ref{double_fermionic_quotient}.

\begin{comment}
There are two nontrivial maps from tubes in $\tube(\halfesix)$ to those in $\tube(\halfesix / y)$ that we will make use of. 
One is the gluing of odd punctures, which we denote by $\Sigma$, while 
the other is the odd isomorphism which acts to pull a $y$ dot through a tube, which we denote by $\Pi$. 
In the notation of \cite{Hong2008}, we find the following maps from idempotents in $\tube(\halfesix)$ to those in in $\tube(\halfesix / y)$:
\dave{Do we want to keep this? 
When I wrote it this section was more at the notes stage, 
and I thought it would be helpful for something but never used it.}
\begin{align}
\xymatrix @!0 @M=1mm @C=10mm{
& &M_1 && M_2 && &&& M_3 &&  \\
 \\
 \quad&\mathds{1} \ar[ruu] \quad & Y \ar[uu]_{\Sigma} & W\ar[dd] \ar@/^1pc/[dd]^{\Sigma} & U\ar[uu] & V \ar[uul]_\Pi \ar@/_1.3pc/[uul]_\Sigma & X_1 \ar[dd] & X_{2,3}\ar[ddr] &X_{3,2} \ar[dd]^\Pi& X_4\ar[uu] & X_5\ar[uul]_\Pi &  \\
\\
&&&Q_1& &&Q_2 &&M_6& &&  \\
},
\end{align}
where the center line lists the idempotents in $\tube(\halfesix)$ and the upper and lower objects are the objects in $\tube(\halfesix / y)$.
\end{comment} 

\begin{table}
{{\tabulinesep=1.2mm
\begin{tabu}{ c c | c c }
type &twist & $\AddDat{\eTube}{B}{}$ & $\AddDat{\hTube}{B}{}$ \\ \hline
$\text{m}_{1}$ &$1$ & $\frac{1}{d \sqrt{3}}$ & $\frac{1}{2 \sqrt{3}}$\\ \hline
$\text{m}_{2}$&$1$ & $ \frac{d}{2 \sqrt{3}}$ & $\frac{-1}{2 \sqrt{3}}$ 
\end{tabu}
}}
\caption{
\label{MIdempotentsprime}
Quasiparticles of $\halfesix$ with bounding spin structure and trivial boundary condition.
The particle is a linear combination of the tubes shown at the top of the table multipled by the coefficients in the row.}
\end{table}

\begin{table}
{\tabulinesep=1.2mm
\begin{tabu}{ c c | c c c c }
type &spin &  $\AddDat{\vTube}{B}{} $ & $\AddDat{\tTube}{B}{} $ & $\AddDat{\XTube{1}{1}}{B}{} $ & $\AddDat{\XTube{1}{2}}{B}{} $ \\ \hline
$\text{m}_{2}$&$1$ & $\frac{1}{2 \sqrt{3}}$  & $\frac{1}{2 \sqrt{3}}$  & $\frac{e^{i \pi /4}}{2 \sqrt{d}}$ & $\frac{- e^{-i \pi/4} }{2 \sqrt{3d}}$ \\
$\Pi(\text{m}_2)$&$-1$  & $\frac{1}{2 \sqrt{3}}$  & $\frac{-1}{2 \sqrt{3}}$  &  $\frac{ e^{-i \pi/4} }{2 \sqrt{3d}}$& $\frac{-e^{i \pi /4}}{2 \sqrt{d}}$ \\ \hline
$\text{m}_3$&$e^{i \pi /3}$  & $\frac{1}{2 + d}$ & $\frac{e^{- i \pi /3}}{2+d}$ & $\frac{- e^{i \pi /12}}{\sqrt{3d}}$ &  \\
$\Pi(\text{m}_3)$&$-e^{i \pi /3}$  & $\frac{1}{2 + d}$ & $\frac{-e^{- i \pi /3}}{2+d} $&  & $\frac{e^{i \pi /12}}{\sqrt{3d}}$  \\ 
\end{tabu}
}
\caption{\label{MIdempotents} Quasiparticles of $\halfesix$ with bounding spin structures, and boundary condition $\rho$.
%All are m-type, with the horizontal lines separating the three quasiparticles. 
%Two of the three quasiparticles are two dimensional, formed out of two smaller simple modules. $\Pi$ is an odd isomorphism. 
The tube with a single $\rho$ line is a direct sum of four simple objects, 
two of which we name $m_2$ and $m_3$.
The other two are oddly isomorphic to $m_2$ and $m_3$ which we denote $\Pi(m_2)$ and $\Pi(m_3)$.
%\kw{It doesn't really make sense to talk about the ``dimension" of an object/quasiparticle.
%I would say that the non-simple object ``strand" is a direct sum of four simple objects, two of which we name
%$M_2$ and $M_3$.
%The other two subobjects are isomorphic (but oddly) to $M_2$ and $M_3$.
%There are various other ways one could say this.
%Let's discuss further on skype.}
%\dave{That sounds good to me.}
%\dave{I will do some table formatting in the near future. Note: old notation for odd isomorphism is $X(.)$}
}
\end{table}

\begin{comment}
\begin{table}
{{\tabulinesep=1.2mm
\begin{tabu}{ c c | c c }
type &twist & $\AddDat{\eTube}{B}{}$ & $\AddDat{\hTube}{B}{}$ \\ \hline
$\text{m}_{1}$ &$1$ & $\frac{1}{d \sqrt{3}}$ & $\frac{1}{2 \sqrt{3}}$\\ \hline
$\text{m}_{2}$&$1$ & $ \frac{d}{2 \sqrt{3}}$ & $\frac{-1}{2 \sqrt{3}}$ 
\end{tabu}
}
$\qquad$ {\tabulinesep=1.2mm
\begin{tabu}{ c c | c c c c }
type &twist &  $\AddDat{\vTube}{B}{} $ & $\AddDat{\tTube}{B}{} $ & $\AddDat{\XTube{1}{1}}{B}{} $ & $\AddDat{\XTube{1}{2}}{B}{} $ \\ \hline
$\text{m}_{2}$&$1$ & $\frac{1}{2 \sqrt{3}}$  & $\frac{1}{2 \sqrt{3}}$  & $\frac{e^{i \pi /4}}{2 \sqrt{d}}$ & $\frac{- e^{-i \pi/4} }{2 \sqrt{3d}}$ \\
$\Pi(\text{m}_2)$&$-1$  & $\frac{1}{2 \sqrt{3}}$  & $\frac{-1}{2 \sqrt{3}}$  &  $\frac{ e^{-i \pi/4} }{2 \sqrt{3d}}$& $\frac{-e^{i \pi /4}}{2 \sqrt{d}}$ \\ \hline
$\text{m}_3$&$e^{i \pi /3}$  & $\frac{1}{2 + d}$ & $\frac{e^{- i \pi /3}}{2+d}$ & $\frac{- e^{i \pi /12}}{\sqrt{3d}}$ &  \\
$\Pi(\text{m}_3)$&$-e^{i \pi /3}$  & $\frac{1}{2 + d}$ & $\frac{-e^{- i \pi /3}}{2+d} $&  & $\frac{e^{i \pi /12}}{\sqrt{3d}}$  \\ 
\end{tabu}
}
\caption{\label{MIdempotents} Quasiparticles of $\halfesix$ with bounding spin structures. 
All are m-type, with the horizontal lines separating the three quasiparticles. 
%Two of the three quasiparticles are two dimensional, formed out of two smaller simple modules. $\Pi$ is an odd isomorphism. 
The tube with a single $\rho$ line is a direct sum of four simple objects, 
two of which we name $M_2$ and $M_3$.
The other two are oddly isomorpich to $M_2$ and $M_3$ which we denote $\Pi(M_2)$ and $\Pi(M_3)$.
%\kw{It doesn't really make sense to talk about the ``dimension" of an object/quasiparticle.
%I would say that the non-simple object ``strand" is a direct sum of four simple objects, two of which we name
%$M_2$ and $M_3$.
%The other two subobjects are isomorphic (but oddly) to $M_2$ and $M_3$.
%There are various other ways one could say this.
%Let's discuss further on skype.}
%\dave{That sounds good to me.}
%\dave{I will do some table formatting in the near future. Note: old notation for odd isomorphism is $X(.)$}
}
}
\end{table}
\end{comment}

\begin{table}
{{\tabulinesep=1.2mm
\begin{tabu}{ c c | c }
type& twist & $\AddDat{\eTube}{N}{}$ \\ \hline
$\text{q}_{1}$ &$1$&1
\end{tabu}
}
$\qquad$
{\tabulinesep=1.2mm
\begin{tabu}{c  c | c c c c }
type& twist & $\AddDat{\vTube}{N}{}$ & $\AddDat{\tTube}{N}{}$  & $\AddDat{\XTube{1}{1}}{N}{}$ & $\AddDat{\XTube{1}{2}}{N}{}$ \\ \hline
$\text{q}_1$&$1$ & $ \frac{1}{d}$&$ \frac{1}{d} $& $\frac{- e^{i \pi/4}}{d\sqrt{d}}$ & $\frac{- e^{i \pi/4}}{d\sqrt{d}}$ \\ \hline
$\text{q}_2$&$-i$ &$ \frac{1}{2+d} $& $\frac{i}{2+d}$ & $\frac{i \gamma}{\sqrt{2+d}}$ & $\frac{i \gamma}{\sqrt{2+d}}$ \\ \hline 
$\text{m}_6$&$e^{5i \pi/6}$&$ \frac{1}{2+d} $&$ \frac{e^{-5i\pi/6}}{2+d} $ & $\frac{- \alpha e^{5 i \pi/6}}{\sqrt{2+d}}$ & $\frac{\beta e^{5 i \pi /6}}{\sqrt{2+d}}$ \\ 
$\Pi(\text{m}_6)$&$e^{5i \pi/6}$&$ \frac{1}{2+d} $&$ \frac{e^{-5i\pi/6}}{2+d} $ &  $\frac{\beta e^{5 i \pi /6}}{\sqrt{2+d}}$  &$\frac{- \alpha e^{5 i \pi/6}}{\sqrt{2+d}}$ 
\end{tabu}
 }
\caption{  \label{QIdempotents} Quasiparticles of $\frac{1}{2} E6$ with vortex (periodic) spin structures. Two are q-type, and one is m-type. The m-type particle is two-dimensional, consisting of of two smaller simple modules.  $\Pi$ is an odd isomorphism, and 
$\alpha = \frac{1}{2} \left( 1+ 1/\sqrt{2d+1} \right)$, and $\beta = \frac{1}{2} \left( 1- 1/\sqrt{2d+1} \right)$. 
%\dave{Note: old notation for odd isomorphism is $X(.)$}
}}
\end{table}

\begin{table}
\begin{flalign*} & \begin{array}{r@{ \quad \quad \quad}  c @{\quad \quad} c @{\quad \quad} c @{\quad \quad \quad \quad } c  @{\quad \quad} c @{\quad \quad} c  }
			\text{particle}				&m_1		&m_2		&m_3	&q_1	&q_2	&m_6 \\[.5ex] \hline \\ [-2ex]
			\text{quantum dimension}		&1			&1+d		&d		&2+d	&d		&d\\ [.5ex]
			\text{twist}				&1			&1		&e^{- i \pi/3}	&1	&i	&e^{-5 i \pi /6} \\
						\end{array} & \end{flalign*}
	\caption{The $\halfesix$ quantum dimensions and twists. 
	 The total quantum dimension is given by $\mcd = d\sqrt{6}$.}
	\label{halfesixFusionRules}
\end{table}


The fusion rules in the condensed theory can be calculated using \eqref{PplusQ} and \eqref{PminusQ}. 
We list the fusion rules in Table \ref{halfesixFusionRules}. 


			\begin{table}
\begin{flalign*} & \begin{array}{c@{ \quad}  | @{\quad \quad} l @{\quad \quad} l @{\quad \quad} l}
			\mca \tp \mca 		&m_1		&m_2		&m_3		\\[.5ex] \hline \\ [-2ex]
			m_1		 		&m_1		&m_2		&m_3		\\
			m_2		 		&m_2		&m_1\oplus \cc^{1|1}m_2	\oplus \cc^{1|1}m_3	&\cc^{1|1}  m_2 \oplus \cc^{0|1}m_3		\\
			m_3		 		&m_3		&\cc^{1|1} m_2 \oplus \cc^{0|1}m_3		&m_1 \oplus \cc^{0|1}m_2 \oplus m_3		\\
			\end{array} & \end{flalign*}
	
\begin{flalign*} & \begin{array}{c@{ \quad}  | @{\quad \quad} l @{\quad \quad} l @{\quad \quad} l}
			\mcv \tp \mca 		&m_1		&m_2									&m_3		\\[.5ex] \hline \\ [-2ex]
			q_1				&q_1		&\cc^{1|1}q_1 \oplus q_2 \oplus \cc^{1|1}m_6		&q_1 \oplus q_2 \oplus \cc^{1|1}m_6		\\
			q_2				&q_2		&q_1 \oplus \cc^{1|1} m_6					&q_1 \oplus  q_2		\\
			m_6		 		&m_6		&q_1 \oplus q_2 \oplus \cc^{0|1}m_6			&q_1 \oplus \cc^{0|1} m_6		\\
			\end{array} & \end{flalign*}		

\begin{flalign*} & \begin{array}{c@{ \quad}  | @{\quad \quad} l @{\quad \quad} l @{\quad \quad} l}
			\mcv \tp \mcv 		&q_1		&q_2		&m_6		\\[.5ex] \hline \\ [-2ex]
			q_1		 		&\cc^{1|1}m_1\oplus \cc^{2|2} m_2 \oplus \cc^{1|1}m_3		&\cc^{1|1}m_2 \oplus \cc^{1|1}m_3		&\cc^{1|1}m_2 \oplus \cc^{1|1}m_3		\\
			q_2		 		&\cc^{1|1}m_2 \oplus \cc^{1|1}m_3		&\cc^{1|1}m_1 \oplus m_3		&\cc^{1|1}m_2		\\
			m_6		 		&\cc^{1|1}m_2 \oplus \cc^{1|1}m_3 		&\cc^{1|1}m_2 		&\text{\textcolor{red}{$\cc^{0|1}$}}m_1 \oplus m_2 \oplus m_3		\\
			\end{array} & \end{flalign*}
	\caption{ \label{halfesixFusionRules} $\halfesix$ fusion rules. We define $\mca = \{ m_1, m_2, m_3\}$ and $\mcv = \{ q_1, q_2, q_6 \}$ as the set of non-vortex and vortex quasiparticles, respectively. 
	The $\cc^{p|q}$ denote the vector space associated with $\Delta^{ab}_c$ which is related to the fusion space through $V^{ab}_c =\Delta^{ab}_c \tp \text{End}(c)$.
	Notice that $m_6$ is self-dual and can only fuse oddly to the vacuum (the relevant fusion channel is marked in red), and hence it has an imaginary Frobenius-Schur indicator.
	\dave{I need to double check these.}
	\dave{I think they are now correct.
	But it may not be a bad idea to double-double check the $m_6$ fusion rules.}
	} 
\end{table}

It is useful to also keep track of the isomorphisms between the different representatives 
of an isomorphism class.
In the present case, we have,
\begin{align}
m_2 = \frac{e^{i \pi /4}}{2} \sqrt{\frac{d}{3}} \AddDatTorusDot{\TaddownTube{1}}{B}{}{}  \cdot  \AddDatTorusDot{\TadupTube{1}}{B}{}{}  \; = \frac{e^{- i \pi /4}}{2} \sqrt{\frac{d}{3}} \AddDatTorusDot{\TaddownTube{1}}{B}{}{\dota} \cdot \AddDatTorusDot{\TadupTube{1}}{B}{}{\dotc} 
\end{align}
\dave{Need to check the $\cdot$ notation matches up with what we used here. 
There may need to be an additional $-$ sign in second equation. }
which can be used to track the isomorphisms across the three representatives of $m_2$.
$m_3$ is oddly isomorphic to $\Pi(m_3)$ with the odd isomorphism 
given by sliding a fermion through the tube, and 
similarly for $m_6$.
Lastly we have,
\begin{align}
q_1 = \frac{e^{i \pi/4}}{\sqrt{d}} \AddDatTorusDot{\TaddownTube{1}}{N}{}{}  \cdot  \AddDatTorusDot{\TadupTube{1}}{N}{}{} 
\end{align}



%%%%%%%%%%%%%%%%%%%%%%%%%%%%%%
\subsection{Modular transformations}
%%%%%%%%%%%%%%%%%%%%%%%%%%%%%%

%%%%%%%%%%%%%%%%%%%%%%%%%%%%%
\subsubsection{Topological and idempotent bases}
%%%%%%%%%%%%%%%%%%%%%%%%%%%%%
There are two natural bases on the torus. 
One is the topological basis, corresponding to (\ref{TorusBasis}), the other is the 
idempotent basis (or quasiparticle basis) given in tables \ref{MIdempotents} and \ref{QIdempotents}.
We will compute the modular transformations in the topological basis first, and then change over to the idempotent basis.

We define the shorthand notation for the tubes as in tables \ref{rhotorho}--\ref{etoe}
%\begin{align}
%\begin{split}
%e =\; \AddDatTorus{\eTube}{}{}{}\quad &h =\; \AddDatTorus{\hTube}{}{}{} \quad v=\; \AddDatTorus{\vTube}{}{}{} \quad  t=\; \AddDatTorus{\tTube}{}{}{} \quad X = \; \AddDatTorus{\XTube{1}{1}}{}{}{}\\
%& \stackrel{\bullet}{h}\; =\; \AddDatTorusDot{\hTube}{}{}{\dotb} \quad \stackrel{\bullet}{v}\;=\; \AddDatTorusDot{\vTube}{}{}{\dotb} 
%\end{split}
%\label{TorusStates}
%\end{align}
We will then denote the spin structure by a subscript, for example,
\begin{align}
h_{BN} =\; \AddDatTorus{\hTube}{B}{N}{}
\end{align}




Every state on the torus can be expanded in terms of the above states, as long 
as we are also careful to use the local relations provided in Section \ref{TorusLocalRelations}.
Hence one can directly compute the change of basis by taking the particles in 
Tables \ref{MIdempotentsprime}, \ref{MIdempotents},  and \ref{QIdempotents} and 
projecting them onto the torus, and then modding out by the local relations described in Section \ref{TorusLocalRelations}.
The change of basis matrices in the spin sectors with at least one bounding cycle are 
\begin{align}
\label{VAA}
\left( \begin{matrix}
{m}_1\\
{m}_2\\
{m}_3\\
\end{matrix} \right)_{BB} 
&= \left( \begin{matrix}
\frac{1}{d\sqrt{3}} & 0 & \frac{1}{2 \sqrt{3}} \\
\frac{d}{2 \sqrt{3}} & 0 & - \frac{1}{2 \sqrt{3}} \\
- \frac{d}{2 \sqrt{3}} & \frac{1}{2} & \frac{1}{2 \sqrt{3}}
\end{matrix} \right)
\left( \begin{matrix}
e\\
v\\
h\\
\end{matrix} \right)_{BB}\\
\label{VAP}  
\left( \begin{matrix}
{m}_1\\
{m}_2\\
{m}_3\\
\end{matrix} \right)_{BN}
&= \left( \begin{matrix}
\frac{1}{d\sqrt{3}} & 0 & \frac{1}{2 \sqrt{3}} \\
\frac{d}{2 \sqrt{3}} & 0 & - \frac{1}{2 \sqrt{3}} \\
\frac{d e^{2 \pi i/3}}{2 \sqrt{3}} & \frac{e^{- i \pi /3}}{2} & \frac{e^{- i \pi /3}}{2 \sqrt{3}}\\
\end{matrix} \right)
\left( \begin{matrix}
e\\
t\\
h\\
\end{matrix} \right)_{BN} \\
\label{VPA} 
\left( \begin{matrix}
{q}_1\\
{q}_2\\
{m}_6\\
\end{matrix} \right)_{NB}
&= \left( \begin{matrix}
1 & 0 & 0 \\
\sqrt{\frac{1+d}{3}} e^{-3i \pi /4} & \frac{e^{i \pi /6}}{\sqrt{3}} & \frac{e^{i \pi /3}}{\sqrt{3}} \\
\frac{e^{7 i \pi /12}}{\sqrt{6}} & \frac{e^{- i \pi / 6}}{2 \sqrt{3}} & \frac{e^{-2 i \pi /3}}{2 \sqrt{3}}\\
\end{matrix} \right)
\left( \begin{matrix}
e\\
v\\
t\\
\end{matrix} \right)_{NB} \\
\end{align}

For the non-bounding $NN$ spin structure the q-type idempotents need to be closed up into a torus with an odd isomorphism
% will be odd (in particular, any q-type idempotent glued around a non-bounding spin structure is odd).
%\dave{Need to come back to this.}
We define $\left[ \stackrel{\bullet}{q}_{i}\right]^2  = q_i$. 
This has a $\pm$ ambiguity, we denote the $\pm$ signs by $\sigma_i$. 
We can also change to the idempotent basis with,
\begin{align}
\label{VPP}
\left( \begin{matrix}
\stackrel{\bullet}{q}_{1}\\
\stackrel{\bullet}{q}_{2}\\
\text{m}_6 \\ 
\end{matrix} \right) \; =\;
\left( \begin{matrix}
\sigma_1 e^{- i \pi /4} &0&0\\
0&\sigma_2 e^{- i \pi /4} &0\\
0&0&1 \\
\end{matrix} \right)
\left( \begin{matrix}
\frac{e^{- i \pi /4}}{\sqrt{2}} & 0&0 \\
- \frac{e^{- i \pi /4}}{\sqrt{2}} & 1 & 0 \\
0 & 0& - \frac{e^{5 i \pi /6}}{\sqrt{2 + d}}\\
\end{matrix} \right)
\left( \begin{matrix}
\stackrel{\bullet}{h} \\
\stackrel{\bullet}{v} \\
X \\
\end{matrix} \right)
\end{align}

The q-type idempotents have norm square of $2$ (due to their two-dimensional endomorphism algebras), 
opposed to the m-type idempotents that have norm square of $1$. 
To ensure that the modular matrices are unitary, we adjust for this by defining $\widehat{q} = q/\sqrt{2}$. 
When written in terms of the $\widehat{q}$, the modular matrices are unitary. 


%%%%%%%%%%%%%%%%%%%%%%
\subsubsection{$S$ transformation}
%%%%%%%%%%%%%%%%%%%%%%

The $S$ transformation exchanges the longitudinal and meridional cycles of the torus.
Since we are drawing the tori as annuli with their boundaries identified, the $S$ transformation looks like
\begin{align}
\xymatrix @!0 @M=1mm  @C=20mm{
S^{XY \rightarrow \widetilde{X}\widetilde{Y}}:& \mch^{T^2}_{(X,Y)} \ar[rr] && \mch^{T^2}_{(\widetilde{X},\widetilde{Y})} & \\
&&&&\\
& \AnnularTubex{\AnnularTubeNoIndex}{}{}{\psi}{X}{Y}\ar@{|->}[rr] &&\SAnnulusx{\psi}{\widetilde{X}}{\widetilde{Y}}\; & 
 }
 \label{STopologicalBasis}
\end{align}
with the transformed spin structure $(\widetilde{X},\widetilde{Y})$ being found with the aid of Figure  \ref{spin_str_mapping_class_group}. 
In terms of the matrix elements of $S$, we have
\begin{align}
\SAnnulusx{\psi}{\widetilde{X}}{\widetilde{Y}} \;= \sum_{\lambda \in A(T^2,\widetilde{X}\widetilde{Y})}  S^{XY \rightarrow \widetilde{X}\widetilde{Y}}_{\psi \lambda} \AnnularTubex{\AnnularTubeNoIndex}{}{}{\lambda}{\widetilde{X}}{\widetilde{Y}} \\
\end{align}
Where we have taken $\psi,\lambda  \in \bigoplus_{ab} V^{aba^*}_b$ modulo local relations.

We can now work out the $S$-matrix for each spin structure in the topological basis, and then change over to the idempotent basis. 
The calculation is the same in each case, we find the linear map on the in to the 
topological basis based on (\ref{STopologicalBasis}), and then change back to the particle basis using (\ref{VPA}--\ref{VPP}).

For the $(B,B)$ spin structure one simply finds that $v$ and $h$ are interchanged so that,
\begin{align}
\left( \begin{matrix}
e\\
v\\
h\\
\end{matrix} \right)_{BB}
\xrightarrow{S^{BB \rightarrow BB}}
\left(\begin{matrix}
1& 0& 0 \\
0& 0&1  \\
0&1 &0 \\ 
\end{matrix} \right)
\left( \begin{matrix}
e\\
v\\
h\\
\end{matrix} \right)_{BB}
\end{align}
We can now write down the $S$-matrix in the idempotent basis using the change of basis in (\ref{VAA})
\begin{align}
\left( \begin{matrix}
m_1\\
m_2\\
m_3\\
\end{matrix} \right)_{BB} \xrightarrow{S^{BB \rightarrow BB}} 
\frac{1}{\sqrt{3}}\left( \begin{matrix}
\frac{1}{d} & \frac{d}{2} & 1\\ 
\frac{d}{2} & \frac{1}{d} & -1\\
1 & -1 & 1\\
\end{matrix} \right)
\left( \begin{matrix}
m_1\\
m_2\\
m_3\\
\end{matrix} \right)_{BB}
\end{align}

Next we compute the $S$-matrix elements that transition between the $(B,N)$ and $(N,B)$ tori.
We first find the action of S on the $(B,N)$ torus:
\begin{align}
\left( \begin{matrix}
e\\
h\\
v\\
\end{matrix} \right)_{BN} 
 \xrightarrow{S^{BN\rightarrow NB}}
\left( \begin{matrix}
1&0&0\\
0&1&0\\
d e^{-i \pi/6} & e^{5 i \pi/6}  & e^{2 \pi i /3}\\
\end{matrix} \right)
\left( \begin{matrix}
e\\
v\\
t\\
\end{matrix} \right)_{NB}
\end{align}
which can be written in the idempotent basis as
\begin{align}\left( \begin{matrix}
{m}_1\\
{m}_2\\
{m}_3\\
\end{matrix} \right)_{BN}
\xrightarrow{S^{BN \rightarrow NB}}
\left( \begin{matrix}
\frac{1}{2} & \frac{1}{2 \sqrt{3}} &  \frac{1}{\sqrt{3}} \\
\frac{1}{2} & - \frac{1}{2\sqrt{3}} & -\frac{1}{\sqrt{3}} \\
0& \frac{1}{\sqrt{3}} & -\frac{1}{\sqrt{3}} \\
\end{matrix} \right)
\left( \begin{matrix}
{q}_1\\
{q}_2\\
{m}_6\\
\end{matrix} \right)_{NB}.
\end{align}
Similarly we can work out the $S$-matrix in the topological basis for the $(N,B)$ torus,
\begin{align}
\left( \begin{matrix}
e \\ 
v\\ 
t\\ 
\end{matrix} \right)_{NB}
  \xrightarrow{S^{NB \rightarrow BN}}
\left( \begin{matrix}
1&0&0\\
0&1&0\\
de^{i \pi/6} & e^{-5 i \pi /6} & e^{-2 i \pi /3}
\end{matrix} \right)
\left( \begin{matrix}
e \\
h\\ 
t\\ 
\end{matrix} \right)_{BN} 
\end{align}
And again we can write this in the idempotent basis:
\begin{align}
\left( \begin{matrix}
{q}_1\\
{q}_2\\
{m}_6\\
\end{matrix} \right)_{NB}
\xrightarrow{S^{NB \rightarrow BN}}
\left( \begin{matrix}
1& 1& 0 \\
\frac{1}{\sqrt{3}} & - \frac{1}{\sqrt{3}} & \frac{2}{\sqrt{3}} \\
\frac{1}{\sqrt{3}} & - \frac{1}{\sqrt{3}} & - \frac{1}{\sqrt{3}} \\
\end{matrix} \right)
\left( \begin{matrix}
{m}_1\\
{m}_2\\
{m}_3\\
\end{matrix} \right)_{BN}
\end{align}
Notice that the $S$-matrix is invertible, but not unitary. 
This is because we didn't normalize our idempotents appropriately. 
As mentioned earlier, we write the normalized $Q$ idempotents with a hat, $\widehat{Q}_i = Q_i /\sqrt{2}$.
Once doing so we find the appropriately normalized $S$-matrix is given by
\begin{align}\left( \begin{matrix}
{m}_1\\
{m}_2\\
{m}_3\\
\end{matrix} \right)_{BN}
\xrightarrow{S^{BN \rightarrow NB}}
\left( \begin{matrix}
\frac{1}{\sqrt{2}} & \frac{1}{\sqrt{6}} &  \frac{1}{\sqrt{3}} \\
\frac{1}{\sqrt{2}} & - \frac{1}{\sqrt{6}} & -\frac{1}{\sqrt{3}} \\
0& \sqrt{\frac{2}{3}} & -\frac{1}{\sqrt{3}} \\
\end{matrix} \right)
\left( \begin{matrix}
\widehat{{q}}_1\\
\widehat{{q}}_2\\
{m}_6\\
\end{matrix} \right)_{NB}
\end{align}
\begin{align}
\left( \begin{matrix}
\widehat{{q}}_1\\
\widehat{{q}}_2\\
{m}_6\\
\end{matrix} \right)_{NB}
\xrightarrow{S^{NB \rightarrow BN}}
\left( \begin{matrix}
\frac{1}{\sqrt{2}}& \frac{1}{\sqrt{2}}& 0 \\
\frac{1}{\sqrt{6}} & - \frac{1}{\sqrt{6}} & \sqrt{\frac{2}{3}} \\
\frac{1}{\sqrt{3}} & - \frac{1}{\sqrt{3}} & - \frac{1}{\sqrt{3}} \\
\end{matrix} \right)
\left( \begin{matrix}
{m}_1\\
{m}_2\\
{m}_3\\
\end{matrix} \right)_{BN}
\end{align}
Notice that the matrix is symmetric, and unitary.

Lastly, we can work out the $S$-matrix on the non-bounding torus.
\begin{align}
\left( \begin{matrix}
\stackrel{\bullet}{h} \\
\stackrel{\bullet}{v} \\
X \\
\end{matrix} \right)_{NN}
\xrightarrow{S^{NN\rightarrow NN}} 
\left( \begin{matrix}
0&i  &0 \\ 
1&0 &0 \\
0&0& -i \\
\end{matrix} \right)
\left( \begin{matrix}
\stackrel{\bullet}{h} \\
\stackrel{\bullet}{v} \\
X \\\end{matrix} \right)_{NN}
\end{align}
In the idempotent basis, this is
\begin{align}   \label{hE6_S_NN}
\left( \begin{matrix}
\stackrel{\bullet}{q}_{1}\\
\stackrel{\bullet}{q}_{2}\\
{m}_6 \\ 
\end{matrix} \right)_{NN}
 \xrightarrow{S^{NN \rightarrow NN}}
\frac{e^{i \pi /4}}{\sqrt{2}}\left( \begin{matrix} 
1&\Sigma &0 \\
\Sigma &-1&0\\
0&0& -\sqrt{2} e^{i \pi /4}\\
\end{matrix} \right)
\left( \begin{matrix}
\stackrel{\bullet}{q}_{1}\\
\stackrel{\bullet}{q}_{2}\\
{m}_6 \\ 
\end{matrix} \right)_{NN}
\quad \quad \text{where $\Sigma = \sigma_1 \sigma_2$.}
\end{align}
Note that $S^{NN\ra NN}$ splits as $S^{NN\ra NN}_q \oplus S^{NN\ra NN}_m$ into blocks which operate 
on q-type and m-type particles, as it must: 
the basis vectors coming from m-type (q-type) idempotents are even (odd), and $S$ preserves fermion parity.
%the $S$-transformations for the non-bounding torus cannot change the type of a given 
%quasiparticle. 

In summary, we have \eqref{hE6_S_NN} together with 
\begin{align}
\left(\begin{matrix}
\left( \begin{matrix}
{m}_1\\
{m}_2\\
{m}_3\\
\end{matrix} \right)_{BN} \\
\\
\left( \begin{matrix}
\widehat{{q}}_1\\
\widehat{{q}}_2\\
{m}_6\\
\end{matrix} \right)_{NB}\\
\\
\left( \begin{matrix}
{m}_1\\
{m}_2\\
{m}_3\\
\end{matrix} \right)_{BB} \\
\end{matrix} \right)
\xrightarrow{S} \left( \begin{matrix}
&&&			\frac{1}{\sqrt{2}} & \frac{1}{\sqrt{6}} &  \frac{1}{\sqrt{3}} &			&&\\
&&&			\frac{1}{\sqrt{2}} & - \frac{1}{\sqrt{6}} & -\frac{1}{\sqrt{3}}& 			&&\\
&&&			0& \sqrt{\frac{2}{3}} & -\frac{1}{\sqrt{3}}& 			&&\\
\frac{1}{\sqrt{2}}& \frac{1}{\sqrt{2}}& 0&			&&& 			&&\\
\frac{1}{\sqrt{6}} & - \frac{1}{\sqrt{6}} & \sqrt{\frac{2}{3}}&			&&& 			&&\\
\frac{1}{\sqrt{3}} & - \frac{1}{\sqrt{3}} & - \frac{1}{\sqrt{3}}&			&&& 			&&\\
&&&			&&&			\frac{1}{d\sqrt{3}} & \frac{d}{2\sqrt{3}} & \frac{1}{\sqrt{3}}\\
&&&			&&& 			\frac{d}{2\sqrt{3}} & \frac{1}{d\sqrt{3}} & -\frac{1}{\sqrt{3}}\\
&&&			&&& 			\frac{1}{\sqrt{3}} & -\frac{1}{\sqrt{3}} & \frac{1}{\sqrt{3}}\\
\end{matrix} \right)
\left(\begin{matrix}
\left( \begin{matrix}
{m}_1\\
{m}_2\\
{m}_3\\
\end{matrix} \right)_{BN} \\
\\
\left( \begin{matrix}
\widehat{{q}}_1\\
\widehat{{q}}_2\\
{m}_6\\
\end{matrix} \right)_{NB}\\
\\
\left( \begin{matrix}
{m}_1\\
{m}_2\\
{m}_3\\
\end{matrix} \right)_{BB} \\
\end{matrix} \right)
\end{align}


%%%%%%%%%%%%%%%%%%%%%%%
\subsubsection{$T$ transformation}
%%%%%%%%%%%%%%%%%%%%%%%

The Dehn twist ($T$-transformation) corresponds to cutting the torus open along one cycle, 
applying a full $2\pi$ rotation and then gluing the torus back together along that cycle.
\begin{align}
\xymatrix @!0 @M=1mm  @C=20mm{
T^{XY \rightarrow \widetilde{X}\widetilde{Y}}:& \mch^{T^2}_{(X,Y)} \ar[rr] && \mch^{T^2}_{(\widetilde{X},\widetilde{Y})} & \\
&&&&\\
& \AnnularTubex{\AnnularTubeNoIndex}{}{}{\psi}{X}{Y}\ar@{|->}[rr] &&\TAnnulusx{\psi}{\widetilde{X}}{\widetilde{Y}}\; & 
 }
 \label{TTopologicalBasis}
\end{align}
with the spin structure transforming according to Figure \ref{spin_str_mapping_class_group}. 
In terms of the matrix elements of $T$, 
\begin{align}
\TAnnulusx{\psi}{\widetilde{X}}{\widetilde{Y}} \;= \sum_{\lambda \in \mch^{T^2}_{(\widetilde{X},\widetilde{Y})}}  T^{XY \rightarrow \widetilde{X}\widetilde{Y}}_{\psi \lambda} \AnnularTubex{\AnnularTubeNoIndex}{}{}{\lambda}{\widetilde{X}}{\widetilde{Y}} \\
\end{align}

%\dave{Depending how we decide on twists, we should make the appropriate modifications here.}
%The application of a Dehn twist acts like the multiplication of a tube in the tube algebra, along with gluing the boundaries together.
Aside from spin structure considerations, all idempotents are eigenstates of the Dehn twist,
with $T$ acting diagonally within each spin-structure block. 

\begin{comment}
\ethan{This is redundant since we have a summary matrix, which is more readable, directly after this}
\begin{align}
\left( \begin{matrix}
\widehat{\text{Q}}_1\\
\widehat{\text{Q}}_2\\
\text{M}_6\\
\end{matrix} \right)_{NB} \xrightarrow{T^{NB \rightarrow NB}}
\left( \begin{matrix}
1 & 0 & 0\\
0 & -i & 0 \\
0 & 0& e^{5i \pi /6} \\
\end{matrix} \right)
\left( \begin{matrix}
\widehat{\text{Q}}_1\\
\widehat{\text{Q}}_2\\
\text{M}_6\\
\end{matrix} \right)_{NB}
\end{align}

\begin{align}
\left( \begin{matrix}
\text{M}_1\\
\text{M}_2\\
\text{M}_3\\
\end{matrix} \right)_{BB} 
\xrightarrow{T_{BB \rightarrow BN}} 
\left(\begin{matrix} 
1 & 0 &0 \\
0 & 1 & 0 \\
0 & 0 & e^{i \pi /3} \\
\end{matrix} \right)
\left( \begin{matrix}
\text{M}_1\\
\text{M}_2\\
\text{M}_3\\
\end{matrix} \right)_{BN}
\end{align}

\begin{align}
\left( \begin{matrix}
\text{M}_1\\
\text{M}_2\\
\text{M}_3\\
\end{matrix} \right)_{BN} 
\xrightarrow{T_{BN \rightarrow BB}} 
\left(\begin{matrix} 
1 & 0 &0 \\
0 & 1 & 0 \\
0 & 0 & e^{i \pi /3} \\
\end{matrix} \right)
\left( \begin{matrix}
\text{M}_1\\
\text{M}_2\\
\text{M}_3\\
\end{matrix} \right)_{BB}
\end{align}
\end{comment}

To find the eigenvalues, we compute the Dehn twist in the topological basis and then change back to the 
the idempotent basis as usual. 
We find
\begin{align}
\left( \begin{matrix}
\stackrel{\bullet}{q}_{1}\\
\stackrel{\bullet}{q}_{2}\\
{m}_6 \\ 
\end{matrix} \right)_{NN} \xrightarrow{T^{NN\rightarrow NN}}
\left( \begin{matrix} 
1 & 0&0 \\
0 & -i & 0 \\
0 & 0& e^{5 i \pi /6}\\
\end{matrix} \right) 
\left( \begin{matrix}
\stackrel{\bullet}{q}_{1}\\
\stackrel{\bullet}{q}_{2}\\
{m}_6 \\ 
\end{matrix} \right)_{NN}
\end{align}
for the $NN$ torus, and 
\begin{align}
\left(\begin{matrix}
\left( \begin{matrix}
{m}_1\\
{m}_2\\
{m}_3\\
\end{matrix} \right)_{BN} \\
\\
\left( \begin{matrix}
\widehat{{q}}_1\\
\widehat{{q}}_2\\
{m}_6\\
\end{matrix} \right)_{NB}\\
\\
\left( \begin{matrix}
{m}_1\\
{m}_2\\
{m}_3\\
\end{matrix} \right)_{BB} \\
\end{matrix} \right)
\xrightarrow{T} \left( \begin{matrix}
&&&			&&&			1&0&0	\\
&&&			&&&			0&1&0	\\
&&&			&&&			0&0&e^{i \pi/3}	\\
&&&			1&0&0&			&&	\\
&&&			0&-i&0&			&&	\\
&&&			0&0&e^{5 i \pi /6}&			&&	\\
1&0&0&			&&&			&&	\\
0&1&0&			&&&			&&	\\
0&0&e^{i \pi /3}&			&&&			&&	\\		
\end{matrix} \right)
\left(\begin{matrix}
\left( \begin{matrix}
{m}_1\\
{m}_2\\
{m}_3\\
\end{matrix} \right)_{BN} \\
\\
\left( \begin{matrix}
\widehat{{q}}_1\\
\widehat{{q}}_2\\
{m}_6\\
\end{matrix} \right)_{NB}\\
\\
\left( \begin{matrix}
{m}_1\\
{m}_2\\
{m}_3\\
\end{matrix} \right)_{BB} \\
\end{matrix} \right)
\end{align}
for the spin tori with at least one bounding cycle.
One can check that the modular matrices defined above satisfy $S^4 = (-1)^F$ and $(ST)^3 = (-1)^F$. 
%\kw{Don't we want to replace $id$ by $(-1)^F$?}
%\kw{Also should be $S^4 = \cdots$, not $S^2 = \cdots$.}
%\ethan{yep}

%\dave{ 
%Can ask: is it poassible 1d anyon spin chain built from this category host more exotic excitations than Majorana zero modes? 
%The folk lore (which stands on more rigoruous grounds, I'm just not familiar with the literature) is that this is impossible unless it ocurrs at the boundary of a 2+1d system.
%The fact that we have to use a half braid to define the condensation suggests that this kind of spin chain could only live at the boundary of some other 2+1D phase.
%}



%\kwsep


%\dave{Also I think I have a good idea how to compute the center of $SU(2)_{2+4k}$ now. 
%We could add an appendix with some of the details, e.g., 6j symbols and particles/fusion rules of the parent theory, condensed theory and formula for the idempotents (there is a nice way to do this in terms of the idempotents of the parent theory), could also try to work out a formula for the S and T matrices of the center.}



