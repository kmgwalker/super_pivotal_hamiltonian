%auto-ignore
%      this ensures the arxiv doesn't try to start TeXing here.
%!TEX root = super_lattice_models_draft.tex
%      prev line helps TeXShop do the right thing


%%%%%%%%%%%%%%%%%%%%%
\section{Spin and pin structures} \label{spin_and_pin}
%%%%%%%%%%%%%%%%%%%%%

%\kw{I think this section could be condensed a bit.
%Also, we should try to use standard, well-established terminology for spin structure stuff where possible.
%}
%
In this appendix we review basic definitions and properties of spin and pin structures.

\medskip

Roughly, a spin structure on an oriented $n$-manifold $M$ is a specification for how fermions pick up phases of $-1$ as they move around $M$.
Locally, we of course require that fermions pick up a minus sign when rotated through $2\pi$,
or when two fermions are exchanged within a small neighborhood of $M$.
But if a fermion moves along a non-contractible loop in $M$, it is not clear what sign we should assign.
A spin structure on $M$ is a consistent set of answers to all possible questions of this form.

More formally, we can define a spin structure on $M$ to be a double covering of the frame bundle $F(M)$, such that on each fiber of $F(M)$
the covering is isomorphic to the standard double covering $Spin(n) \to SO(n)$.
Such double coverings correspond to cohomology classes in $H^1(F(M), \zz/2)$ which restrict to the generator of $H^1(SO(n), \zz/2)$ on each fiber.
It follows that the difference between any two spin structures is canonically identified
with an element of $H^1(M, \zz/2)$; spin structures on $M$ form a $H^1(M, \zz/2)$-torsor.
In particular, the number of distinct spin structures on $M$ is given by the number of elements in $H^1(M, \zz/2)$.
(Things need to said differently when $n<2$.
One way around this problem is to work with the stabilized frame bundle for 0- and 1-manifolds.)

It is important to note that there is no canonical correspondence between spin structures on $M$ and $H^1(M, \zz/2)$; simply naming
a cohomology class does not pick out a spin structure.

One way to specify a spin structure is to specify a framing on the 1-skeleton of a cell decomposition of $M$.
We can think of this framing as an embedded graph in the frame bundle $F(M)$, and the spin structure is uniquely determined by requiring
that the double covering of $F(M)$ be trivial when restricted to this graph.
We can also think of the 1-skeleton framing as specifying a collection of possible fermion paths which do not pick up a factor of $-1$.

In this paper, most of the diagrams we draw are embedded in the page/blackboard/$\rr^2$.
$\rr^2$ has a standard framing, so the 1-skeleton of any such diagram inherits a framing, and unless stated otherwise we work in the spin
structure associated to that framing.
In practice, this means that fermions pick up a minus sign only when their framing rotates with respect to the page.

If we have designated a reference spin structure on $M$ (for example, the blackboard spin struture), 
then any other spin structure can be specified by giving an
element of $H^1(M, \zz/2)$, or equivalently by giving the Poincar\'e dual element in $H_1(M, \zz/2)$.
In this context, we refer to the Poincar\'e dual homology class as a ``branch cut".
Fermions obey the rules of the reference spin structure, except that they pick up a $-1$ whenever they cross the branch cut.

\medskip

For unoriented manifolds, we must replace $SO(n)$ with $O(n)$ and replace $Spin(n)$ with a $\zz/2$ extension of $O(n)$.
There are two such extensions, called $Pin_+(n)$ and $Pin_-(n)$.
In $Pin_+(n)$, lifts of reflections in $O(n)$ square to the identity, while in $Pin_-(n)$ such lifted reflections square to the ``spin flip"
in $Spin(n)$.
(When $n=1$, we have $O(1)\cong\zz/2$, $Pin_+(1)\cong\zz/2\times\zz/2$, and $Pin_-(1)\cong\zz/4$.)
Roughly speaking, in pin+ manifolds reflecting a fermion twice returns us to the same state, while in pin$-$ manifolds reflecting a fermion twice
picks up a factor of $-1$.

None of the examples in this paper have a pin$-$ reflection structure; we only work with pin+ structures.

Specifying pin+ structures in terms of framings (as we did above for spin structures) is a little awkward.
It is usually more convenient to use lifts of classifying maps or, when $n=2$, quadratic refinements of intersection pairings.

