
%%%%%%%%%%%%%%%%%%%%%%%%%%%
\section{$\frac{1}{2}$E6 data}
%%%%%%%%%%%%%%%%%%%%%%%%%%%

%%%%%%%%%%%%%%%%
\subsection{Associators}
%%%%%%%%%%%%%%%%
\label{E6Fsymbols}
There are four unitary solutions to the pentagon equation for $\frac{1}{2}$E6. They split into two sets, one pair has all positive quantum dimensions, while the other has negative quantum dimension on the $x$ particle. 
The solutions in each set are related by complex conjugation. 
Here we present one of the solutions with positive quantum dimensions on all particles. 
Several of the $F$ symbols are trivial,
\begin{align}
F^{yyy}_y = F^{xyy}_x = F^{yyx}_x = F^{xxy}_y  = F^{yxx}_y = F^{xyx}_{\mathds{1}} = F^{xxy}_{\mathds{1}} = F^{yxx}_{\mathds{1}}= 1
\end{align}
Let $v_1, v_2$ be orthogonal unit vectors for the two-dimensional splitting space $V^{xx}_x$. 
We define the $F$ symbols acting on these vectors by $V^{xy}_x \tp V^{xx}_x  = F^{xyx}_x V^{xx}_x \tp V^{yx}_x$ in the basis $(v_1, v_2)^{T}$.
Explicitly we have,
\begin{align}
&F^{xyx}_x = \left(\begin{matrix}
0&1\\
1&0
\end{matrix} \right) \quad
F^{xxy}_x = \left(\begin{matrix}
0&-i\\
i&0
\end{matrix} \right) \quad
F^{yxx}_x = \left(\begin{matrix}
1&0\\
0&-1
\end{matrix} \right) 
\\
\\
&F^{xxx}_{\mathds{1}} = c_2^* \left(\begin{matrix}
1&-i\\
1&i
\end{matrix} \right) \quad
F^{xxx}_y = c_2^*\left(\begin{matrix}
1&-i\\
-1&-i
\end{matrix} \right), \quad \quad c_2  = \frac{e^{7 i \pi/12}}{\sqrt{2}}
\end{align}

Lastly we have the $F$ symbol whose four external legs are all labeled by $x$.
We write this $F$ symbol down in the basis $(0,y, v_1\tp v_1,v_1 \tp v_2, v_2 \tp v_1, v_2 \tp v_2)$, where we have 
\begin{align}
F^{xxx}_x  = 
 \left(\begin{matrix}
\frac{1}{d} & \frac{1}{d} & \frac{c_1^*}{\sqrt{2}} & \frac{c_1^*}{\sqrt{2}} & \frac{c_1^*}{\sqrt{2}} & - \frac{c_1^*}{\sqrt{2}} \\
\frac{1}{d} & - \frac{1}{d} & \frac{c_1^*}{\sqrt{2}} & -\frac{c_1^*}{\sqrt{2}} & \frac{c_1^*}{\sqrt{2}} & \frac{c_1^*}{\sqrt{2}} \\
\frac{e^{- i \pi/4}}{\sqrt{d}} & 0 & -\frac{1}{d} & 0 & -i c_4^* & 0 \\
0 & \frac{e^{i \pi /4}}{\sqrt{d}} & 0 &c_4^* & 0 & \frac{ i}{d} \\
0 & \frac{e^{- i \pi /4}}{\sqrt{d}} & 0 & -\frac{1}{d} & 0 & i c_4^* \\ 
\frac{e^{i \pi /4}}{\sqrt{d}} & 0 & c_4^* & 0 & - \frac{i}{d} & 0
\end{matrix} \right) \\
d = 1 + \sqrt{3} \quad \quad c_1 = \frac{e^{-5 i \pi/6} }{\sqrt{d}} \quad \quad c_4 = \frac{e^{-i \pi/4}}{\sqrt{2}}
\end{align}

\newpage
\subsection{Idempotents}
\label{IdempotentsHalfESix}
%\dave{These are data from an older version of the notes. We don't have to include them but they could be useful for other people. }

\begin{table}
{\tabulinesep=1.2mm
\begin{tabu}{ c c | c c c c }
particle & spin &$e$&$l_x$&$l_y$& \\ \hline
$1$ & $
       1       

$ &
$
\frac{1}{2+d^2}
$
 &
$
\frac{1}{2 \sqrt3}
$
 &
$
\frac{1}{2+d^2}
$
 &
\\
$2$ & $
       1       

$ &
$
\frac{1}{2}
$
 &
$
$
 &
$ -
\frac{1}{2}
$
 &
\\
$3$ & $
       1       

$ &
$
\frac{d}{4 \sqrt{3}}
$
 &
$ -
\frac{1}{2 \sqrt3}
$
 &
$
\frac{d}{4 \sqrt{3}}
$
 &
\end{tabu}
\caption{Charge $0$ idempotents. We have used the notation $e$ = empty diagram, $l_x$ = tube with $x$-loop around the meridional cycle, and $l_y$ = tube with a $y$-loop around the meridional cycle.}
}
{\tabulinesep=1.2mm
\begin{tabu}{ c c | c c c c }
particle & spin &$t_y$&$t_y l_x$&$v_y$& \\ \hline
$2$ 
& $       1       $ 
&$\frac{1}{2}$
 &$$
 &$\frac{1}{2}$
 &
\\
$4$ & $
       1/2     

$ &
$ -
\frac{1}{2+d^2}
$
 &
$ e^{\frac{6i\pi}{12}}
\frac{1}{2 \sqrt3}
$
 &
$
\frac{1}{2+d^2}
$
 &
\\
$5$ & $
       1/2     

$ &
$ -
\frac{d}{4 \sqrt{3}}
$
 &
$ e^{\frac{-6i\pi}{12}}
\frac{1}{2 \sqrt3}
$
 &
$
\frac{d}{4 \sqrt{3}}
$
 &
\end{tabu}
}
\caption{Charge $2$ idempotents. We have used the notation $v_y$ = tube with a longitudinal $y$ loop, $t_y$ = tube with a $y$ line wrapping around both cycles, and $t_y l_x$ = tube with a $y$ line around the longitudinal cycle and an $x$ line around the meridional cycle.}
\end{table}

%%%%%%%%%%%%%%%%%%%%%%%%%%%%%%%%%%%%%%%%%%%%%%%%%%%%%%%%%%%%%%%%%%%%%%%%%%%%%%%%

Here we identify the primitive idempotents of the tube algebra of the $\halfesix$ theory in the notation of \cite{Hong2008}. 
The spins here differ from those in \cite{Hong2008} by complex conjugation. Also, note that at the level of fusion rules and spins, particles $6$ and $8$ are identical---extra data is needed to distinguish them\footnote{In fact, the primitive idempotents $6$ and $8$ in the charge $1$ sector are interchanged under the transformation $(11, 12, 21, 22) \mapsto (12,11,-22,-21)$. }.

\begin{align}
\xymatrix{
1&2&3&4&5&6&7&8&9 &10 \\
\mathds{1} & W& U & Y& V& X_{2,3} &X_1 & X_{3,2}& X_5 & X_4
}
\end{align}

\begin{align}
\xymatrix @!0 @M=2mm @R=22mm @C=19mm{
W \ar@/^2pc/[rr]^{w_{02}} \ar@<.5ex>[r]^{w_{01}}
& W \ar@<.5ex>[l]^{w_{10}} \ar@<.5ex>[r]^{w_{12}}
& W\ar@<.5ex>[l]^{w_{21}} \ar@/^2pc/[ll]^{w_{20}}
}\qquad
\xymatrix @!0 @M=2mm @R=22mm @C=19mm{
U\ar@<.5ex>[r]^{u_{01}}
& U \ar@<.5ex>[l]^{u_{10}} 
}\qquad
\xymatrix @!0 @M=2mm @R=22mm @C=19mm{
V\ar@<.5ex>[r]^{v_{12}}
& V \ar@<.5ex>[l]^{v_{21}} 
}
\end{align}
Where
\begin{align}
w_{01} &= \frac{i}{\sqrt{2 \sqrt{2d}}}(t_{0111;11} - t_{0111;12}) \\
w_{10}  &= \frac{-i}{\sqrt{2\sqrt{2d}}} (t_{1101;11}-t_{1101;21}) \\
w_{12} & = \frac{1}{(8d)^{\frac14}}(t_{1121;11} + t_{1121;21}) \\
w_{21} & = \frac{1}{(8d)^{\frac14}}(t_{2111;11} + t_{2111;12}) \\
w_{02} & \frac{-e^{-i\pi/4}}{\sqrt{2}} t_{0121;11}\\
w_{20} & \frac{-e^{i\pi/4}}{\sqrt{2}} t_{2101;11}
\end{align}
\begin{align}
u_{01}  & = \frac{1}{2}\sqrt{\frac{d}{6}}( t_{0111;11}+t_{0111;12})\\
u_{10} & = \frac{1}{2}\sqrt{\frac{d}{6}}(t_{1101;11}+t_{1101;21})\\
v_{12} & = \frac{i}{2}\left( \frac{d}{6} \right)^{\frac14} (t_{1121;11}-t_{1121;21})\\
v_{21} & = \frac{i}{2}\left( \frac{d}{6} \right)^{\frac14} (t_{2111;11}-t_{2111;21})
\end{align}
so that we have $w_{12} \circ w_{01} = w_{02}$ and $w_{10}\circ w_{21} = w_{20}$, and we have defined
\begin{align}
t_{abcd; \mu \nu } = \AnnularTubexp{\AnnularTubeNoIndex}{a}{b}{c}{d}{\mu}{\nu}{X}{}
\end{align}

\begin{table}
\resizebox{\textwidth}{!}{
{\tabulinesep=1.2mm
\begin{tabu}{ c c | c c c c c c c c c }
particle & spin &$t_x$&$v_x$&$X_{11}$ & $X_{12}$ &$X_{21}$ & $X_{22}$ & $v_x l_y$ & $t_x h_y$ \\ \hline
$2$ & $
       1       
$ &
$
\frac{1}{2d}
$
 &
$
\frac{1}{2d}
$
 &
$ e^{\frac{-9i\pi}{12}}
\frac{\sqrt{3d-8}}{4}
$
 &
$ e^{\frac{-9i\pi}{12}}
\frac{\sqrt{3d-8}}{4}
$
 &
$ e^{\frac{9i\pi}{12}}
\frac{\sqrt{3d-8}}{4}
$
 &
$ e^{\frac{-3i\pi}{12}}
\frac{\sqrt{3d-8}}{4}
$
 &
$ e^{\frac{6i\pi}{12}}
\frac{1}{2d}
$
 &
$ e^{\frac{-6i\pi}{12}}
\frac{1}{2d}
$
 &
\\
$3$ & $
       1       

$ &
$
\frac{1}{4 \sqrt{3}}
$
 &
$
\frac{1}{4 \sqrt{3}}
$
 &
$ e^{\frac{3i\pi}{12}}
\frac{1}{4 \sqrt{d} }
$
 &
$ e^{\frac{9i\pi}{12}}
\beta
$
 &
$ e^{\frac{-9i\pi}{12}}
\beta
$
 &
$ e^{\frac{-3i\pi}{12}}
\frac{1}{4 \sqrt{d} }
$
 &
$ e^{\frac{-6i\pi}{12}}
\frac{1}{4 \sqrt{3}}
$
 &
$ e^{\frac{6i\pi}{12}}
\frac{1}{4 \sqrt{3}}
$
 &
\\
$5$ & $
       1/2     

$ &
$ -
\frac{1}{4 \sqrt{3}}
$
 &
$
\frac{1}{4 \sqrt{3}}
$
 &
$ e^{\frac{-3i\pi}{12}}
\beta
$
 &
$ e^{\frac{-9i\pi}{12}}
\frac{1}{4 \sqrt{d} }
$
 &
$ e^{\frac{-3i\pi}{12}}
\frac{1}{4 \sqrt{d} }
$
 &
$ e^{\frac{-9i\pi}{12}}
\beta
$
 &
$ e^{\frac{-6i\pi}{12}}
\frac{1}{4 \sqrt{3}}
$
 &
$ e^{\frac{-6i\pi}{12}}
\frac{1}{4 \sqrt{3}}
$
 &
\\
$6$ & $
      -5/12    

$ &
$ e^{\frac{-10i\pi}{12}}
\frac{1}{2+d^2}
$
 &
$
\frac{1}{2+d^2}
$
 &
$ e^{\frac{-2i\pi}{12}}
\gamma
$
 &
$ e^{\frac{10i\pi}{12}}
\alpha
$
 &
$ e^{\frac{4i\pi}{12}}
\alpha
$
 &
$ e^{\frac{4i\pi}{12}}
\gamma
$
 &
$ e^{\frac{6i\pi}{12}}
\frac{1}{2+d^2}
$
 &
$ e^{\frac{8i\pi}{12}}
\frac{1}{2+d^2}
$
 &
\\
$7$ & $
       1/4     

$ &
$ e^{\frac{6i\pi}{12}}
\frac{1}{2+d^2}
$
 &
$
\frac{1}{2+d^2}
$
 &
$ e^{\frac{6i\pi}{12}}
\frac{1}{2 \sqrt{6d}}
$
 &
$ e^{\frac{6i\pi}{12}}
\frac{1}{2 \sqrt{6d}}
$
 &
$
\frac{1}{2 \sqrt{6d}}
$
 &
$ -
\frac{1}{2 \sqrt{6d}}
$
 &
$ e^{\frac{6i\pi}{12}}
\frac{1}{2+d^2}
$
 &
$
\frac{1}{2+d^2}
$
 &
\\
$8$ & $
      -5/12    

$ &
$ e^{\frac{-10i\pi}{12}}
\frac{1}{2+d^2}
$
 &
$
\frac{1}{2+d^2}
$
 &
$ e^{\frac{10i\pi}{12}}
\alpha
$
 &
$ e^{\frac{-2i\pi}{12}}
\gamma
$
 &
$ e^{\frac{-8i\pi}{12}}
\gamma
$
 &
$ e^{\frac{-8i\pi}{12}}
\alpha
$
 &
$ e^{\frac{6i\pi}{12}}
\frac{1}{2+d^2}
$
 &
$ e^{\frac{8i\pi}{12}}
\frac{1}{2+d^2}
$
 &
\\
$9$ & $
       1/3     

$ &
$ e^{\frac{8i\pi}{12}}
\frac{1}{2+d^2}
$
 &
$
\frac{1}{2+d^2}
$
 &
$
$
 &
$ e^{\frac{1i\pi}{12}}
\frac{1}{2 \sqrt{3d} }
$
 &
$ e^{\frac{7i\pi}{12}}
\frac{1}{2 \sqrt{3d} }
$
 &
$
$
 &
$ e^{\frac{-6i\pi}{12}}
\frac{1}{2+d^2}
$
 &
$ e^{\frac{-10i\pi}{12}}
\frac{1}{2+d^2}
$
 &
\\
$10$ & $
      -1/6     

$ &
$ e^{\frac{-4i\pi}{12}}
\frac{1}{2+d^2}
$
 &
$
\frac{1}{2+d^2}
$
 &
$ e^{\frac{-11i\pi}{12}}
\frac{1}{2 \sqrt{3d} }
$
 &
$
$
 &
$
$
 &
$ e^{\frac{7i\pi}{12}}
\frac{1}{2 \sqrt{3d} }
$
 &
$ e^{\frac{-6i\pi}{12}}
\frac{1}{2+d^2}
$
 &
$ e^{\frac{2i\pi}{12}}
\frac{1}{2+d^2}
$
 &
\end{tabu}
}
}
\caption{Charge $1$ idempotents. We have used the notation $t_x$ = tube with an $x$ line wrapping around both cycles, $v_x$ = tube with a longitudinal $x$ loop, $X_{ij}$ = tube with all strings labeled by $x$ with index $i$ at the bottom fusion vertex and index $j$ at top fusion vertex, $v_x l_y$ = tube with longitudinal $x$ loop and meridional $y$ loop, and $t_x h_y$ = tube with an $x$ line wrapping both cycles and a $y$ line in the center. }
\end{table}

%%%%%%%%%%%%%%%%%%%%%%%%%%%%%%%%%%%%%%%%%%%%%%%%%%%%%%%%%%%%%%%%%%%%%%%%%%%%%%%%

