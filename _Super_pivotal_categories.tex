%auto-ignore
%      this ensures the arxiv doesn't try to start TeXing here.
%!TEX root = super_lattice_models_draft.tex
%      prev line helps TeXShop do the right thing




%%%%%%%%%%%%%%%%%%%%%%
\section{Super pivotal categories}  \label{def_sect}
%%%%%%%%%%%%%%%%%%%%%%

In this section we give a more formal (though not completely formal) definition of super pivotal categories.
Since the usual bosonic case is well covered in the literature, we will concentrate on
the differences between the bosonic case and the fermionic/super case.
We will also fix some notation and conventions used elsewhere in the paper.


There are various ways of axiomatizing string nets, including Kuperberg spiders \cite{kup_spider}, 
planar algebras \cite{jones_pa},
disk-like 2-categories \cite{blob_paper}, and pivotal tensor categories \cite{Joyal1991}.

The first three are better suited to our applications, but the last one is likely the most familiar to a majority or our readers,
so we will describe a fermionic/super version of pivotal tensor categories.

So far as we know, the earliest definition of a super pivotal tensor category was given in \cite{blob_paper}.
In the higher category definition given in that paper, one of the parameters was the type of balls used to 
specify morphism spaces.
If we take those balls to be 2-dimensional and equipped with spin structures, then one has a definition of a super pivotal 2-category.
We can then take a super pivotal tensor category to be a super pivotal 2-category with only one 0-morphism.
The more traditional-style definition given below is reverse-engineered to be
equivalent with the definition already contained in \cite{blob_paper}.

\medskip

Recall from 
\cite{Joyal1991}
that the data of a pivotal category includes:
\begin{itemize}
\item A set of objects $\spc^1$.
\item A set of morphisms $\spc^2$.
\item A binary operation $\otimes$ (horizontal composition) on objects and morphisms.
\item A binary operation $\circ$ on morphisms.
\item A pivotal structure $*$ defined on both objects and morphisms.
\end{itemize}


The definition of a super pivotal category differs from the usual bosonic case in the following ways:

\begin{enumerate}
	\item The space of morphisms between two objects has the structure of 
	a super vector space.
	Morphisms also satisfy the {\em super interchange law} \cite{brundan2016}:
	\begin{align}
	(f_1\tp f_2) \circ (g_1\tp g_2) = (-1)^{|f_2| |g_1|} (f_1\circ g_1)\tp (f_2\circ g_2)
	\end{align}
	where $|f|$ is the parity of the morphism $f$. 
	\item There are two distinct types of simple object, ``m-type" and ``q-type".
	m-type simple objects have trivial endomorphism algebras $\cc$, as is the case in bosonic theories.  
	q-type simple objects have endomorphism algebras isomorphic to $\cliff_1$, 
	and so their endomorphism algebras contain odd elements in addition to scalars. (See Section \ref{def_sob_ss}.)
	\item In order to keep track of Koszul signs arising from exchanging fermions, 
	we must keep track of a sign-ordering of individual fusion spaces (See \ref{koszul_signs}).
	\item Fusion spaces $V^{abc}$, $V^{ab}_c$, $V_{abcd}$, etc.\ are not merely supervector spaces; they come equipped with an action of the
	endomorphism algebras of the objects being fused. 
	For example, $V^{abc}$ possess an action of $\End(a)\tp \End(b)\tp \End(c)$.
	(See \ref{fusion_spaces}.)
	\item When combining basic 3-valent fusion spaces $V^{ab}_c$ to form fusion spaces of higher valence, we must take
	tensor products over the endomorphism algebras of the simple objects which connect two fusion spaces. 
	For example, we form the fusion space $V^{ab}_{cd}$ as $V^{ab}_{cd} \cong \bigoplus_e V^{ab}_e \tp_{\End(e)} V^e_{cd}$.
	If $e$ is m-type, this is just the usual tensor product over $\cc$, as in the bosonic case.
	But if $e$ is q-type, then we must take a non-trivial tensor product over $\cliff_1$.
	(See \ref{modified_tensor_product}.)
	\item The square of the pivotal anti-automorphism is 
	the fermion parity functor $(-1)^F$, rather than the identity functor (see \ref{pivotal_structure}). 
	If $*$ is the pivotal anti-automorphism, then 
	\be f^{**} = (-1)^{|f|}f.\ee
	In order to keep track of minus signs that result from rotating fermions by $2\pi$, we must keep track of a spin-framing at each fusion space. 
	\item The coherence equations for the basic data of the theory (e.g.\ the pentagon relations) are modified to incorporate  
	 Koszul signs resulting from reordering various fusion spaces. 
	 They are also modified to incorporate the tensor products over endomorphism algebras mentioned above. (see \ref{Fsymbols})
	\item In order to define an inner product, we need to equip the manifold on which our string-nets are 
	defined with a $pin_\pm$ structure, which is discussed in Appendix \ref{fermion_line_bundle}.
\end{enumerate}





%%%%%%%%%%%%%%%
\subsection{Simple objects}  \label{def_sob_ss}
%%%%%%%%%%%%%%%

We will assume that our category $\spc$ is {\it idempotent complete} -- 
every idempotent is the identity morphism of an associated object.
We also assume that $\spc$ is {\it additive} -- we can take direct sums of objects.

An object $a$ of $\spc$ is called {\it simple} if any homogeneous non-zero endomorphism of $a$ is an isomorphism.
Equivalently, $a$ is simple if it has no quotient objects.
(We also stipulate that the zero object is not a simple object.)

In the usual bosonic, non-super case, the only possible endomorphism algebra for a simple object
is the trivial algebra $\cc$ (scalars).
In the fermionic/super case, there is a second possibility: the complex Clifford algebra $\cliff_1$, 
which is the only nontrivial $\zt$-graded division algebra other than $\cc$.
$\cliff_1$ is generated over $\cc$ by the identity (which is even) and an odd element $f$ such that $f^2 = \lambda \cdot \id$
(or $f^2 = \lambda$ for short) for some non-zero complex number $\lambda$. 
Note that by rescaling the odd generator $f$ we can make $\lambda$ in the definition of $\cliff_1$ any nonzero complex number. 

It follows that simple objects in a super pivotal category fall into two classes, according to whether their endomorphism algebras are $\cc$ or $\cliff_1$. 
These are the m-type and q-type objects discussed earlier. 
A simple object is {m-type} if its endomorphism algebra
is $\cc$, and {q-type} if its endomorphism algebra is $\cliff_1$:
\begin{align}
\vcenter{\xymatrix @!0 @M=1mm @C=25mm{
& \text{End}(x) = \mathbb{C} \ar@{<->}[rr] &   &\text{$x$ is a simple m-type object}&  \\
&\text{End}(x) = \mathbb{C} \ell_1 \ar@{<->}[rr]  &  &\text{$x$ is a simple q-type object}&
	}}
\end{align}
This terminology comes from the notation of \cite{jozefiak1988}, which classifies simple super algebras over $\cc$ as either
$M(p|q) = \End(\cc^{p|q})$ or $Q(n)$ (see appendix \ref{superstuff}).
Note that we are using ``simple" here in two different (and well-established) senses: 
any $M(p|q)$ or $Q(n)$ is a simple super algebra
(because it has no non-trivial ideals), but the endomorphism algebra of a simple object must be either
$M(1|0) \cong M(0|1) \cong \cc$ or $Q(1) \cong \cliff_1$,
because all of the larger $M(p|q)$ or $Q(n)$ contain non-invertible elements.

The existence of q-type particles is responsible for much of the novel physics present after performing fermion condensation. 
q-type objects were also discussed in \cite{usher2016,gaiotto2016}, where they were referred to as ``Majorana objects''. 
(We prefer the m-type/q-type terminology, since it makes clearer the relationship to the Morita classification of 
simple super algebras.)



%%%%%%%%%%%%%%%%%%%%%%%%%%%
\subsection{Fusion spaces} \label{fusion_spaces}
%%%%%%%%%%%%%%%%%%%%%%%%%%%

Arbitrary morphism spaces in a super pivotal category can be built out of basic fusion spaces
$V^{ab}_c = \mor(c\to a\tp b)$,
where $a$, $b$ and $c$ are simple objects (equivalently, minimal idempotents).
This is a super vector space of dimension
$N^{ab}_c = \dim V^{ab}_c = p|q$, where $p$ is the dimension of the even part
of $V^{ab}_c$ and $q$ is the dimension of the odd part of $V^{ab}_c$.

Alternatively, we can treat $a$, $b$ and $c$ more symmetrically and define
$V^{abc} = \mor(\unit\to a\tp b\tp c)$, a super vector space of dimension $N^{abc}$.
In most of this paper we use $V^{ab}_c$, but in Sections \ref{Super_pivotal_Hamiltonian} and \ref{state_sums} we find it more convenient
to use $V^{abc}$.
Elements of the morphism spaces $V^{abc}$ and $V^{ab}_c$ are depicted by
\begin{align}
\PitchFork{a}{b}{c}{\mu} \quad \quad \quad \text{and} \quad \quad \quad \Nonpitchfork{a}{b}{c}{\mu}.
\end{align}
We call the fusion spaces $V^{abc}$ ``pitchforks" because of their graphical depiction.
Of course we have $V^{ab}_c \cong V^{abc^*}$ (see \ref{pivotal_ss} below).

More generally, we define $V^{ab}_{cd} = \mor(c\tp d\to a\tp b)$, $V^{ab}_{cde} = \mor(c\tp d\tp e\to a\tp b)$,
$V_{abcd} = \mor(a\tp b\tp c\tp d\to \unit)$, and so on.
In general, we do not require that the objects $a$, $b$ etc. be simple.

\medskip

It is very important to note that $V^{ab}_c$ is not merely a super vector space -- it also comes equipped with an action
of (i.e.\ module structure for) the endomorphism algebras of $a$, $b$ and $c$, and hence admits an action of $\End(a)\tp \End(b)\tp \End(c)$. 
It is impossible to construct the full super pivotal category without knowing this module structure (see \ref{modified_tensor_product} below), 
so the module
structure is part of the input data.
Note that the module structure implies that $N^{ab}_c = n|n$ if any of $a$, $b$ or $c$ is q-type.
This is because any representation of $\cliff_1$ has equal even and odd dimensions.
Acting with the odd (and invertible) element of $\cliff_1$ gives an isomorphism between the even and odd parts of $V^{ab}_c$, 
and hence they must have the same dimension.

The explicit matrix elements of these isomorphisms can be defined in the following way. 
Let $\psi \in V^{ab}_c$ and suppose that $b$ is a q-type simple object, and let $\Gamma_b \in \End(b)$ be an odd endomorphism.
Then 
\begin{align}
\Gamma_b \left( \Vertexa \right) = \sum_{\eta \in V^{ab}_c} \left[ \Gamma_b \right]_{\psi \eta }\Vertexe, 
\end{align}
where the matrix elements are obtained from 
\begin{align}
\label{Gamma_y_def}
\Gamma_b \left( \Vertexa \right) \; = \; \Vertexb \; =\; \sum_{\eta \in V^{ab}_c} \; \Vertexe \; \left( 
{\Vertexc}\middle/ {\Vertexd} \right), 
\end{align}
where the $\eta^*$ provide a complete orthogonal basis (with respect to the pairing \eqref{reflection_pairing_defn}) for $V^{b^*a^*}_{c^*}$.
If either $a$ or $c$ are q-type, then $\Gamma_a$ and $\Gamma_c$ matrices can be defined in a similar way. 




\medskip

If at least one of $a,b,c$ is q-type, we can simplify the description of $V^{ab}_c$ slightly, 
which we have done when working out the examples considered earlier.
Suppose $c$ is q-type, and that $\{|\psi_i\rangle\} \in [V^{ab}_c]^0, i =1,\dots, r$ are the even basis vectors in $V^{ab}_c$. 
Then we can define a complete set of odd basis vectors $\{ |\eta_i\rangle \}$ for $[V^{ab}_c]^1$ by $|\eta_i\rangle = f|\psi_i\rangle$, where $f$ is the odd element of $\End(c) \cong \cc^{1|1}$.
When we write $|\eta_i\rangle$ graphically, we will write it as $f |\psi \rangle$, which allows us to  ``shift the oddness out of the vertex onto the edge'' by transferring the fermion residing on the fusion space to the q-type particle $c$. 
Graphically, this means that we are allowed to ``displace'' dots from trivalent vertices onto q-type worldlines:
\begin{align} \label{dot_displacement} 
\NonPitchforkDot{a}{b}{c}{\eta_i} \;\; = \;\; \NonPitchforkDotDisplaced{a}{b}{c}{\psi_i}
\end{align}
where the picture on the left is $|\eta_i\rangle$ and the one on the right is $f|\psi_i\rangle$, and $c$ is assumed to be q-type. 
Although this is not a deep fact, it proves to be helpful when doing graphical manipulations, 
and operators implementing transformations like \eqref{dot_displacement} will be crucial for writing down the lattice 
Hamiltonian which realizes the
super pivotal version 
of the Levin-Wen Hamiltonian. 

\medskip

\dave{Modified this to amend comments as per discussion. 
EL and KW may want to look at it.}
\ethan{made some tiny changes; looks good to me}
\dave{Looks good to me.}
Lastly, we will define a non-degenerate bilinear pairing between vectors in the vector space assigned to a disk with $n$ marked points. 
We will focus on fusion spaces of the form $V^{x_1\dots x_n}$, but the construction
for different types of fusion spaces is analogous. 
The pairing is defined by 
\begin{align} \label{reflection_pairing_defn}	
\mcb:\;  &V^{x_n^* \cdots x_2^* x_1^*} \tp V^{x_1 x_2 \cdots x_n}  \ra \cc \\
\nonumber &\Bilineara \tp \Bilinearb \mapsto \Bilinearc
\end{align} 
where $\mu\in V^{x_n^* \cdots x_2^* x_1^*}$ and $\nu  \in V^{x_1 x_2 \cdots x_n}$. 
We are working with the convention that the Koszul ordering of the tensor product increases 
in a left-to-right fashion (indicated by the numbers $1,2$ in the bottom right of \eqref{reflection_pairing_defn}); we will elaborate on this convention in Section \ref{koszul_signs}. 
This bilinear pairing is just the evaluation map, and is $\cc$-linear in both its arguments. 
It is non-degenerate, meaning that if $\nu_j$ is a complete basis for the fusion space $V^{x_1 x_2 \cdots x_n}$ and $\mu_i$ is a complete basis for the dual fusion space $V^{x_n^* \cdots x_2^* x_1^*} $, then the matrix $\mcb_{ij} = \mcb(\mu_i \tp \nu_j)$ is invertible. 
Hence we can define a set of vectors $\mu_j^* = \sum_i  (\mcb^{-1} )_{ji}\nu_i $ so that
$\mcb( \mu_j^* \tp \mu_i)=\delta_{ij}$.
Alternatively, we will can choose the normalization convention
\begin{align}  \label{b_pairing_defn}
\mcb( \mu_j^* \tp \mu_i )  = \sqrt{d_a d_b d_c} \, \delta_{ij},
\end{align} 
with $\mu_i \in V^{abc}$ and $\mu_j^* \in V^{c^* b^* a^*}$.

%%%%%%%%%%%%%%%%%%%%%%%%%%
\subsection{Pivotal structure}  \label{pivotal_ss}   \label{pivotal_structure} 
%%%%%%%%%%%%%%%%%%%%%%%%%%

The pivotal structure assigns to each object $a$ a dual object $a^*$.
It also provides linear isomorphisms
\be
	P_L : \mor(a \to b\tp c) \to \mor(b^*\tp a \to c)
\ee
and
\be
	P_R : \mor(a\tp b \to c) \to \mor(a \to c\tp b^*)
\ee
for any objects $a$, $b$ and $c$.
These isomorphisms are required to be functorial with respect to $a$ and $c$.
In addition, they are required to be twisted-functorial with respect to $b/b^*$, where we use the 
$*$ functor (defined below) to relate morphisms with domain $b$ to morphisms with range $b^*$.

For objects $a$, we require that $a^{**} = a$ on the nose (strict pivotal).
For morphisms $f : a\to b$, we define $f^* : b^* \to a^*$ by $f^* = P_L(P_R(f))$, diagrammatically by,
\begin{align}
\fstar \; = \; \fPLPR
\end{align}
(We have implicitly added and then removed some tensor units (trivial objects) here appearing 
in fusion spaces like $V^{aa^*}_\unit$.)
We think of $f^*$ as a $+\pi$ rotation of the morphism $f$.
We have
\be
	(f\cdot g)^* = (-1)^{|f||g|} g^* \cdot f^* .
\ee
In other words, $*$ is a contravariant functor if one takes Koszul signs into account.

For (strict) pivotal bosonic categories, one requires that $**$ is the identity functor, but
in the fermionic case one requires that $**$ is the spin-flip functor $(-1)^F$.
More specifically, we require
\be
\label{spin_flip_functor}
	f^{**} = (-1)^{|f|} f ,
\ee
since $f^{**}$ is a $2\pi$ rotation of $f$.

The part of the pivotal structure most used in calculations is the $+2\pi/3$ rotation on the 
basic trivalent fusion spaces.
We define the ``pivot" $P^{ab}_c : V^{ab}_c \to V^{bc^*}_{a^*}$
as $P_R\circ P_L$.
In terms of diagrams and matrices, this looks like
\begin{align}
P^{ab}_c \left( \;  \PivotYa \; \right) = \PivotYb
\end{align} 

\kw{Some edits below; should be uncontroversial, but still worth reviewing}
\dave{Looks good to me. Is there a reason why the eigenvalue of the odd part of $V^{q}_{q^*}$ isn't a FS indicator?}

The Frobenius-Schur indicator $\kappa_a$ can be computed in terms of pivot maps.
If $a\cong a^*$, then $\kappa_a$ is the eigenvalue of the composite map
\be
	V^{a}_{a^*} \stackrel{U_r}\longrightarrow V^{a1}_{a^*} \stackrel{P}\longrightarrow V^{1a}_{a^*} \stackrel{U_l^{-1}}\longrightarrow V^{a}_{a^*} ,
\ee
where $U_r$ and $U_l^{-1}$ are given by post-composition with the canonical isomorphisms (a.k.a.\ unitors)
$a \stackrel{\sim}{\to} a\tp \unit$ and $a \stackrel{\sim}{\to} \unit\tp a$.
(If $a$ is q-type then we take the eigenvalue for the even part of $V^{a}_{a^*}$.)

We also note that the modular $S$ matrix gives the square of the Frobenius-Schur indicator.
If $a \cong a^*$ we have $\kappa_a^2 = (S^2)_{aa}$.
For a bosonic theory the Frobenius-Schur indicators are $\pm1$, and this provides no new information. 
But in fermionic theories the oddly self-dual simple objects have Frobenius-Schur indicators of $\pm i$,
and this is detected by the diagonal entries of $S^2$.
The $m_4$ particle in the $\halfesix/y$ theory is an example of this; see \eqref{hE6_S_NN}.

We can similarly define $P^{abc} : V^{abc}\to V^{bca}$. 
In terms of diagrams, 
\begin{align}
\label{Pitchfork_pivot}
P^{abc} \left(  \Pitchforkabc \right ) \;=\;  \Pitchforkabcrot.
\end{align}
We will usually write simply $P$, since the $a$, $b$ and $c$ are typically clear from context.

Since $P^3$ acts as a $2\pi$ rotation, we have $P^3 = (-1)^F$. Diagrammatically, when acting on $V^{abc}$ this is written 
\begin{align}
P^3 \left(  \Pitchforkabc \right )\; = \; \PivotThreeTimes \; = \;(-1)^F \Pitchforkabc.
\end{align}




%%%%%%%%%%%%%%%%%%%%%%%%%%
\subsection{Fusion rules and fusion spaces} \label{fusion_rules_and_fusion_spaces}
%%%%%%%%%%%%%%%%%%%%%%%%%%

In this section we elaborate on the differences arising in fermionic theories between fusion spaces and the super vector spaces appearing in the fusion rules. 

We assume that our categories are additively complete, which means that it makes sense to
multiply objects by super vector spaces. 
(This is a categorified version of multiplying vectors by scalars.
Vectors are promoted to objects and scalars are promoted to vector spaces.)
For any collection of super vector spaces $\{W_a\}$ indexed by 
a finite set of objects $\{a\}$ in our category, 
we therefore have an object of the form
\be 
	\bigoplus_a W_a \cdot a .
\ee

Morphisms between these more general objects are calculated as  
\be  \label{amordef}
	\mor(\bigoplus_a W_a\cdot a \to \bigoplus_b W'_b\cdot b) = \bigoplus_{a,b} \Hom(W_a \to W'_b)\otimes_\cc \mor(a\to b) .
\ee

Because our category is semisimple, there exists a finite collection $\sob(\spc)$ of mutually non-isomorphic simple objects $x$ such
that any object $a$ is isomorphic to one of the form
\be \label{asumx}
	a \cong \bigoplus_{x\in \sob(\spc)} W_x\cdot x .
\ee
If we want the isomorphism to be canonical, we can take $W_x = \mor(x\to a)$ if $x$ is m-type, or $W_x$ to be the even morphisms in $\mor(x\to a)$
if $x$ is q-type.

Combining \eqref{asumx} and \eqref{amordef}, we can compute endomorphisms of objects by
\be
	\End(a) \cong \bigoplus_{x\in \sob(\spc)} \End(W_x) \otimes_\cc \End(x).
\ee

We are now ready to discuss fusion rules. 
For any $a$ and $b$, define the vector spaces $\Delta^{ab}_c$ by
\be
	a \otimes b \cong \bigoplus_{c\in \sob(\spc)} \Delta^{ab}_c \cdot c .
\ee
The $\Delta^{ab}_c$ are the fusion rule coefficients. 

The fusion spaces $V^{ab}_c$ are defined as the vector space of morphisms from $c$ to $a\tp b$:
\be \label{defn_of_V_by_Delta}
	V^{ab}_c = \mor(c \to a\tp b),
\ee
where $a,b,c$ are simple objects. 
Decomposing the tensor product and using the simplicity of $c$, we see that 
\be V^{ab}_c \cong \Delta^{ab}_c \tp \rm{mor}(c \ra c) \cong \Delta^{ab}_c \otimes \End(c).\ee
Thus, the fusion spaces can be larger than the vector spaces appearing in the fusion rules 
(in contrast to bosonic theories, where the fusion spaces and fusion rule coefficients are always equal).
As examples, in the $C_2$ theory studied earlier, we have 
\be \Delta^{q_\sigma q_\sigma}_{m_\psi} \cong \cc^{1|1},\quad\Delta^{q_\sigma m_\unit}_{q_\sigma} \cong \cc,\ee
while 
\be V^{q_\sigma q_\sigma}_{m_\psi} \cong V^{q_\sigma m_\unit}_{q_\sigma} \cong \cc^{1|1}.\ee

$V^{ab}_x$ is cyclically symmetric (up to isomorphism) in $a,b,x$ (if $a$ and $b$ are simple).
Explicitly, this is because 
\be \mor(c \ra a\tp b) \cong \mor(\unit \ra a\tp b \tp c^*) \cong \mor(a^* \ra b\tp c^*),\ee 
which allows us to cyclicly permute the indices of $V$, so long as we take the duals of any objects that move from subscripts to superscripts, and vice versa. 
For example, we have $V^{ab}_c \cong V^{bc^*}_{a^*} \cong V^{c^*a}_{b^*}$. 

On the other hand, $\Delta^{ab}_c$ is {\it not} cyclically symmetric in $a,b,c$, as the $C_2$ theory example shows. 
Additionally, while $V^{ab}_c$ has an action of $\End(a)\otimes\End(b)\otimes\End(c)$ (as mentioned earlier), $\Delta^{ab}_c$
only has an action of $\End(a)\otimes\End(b)$.





%%%%%%%%%%%%%%%%%%%%%%%%%%%%%%%%%%
\subsection{Koszul sign rule and unordered tensor products} \label{koszul_signs}
%%%%%%%%%%%%%%%%%%%%%%%%%%%%%%%%%%

We will treat Koszul signs as in \cite[Section 1.2]{deligne1999}.
This approach doesn't really do away with Koszul signs.
Rather, it pushes them to the background, where they don't need to be mentioned as frequently.
For explicit calculations, they must again be brought to the foreground.

Let $I$ be a finite (and unordered) index set.
For each $i\in I$, let $W_i$ be a super vector space.
We define the unordered tensor product,
\be
	\bigotimes_{i\in I} W_i ,
\ee
as follows.
For each bijection $f: \{1, \ldots,m\} \to I$ (i.e.\ for each ordering of $I$),
we have the ordered tensor product
\be
	T_f = W_{f(1)}\otimes\cdots\otimes W_{f(m)} ,
\ee
generated by elements
\be
	w_{f(1)}\otimes\cdots\otimes w_{f(m)}.
\ee
For any two orderings $f$ and $g$, there is a Koszul isomorphism
\be
	K_{fg} : T_f \to T_g ,
\ee
characterized by\footnote{
We should stress that in this section, 
the left-to-right ordering of tensor factors appearing in equations is tied to their Koszul ordering only, and
is independent of the order in which they appear when written down in diagrams. 
This is in contrast to several other points in the 
paper, where $a\tp b$ translates graphically into placing $a$ horizontally next to $b$ on the page.
} 
\begin{multline}
	K_{fg} : w_{f(1)}\otimes\cdots\otimes w_{f(k)} \otimes w_{f(k+1)} \otimes \cdots \otimes w_{f(m)} \mapsto \\
				(-1)^{|w_{f(k)}||w_{f(k+1)}|} w_{f(1)}\otimes\cdots\otimes w_{f(k+1)} \otimes w_{f(k)} \otimes \cdots \otimes w_{f(m)}
\end{multline}
when $g$ differs from $f$ by a simple transposition at $k$, and
\be
	K_{gh} \circ K_{fg} = K_{fh} .
\ee
An element of the unordered tensor product is then defined as an assignment to each ordering $f$ of an element $t_f\in T_f$
such that
\be
	K_{fg}(t_f) = t_g
\ee
for all orderings $f$ and $g$.
In other words, an element of the unordered tensor product is a collection of elements in all 
possible ordered tensor products which are related by the usual Koszul sign rules.


Note that to specify an element of the unordered tensor product, it suffices to give an element $t_f$ of one
particular ordered tensor product $T_f$.
All of the other $t_g$ are uniquely determined by $t_f$.

When writing equations involving particular ordered tensor products of fusion spaces, we will adopt the 
convention that the Koszul ordering is left-to-right on the page, unless explicitly 
indicated otherwise. If we want to indicate an ordering which departs from this left-to-right convention, 
we will indicate the ordering explicitly with numerical subscripts, 
e.g. $V^{ab}_{c,1} \tp V^{cd}_{e,2}$.
This explicit notation is often better suited to our diagrammatic calculus, where 
we frequently label the Koszul ordering of fermion dots in a way that is not tied to the 
left-to-right order in which we write down tensor products (this was done throughout Sections
\ref{C2_condense_sect} and \ref{C2_quasiparticles}, for example). 

When drawing diagrams with a particular ordering in mind, we always indicate the ordering explicitly by numbers near each fusion space (i.e.\ near
each vertex in the string net).
Another possible convention would be to use the ordering corresponding to (say) bottom-to-top on the page,
but this creates opportunities for error when changing diagrams by isotopies,
and is not really workable for diagrams drawn on higher genus surfaces.

A map between unordered tensor products
\be
	\bigotimes_i W_i \; \to \; \bigotimes_j V_j
\ee
is defined to be a collection of maps between
all possible pairs of ordered tensor products.
We will call such a collection an ``unordered map".
These maps are required to commute with the Koszul isomorphisms on either side.
To specify such a map, it suffices to give a single map between one particular pair of ordered tensor products.
All other maps in the collection are uniquely determined by this choice and the commutativity requirement.
This map will be called an ``ordered representative" of the unordered map.
See \ref{coherence_ss} for a further discussion of the distinction between unordered maps and
their ordered representatives.





%%%%%%%%%%%%%%%%%%%%%%%%%%%%%%%%%%%%%%%
\subsection{Modified tensor product} \label{modified_tensor_product}
%%%%%%%%%%%%%%%%%%%%%%%%%%%%%%%%%%%%%%%



Let $\sob(\spc)$ be a complete collection of simple objects (minimal idempotents) in some 
input super fusion category $\spc$, one from each equivalence class.
(In this subsection, as in most of the paper, we are assuming that our category $\spc$ is semisimple with finitely many equivalence
classes of simple objects.)
For arbitrary objects $x$ and $y$, we have
\be \label{sobdecomp}
	\mor(x\to y) \: \cong \: \bigoplus_{a\in\sob(\spc)} \mor(x\to a) \tp_{\End(a)} \mor(a\to y) .
\ee
Recall that the relative tensor product $\tp_{\End(a)}$ on the RHS above is defined as the usual tensor product over scalars, modulo elements
of the form $\alpha \cdot f \tp \beta - \alpha\tp f \cdot \beta$ with $\alpha \in \mor(x \to a)$, $\beta \in \mor(a \to y)$ and $f\in \End(a)$.
Clearly such elements are in the kernel of the composition map $\mor(x\to a) \tp \mor(a\to y) \to \mor(x\to y)$.
Our semisimplicity assumption implies that the composition map (summing over all $a\in \sob(\spc)$) is surjective and that such
elements generate all of the kernel.

In terms of diagrams, the relative tensor product is responsible for allowing fermionic dots to move across edges labeled by q-type simple objects.
Loosely, taking a tensor product over $\End(a)$ when $a$ is q-type allows us to identify diagrams that 
differ only by the position of a fermionic dot on an $a$ strand. 


It follows (though not quite directly) from \eqref{sobdecomp} that we have isomorphisms
\be \label{pfi1}
	V^{abc}_d \cong \bigoplus_{x\in \sob(\spc)} V^{ab}_x \tp_{\End(x)} V^{xc}_d
\ee
and also
\be \label{pfi2}
	V^{abc}_d \cong \bigoplus_{y\in \sob(\spc)} V^{ay}_d \tp_{\End(y)} V^{bc}_y .
\ee
Diagrammatically these read,
\begin{align} 
\bigoplus_{x}  \Fusionspaceb \cong \Fusionspacea \cong  \bigoplus_{y}  \Fusionspacec
\end{align} 
where the unlabeled trivalent vertices denote the fusion spaces $V^{ab}_x$, $V^{xc}_d$, $V^{ay}_d$, $V^{bc}_y$, 
and the unlabeled tetravalent vertex in the middle diagram denotes the fusion space $V^{abc}_d$.
The tensor product over endomorphisms is implicit in the diagram.
Similarly, 
\be
	V^{abcd} \cong \bigoplus_{x\in \sob(\spc)} V^{ab}_x \tp_{\End(x)} V^{xcd} ,
\ee
and using the isomorphism $P_R: V^{ab}_x \to V^{abx^*}$ this becomes
\be  \label{pfpfi1}
	V^{abcd} \cong \bigoplus_{x\in \sob(\spc)} V^{abx} \tp_{\End(x)} V^{x^*cd} .
\ee
(We are implicitly using the $*$ functor to convert an $\End(x)$ action into an $\End(x^*)$ action.)
Alternatively,
\be  \label{pfpfi2}
	V^{abcd} \cong \bigoplus_{x\in \sob(\spc)} V^{bc}_x \tp_{\End(x)} V^{axd} \cong \bigoplus_{x\in \sob(\spc)} V^{axd}  \tp_{\End(x)} V^{x^*bc}.
\ee
Diagrammatically we have,
\begin{align}
\vcenter{\xymatrix @!0 @M=4mm @R=14mm @C=20mm{
 \bigoplus_x \TensorProductaprime &\quad \cong& \TensorProducteprime  &\cong\quad&\bigoplus_x \TensorProductcprime   \\
\quad \quad \rotatebox[origin=c]{90}{$\cong$}   &          &                                      &         &\quad\quad\rotatebox[origin=c]{90}{$\cong$}\\
 \bigoplus_x \TensorProductbprime  &          &                                      &         &\bigoplus_x \TensorProductdprime  
	}}
\end{align}





%%%%%%%%%%%%%%%%%%
\subsection{F-symbols} \label{Fsymbols}
%%%%%%%%%%%%%%%%%%

It follows from \ref{pfi1} and \ref{pfi2} that there is an isomorphism
\be  \label{fdef21}
	F^{abc}_d : \bigoplus_{x\in \sob(\spc)} V^{ab}_x \tp_{\End(x)} V^{xc}_d \;\to\; \bigoplus_{y\in \sob(\spc)} V^{ay}_d \tp_{\End(y)} V^{bc}_y .
\ee
For the pitchfork version, we instead use \ref{pfpfi1} and \ref{pfpfi2} to obtain
\be  \label{fdef30}
	F^{abcd} : \bigoplus_{x\in \sob(\spc)} V^{abx^*} \tp_{\End(x)} V^{xcd}   \;\to\;   \bigoplus_{x\in \sob(\spc)} V^{bcx^*} \tp_{\End(x)} V^{axd} .
\ee
The tensor products appearing in the above isomorphisms are unordered tensor products. 
For numerical applications, particular ordered representatives of the tensor products need to be chosen. 
For the ordered $F^{abc}_d$ isomorphism, 
we will adopt the convention
\be \label{Fmove_ordering_conventions}
	F^{abc}_d : \bigoplus_{x\in \sob(\spc)} V^{xc}_{d} \tp_{\End(x)} V^{ab}_{x} \;\to\; \bigoplus_{y\in \sob(\spc)} V^{ay}_{d} \tp_{\End(y)} V^{bc}_{y} ,
\ee
with implicit sign ordering which increases from left to right on the page.
Graphically, and written as a matrix equation, we thus have
\begin{align}
 \label{graphical_Fmove} 
\FusionSpaceLeftOrdered =\sum_y \sum_{\sigma \rho}\left( F^{abc}_d \right)_{(x; \mu \nu)(y;\sigma \rho)}   \FusionSpaceRightOrdered
\end{align}
where the greek indices label particular fusion space basis vectors, with $\mu\in V^{ab}_x,\nu\in V^{xc}_d,\alpha\in V^{bc}_y,$ and $\beta\in V^{ay}_d$, 
and where the first sum is over $y\in \sob(\spc)$. 
We also stick with the left-to-right ordering convention for the $F^{abcd}$ move:
\be  
	F^{abcd} : \bigoplus_{x\in \sob(\spc)} V^{abx} \tp_{\End(x)} V^{x^*cd}   \;\to\;   \bigoplus_{y\in \sob(\spc)} V^{ayd} \tp_{\End(x)} V^{y^*bc} .
\ee
Which written as a matrix equation is
\begin{align}
\label{PitchforkFMove}
\PitchforkFLeftprime = \sum_{y} \sum_{\rho \sigma} \left( F^{abcd} \right)_{(x; \mu \nu)(y; \rho \sigma)} \PitchforkFRightprime .
\end{align}

We will also find the following identity helpful:
\begin{align}
\label{idpitchfork}
\Idabba = \sum_{x} \frac{d_x}{\mcb(\mu^* \tp \mu )} \IdPitchfork, 
\end{align}
where the pairing $\mcb$ is defined in \eqref{reflection_pairing_defn}.




%%%%%%%%%%%%%%%%%%
\subsection{Coherence relations} \label{coherence_ss}
%%%%%%%%%%%%%%%%%%

We will not list all coherence relations here.
Instead we will give a few examples, in order to highlight how the bosonic case
must be changed to take account of Koszul signs and relative tensor products.

\medskip

We will start with the well-known pentagon equation, in the version that uses the basic fusion spaces $V^{ab}_c$.
If we work in terms of {\it unordered maps}, with \eqref{fdef21} interpreted as an unordered map between direct sums of
unordered tensor products, then the fermionic pentagon equation looks just like the bosonic case, namely
that the following diagram commutes:
\begin{align}
\newcommand{\A}{\bigoplus_{p,q}V^{xy}_{p}\tp_p V^{pz}_{q}  \tp_q V^{qw}_{u}}
\newcommand{\AB}{\ar[rrrru]^{F^{pzw}_u} \ar[rrd]^{F^{xyz}_q}}
\newcommand{\B}{\bigoplus_{p,t} V^{xy}_{p} \tp_p V^{pt}_{u} \tp_t V^{zw}_{t}}
\newcommand{\BC}{\ar[rrrrd]^{F^{xyt}_u}}
\newcommand{\D}{\bigoplus_{r,q}V^{xr}_{q}\tp_r V^{yz}_{r}  \tp_q V^{qw}_{u}}
\newcommand{\DE}{\ar[rrrr]^{F^{xrw}_u} }
\newcommand{\E}{\bigoplus_{s,r}V^{xs}_{u}\tp_s V^{yz}_{r}  \tp_r V^{rw}_{s}}
\newcommand{\EF}{\ar[rru]^{F^{yzw}_s}} 
\newcommand{\F}{\bigoplus_{t,s} V^{xs}_{u} \tp_s V^{yt}_{s} \tp_t V^{zw}_{t}}
\vcenter{\xymatrix @!0 @M=5mm @R=28mm @C=16mm{
&&&&\B\BC&&&&\\
\A\AB&&&&&&&&\F\\
&&\D\DE&&&&\E\EF&&
	}} 
		\label{bosonPentagon}
\end{align}
where all the sums are over a representative set of simple objects and we have used the notation $\tp_x \equiv \tp_{\text{End}(x)}$.

However, if we peer under the hood and look at {\it ordered representatives} (as we would need to do if, for example,
we were checking the pentagon equation on a computer), then we
see that a Koszul sign appears:



\begin{align}
\newcommand{\A}{\bigoplus_{p,q}\Penta}
\newcommand{\AB}{\ar[rru]^{F^{pzw}_u} \ar[rrd]^{F^{xyz}_q}}
\newcommand{\B}{\bigoplus_{p,t} \Pentb}
\newcommand{\BC}{\ar[rrr]^{K_{23}}}
\newcommand{\C}{\bigoplus_{p,t} \Pentc}
\newcommand{\CF}{\ar[rrd]^{F^{xyt}_u}}
\newcommand{\D}{\bigoplus_{r,q}\Pentf}
\newcommand{\DE}{\ar[rrr]^{F^{xrw}_u} }
\newcommand{\E}{\bigoplus_{s,r}\Pente}
\newcommand{\EF}{\ar[rru]^{F^{yzw}_s}} 
\newcommand{\F}{\bigoplus_{t,s} \Pentd}
\vcenter{\xymatrix @!0 @M=2mm @R=28mm @C=19mm{
&&\B \BC &&&\C \CF &&\\
\A \AB &&&&&&&\F \\
&&\D \DE&&&\E \EF&&
	}} 
	\label{endoPentagonDiagram}
\end{align}
or equivalently, 
\begin{align}
\newcommand{\A}{\bigoplus_{p,q}V^{qw}_{u}\tp_q V^{pz}_{q}  \tp_p V^{xy}_{p}}
\newcommand{\AB}{\ar[rru]^{F^{pzw}_u} \ar[rrd]^{F^{xyz}_q}}
\newcommand{\B}{\bigoplus_{p,t} (V^{pt}_{u} \tp_t V^{zw}_{t}) \tp_p V^{xy}_{p}}
\newcommand{\BC}{\ar[rrr]^{K_{23}}}
\newcommand{\C}{\bigoplus_{p,t} (V^{pt}_{u} \tp_p V^{xy}_{p}) \tp_t V^{zw}_{t}}
\newcommand{\CF}{\ar[rrd]^{F^{xyt}_u}}
\newcommand{\D}{\bigoplus_{r,q}V^{qw}_{u}\tp_q V^{xr}_{q}  \tp_r V^{yz}_{r}}
\newcommand{\DE}{\ar[rrr]^{F^{xrw}_u} }
\newcommand{\E}{\bigoplus_{s,r}V^{xs}_{u}\tp_s V^{rw}_{s}  \tp_r V^{yz}_{r}}
\newcommand{\EF}{\ar[rru]^{F^{yzw}_s}} 
\newcommand{\F}{\bigoplus_{t,s} V^{xs}_{u} \tp_s V^{yt}_{s} \tp_t V^{zw}_{t}}
\vcenter{\xymatrix @!0 @M=2mm @R=22mm @C=19mm{
&&\B \BC &&&\C \CF &&\\
\A \AB &&&&&&&\F \\
&&\D \DE&&&\E \EF&&
	}} 
	\label{endoPentagon}
\end{align}
Here, $K_{23}$ denotes the Koszul isomorphism associated to transposing the second and third tensor factors.
Again, we are using the implicit left-to-right Koszul ordering of each tensor product.
In terms of the matrix elements of the $F$-symbols, this reads 
\be  
\begin{aligned}
 \sum_{r\in \sob(\spc)}
 \sum_{\sigma\in V^{xr}_q}
 \sum_{\omega \in V^{yz}_r}
 \sum_{\eta\in V^{rw}_s}
 & 
 [F^{xyz}_q]_{(p;\mu\nu)(r;\omega\sigma)}
 [F^{xrw}_u]_{(q;\sigma\lambda)(s;\eta\gamma)}
 [F^{yzw}_s]_{(r;\omega \eta)(t;\alpha\delta)} \\ 
 & = \sum_{\beta \in V^{pt}_u}
 [F^{pzw}_u]_{(q;\nu\lambda)(t;\alpha\beta)}
 (-1)^{|\mu||\alpha|}
 [F^{xyt}_u]_{(p;\mu\beta)(s;\delta \gamma)}, 
\end{aligned} 
\ee
where $\mu\in V^{xy}_p,\nu\in V^{pz}_q,\lambda\in V^{qw}_u,\gamma\in V^{xs}_u,\delta\in V^{yt}_s,$ and $\alpha\in V^{zw}_t$. 
The Koszul sign $K_{23}$ appearing in \eqref{endoPentagon} appears in the above formula as $(-1)^{|\mu||\alpha|}$. 



Other coherence relations are modified to take into account Koszul 
signs.
For example, requiring consistency between $F$-moves and the pivot means that the following 
diagram must commute:
\begin{align}
\newcommand{\A}{\bigoplus_x \PitchforkFLeftprimesmall}
\newcommand{\B}{\bigoplus_x  \PivotCoherenceb}
\newcommand{\C}{\bigoplus_x\;  \PivotCoherencec  }
\newcommand{\D}{\bigoplus_y   \PivotCoherencea }
\newcommand{\E}{\bigoplus_y \PitchforkFRightprimesmall }
\vcenter{\xymatrix @!0 @M=1.5mm @R=25mm @C=30mm{
&\A  \ar[rr]^{F^{abcd}} &&\E \ar[rd]^{(P^{ayd})^{-1}}&\\
\B\ar[ru]^{K_{12}} &&&&\D \ar[lld]^{F^{dabc}}   \\
&&\C \ar[llu]^{P^{dxc} \tp P^{x^*ab} } &&
	}} 
	\label{pivotconsistent}
\end{align} 




%%%%%%%%%%%%%%%%%%
\subsection{Reflection structure} \label{reflection_ss}
%%%%%%%%%%%%%%%%%%

A {\it reflection structure} on $\spc$ is an antilinear anti-automorphism $r$ from $\spc$ to itself which preserves (rather than reverses) 
the the tensor product:
\begin{align}
	a & \; \mapsto \; r(a) \\
	\alpha: a\to b & \; \mapsto \; r(\alpha): r(b) \to r(a) \\
	r(\lambda \alpha) & \;=\;  \bar\lambda r(\alpha) \\
	r(\alpha\beta) & \;=\; r(\beta)r(\alpha) \\
	r(a\tp b) & \;=\;  r(a) \tp r(b) \\
	r(\alpha\tp \beta) & \;=\;  r(\alpha) \tp r(\beta)
\end{align}
for objects $a,b$ and morphisms $\alpha,\beta$. Diagrammatically, the action of  $r$ 
reflects diagrams about the horizontal axis (while acting as complex conjugation on $\cc$). 
Outside of this section, we usually denote $r$ by a bar: $\bar a = r(a)$ and $\bar\alpha = r(\alpha)$.

For objects, we require that $r(r(a)) = a$.\footnote{
We also frequently restrict our attention to reflection-invariant objects which satisfy
$r(a) = a$, but this is not a requirement for all objects.}
For morphisms, we have two choices.
In a pin+ reflection structure, we require $r^2$ to be the identity functor:
\be
	r^2 = \id .
\ee
In a pin$-$ reflection structure, we require $r^2$ to be the spin flip functor,
\be
	r^2 = (-1)^F .
\ee
The main examples of this paper all have pin+ reflection structures.

We require $r$ to be compatible with the other structure maps of $\spc$ (pivots, $F$, etc.).
For example, 
we require the following diagrams to be commutative:
\begin{align}
\vcenter{
\xymatrix @!0 @M=2mm @R=34mm @C=19mm{
\refa \ar[rr]^r \ar[d]_P&& \refb\\
\refd \ar[rr]^r&&\refc \ar[u]^P\\
	}}
\end{align}
and
\begin{align}
\vcenter{
\xymatrix @!0 @M=2mm @R=34mm @C=19mm{
\refcoha\ar[rr]^{F^{abc}_d} \ar[d]_r &&\refcohb \ar[rr]^r && \refcohc \ar[d]^{P_L^{-1} \tp P_R} \\
\refcohh\ar[rr]^{P_R \tp P_L^{-1}} &&\refcohg \ar[d]_{K_{12}} && \refcohd \ar[d]^{K_{12}} \\
    &&\refcohf && \refcohe \ar[ll]^{F^{\bar{a}^* \bar{d} \bar{c}^*}_{\bar{b}}} \\
	}}
\end{align}

A pin+ reflection structure on $\spc$ allows us to define the action of pin+ diffeomorphisms on $\spc$ string nets.

It follows from the ``back wall" line bundle construction of Appendix \ref{flb_appendix} that super pivotal categories $\mcc/\psi$
obtained via fermion condensation will have pin+ reflection structures whenever the parent category has an
ordinary bosonic reflection structure.

Our main use for pin+ reflections is to define a sesquilinear inner product on the string net space $A(Y; c)$.
Let $Y$ be a spin surface and let $-Y$ denote the same underlying surface but with the reversed spin structure.
The ``identity" map from $Y$ to $-Y$ is not a spin diffeomorphism (as it reverses orientation), but it is a pin+ diffeomorphism.
Using the reflection structure on $\spc$, we can use this pin+ diffeomorphism to map string nets in $A(Y; c)$ to 
string nets in $A(-Y; r(c))$.
If $r(c) = c^*$, then string nets in $A(Y;c)$ and $A(-Y;r(c))$ can be glued together to get a string net on the closed
spin surface $Y\cup_{\bd Y} -Y = \bd(Y\times I)$.
Using the path integral $Z(Y\times I) : A(Y\cup_{\bd Y} -Y) \to \cc$ now yields a sesquilinear inner product on $A(Y;c)$.
Since $A(Y;c)$ is finite-dimensional, we also get an inner product on the dual space $Z(Y;c)$.

Let $M$ be a spin 3-manifold.
Then the path integrals $Z(M): A(\bd M)\to \cc$ and $Z(-M): A(-\bd M)\to \cc$ are related by
\be \label{3man-pi-orev}
	Z(-M) = Z(M)\circ R ,
\ee
where $R:A(-\bd M) \to A(\bd M)$ is the antilinear map induced by the orientation-reversing identity map from $-\bd M$ to $\bd M$.
Equivalently, $Z(M)\in Z(\bd M)$ and $Z(-M)\in Z(- \bd M)$ and
\be \label{3man-pi-orevz}
	Z(M) = R(Z(-M)) .
\ee

Let $Y_1$ and $Y_2$ be spin surfaces and let $M$ be a cobordism from $Y_1$ to $Y_2$
(i.e.\ $\bd M = Y_2\cup -Y_1$).
Then $-M$ is a cobordism from $Y_2$ to $Y_1$.
The path integrals can be viewed as maps $Z(M): Z(Y_1)\to Z(Y_2)$ and $Z(-M):Z(Y_2)\to Z(Y_1)$.
It follows from \eqref{3man-pi-orevz} that $Z(-M)$ is the adjoint of $Z(M)$ with respect to the inner products on $Z(Y_1)$
and $Z(Y_2)$ defined above.








