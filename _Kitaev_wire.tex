%auto-ignore
%      this ensures the arxiv doesn't try to start TeXing here.
%!TEX root = super_lattice_models_draft.tex
%      prev line helps TeXShop do the right thing

%%%%%%%%%%%%%%%%
\section{Kitaev chain} \label{kitaev_wire}
%%%%%%%%%%%%%%%%

In this section, we show how the graphical formalism developed in previous sections
can be used to succinctly capture the salient features of the `Kitaev wire', 
Kitaev's toy model of a one-dimensional spinless $p$-wave superconductor \cite{kitaev2001}. 
This elucidates the connection between Majorana zero modes and Ising anyons and serves 
as a nice application of the graphical calculus of the $C_2$ theory, although the procedure we work 
out can be carried out for any theory containing at least one $q$-type object.
%, and provides 
%a simple (alas, perhaps too simple) illustration of the shadow world approach detailed in the previous section 
%(with appropriate modifications for the reduced dimensionality of the problem). 
The Hamiltonian we write down and the associated wavefunctions we construct are 
the same as those found in e.g. \cite{fidkowski2011}, but presented in a more graphical formalism and 
discussed within the framework of the tools developed in this paper. 

Recall that the $C_2$ theory has two simple objects $\unit,\beta$,
with $\beta\tp \beta\cong \cc^{1|1}\unit$,
and $\End(\beta) \cong \cliff_1$.
%We will see that the $\beta$ object plays the role of a Majorana fermion in the Kitaev wire, with a strand 
%of $\beta$ string being described by the physics of the Kitaev wire. 
We will focus on the $\beta$ object in the $C_2$ 
theory for concreteness, but our analysis works equally well on q-type objects $q$ in any theory.
% that satisfy $q\tp q \cong \cc^{1|1} \unit \oplus \dots$.
%\dave{ Probably don't need the last bit, but could say: $q\tp q^* \cong \cc^{1|1} \unit \oplus \dots$}

In what follows, we will show that a single strand of $\beta$ string is a diagrammatic description for the zero correlation length limit of the Kitaev chain.
This means that the string-net Hamiltonian in Section \ref{Super_pivotal_Hamiltonian} 
based on the $C_2$ theory describes a phase of fluctuating Kitaev wires, 
an idea previously investigated in \cite{tarantino2016,ware2016}.

The basic idea is to cut a single $\beta$ strand into pieces and examine what must be done to glue the 
pieces back together.
%Versions of these sentences have been constructed elsewhere
%Algebraically, this contraint is just $\tp_{\End(\beta)}$, 
%physically we can impliment this constraint by taking a free tensor product and project into the ground space of a Hamiltonian implimenting that constraint. 
%The Hamiltonian turns out to be a limiting case of the Kitaev wire, the zero correlation length limit.
We first note that the vector space associated to a single interval $I$ of $\beta$ strand can be written graphically as 
\dave{Did we use $V$ for any particular reason, or should it be $\mca$.} \ethan{I don't know if it ``should'' be anything, it's just a choice}
\be\label{VIbetabeta}
 V(I;\beta,\beta) = \cc \left[ \halfchain\;, \; \halfchaindot \right] = \cc^{1|1}.\ee
 
 Now we can consider splitting the interval $I$ into two smaller intervals $I_1,I_2$, such that $I_1\cup I_2 = I$.
 We then can reconstruct the vector space $V(I;\beta,\beta)$ from the vector 
 spaces $V(I_1;\beta,\beta),V(I_2;\beta,\beta)$ by gluing the two intervals $I_1,I_2$ together. 
Algebraically, this gluing is implemented by the tensor product. However, 
we must be sure to make the proper choice 
of tensor product to ensure that we don't produce any extra degrees of freedom during the gluing. 
The standard tensor product $\tp_\cc$ doesn't work, since then $V(I_1;\beta,\beta) \tp_\cc V(I_2;\beta,\beta) \cong \cliff_2 \not\cong V(I;\beta,\beta)$. 

The correct tensor product to use is the relative tensor product $\tp_{\End(\beta)}$ (a.k.a $\tp_{\cliff_1}$) discussed at length in Section \ref{modified_tensor_product}. 
With this tensor product, we (rather trivially) have 
%We thus can write 
\be V(I;\beta,\beta) \cong V(I_1;\beta,\beta) \tp_{\End(\beta)} V(I_2;\beta,\beta),\ee
which tells us how to split apart the $I$ interval correctly. 

Graphically, the relative tensor product $\tp_{\End(\beta)}$  is needed to mod out by local relations involving the sliding of fermions along $\beta$ lines, 
as was discussed Section \ref{modified_tensor_product}. 
%where the relative tensor product $\tp_{\End(\beta)}$ (a.k.a $\tp_{\cliff_1}$) is needed to mod out by local relations involving the sliding of fermions along $\beta$ lines, 
%as was discussed at length in Section \ref{modified_tensor_product}. 
Utilizing $\tp_{\cliff_1}$ is equivalent to performing the regular tensor product $\tp_\cc$ and modding out by the equivalence relations
\be
\label{graphical_equiv_reln} 
\halfchain \tp_\cc \halfchaindot \; =\; \halfchaindot \tp_\cc \halfchain\; \quad \quad \text{and} \quad \quad 
\halfchaindot \tp_{\cc} \halfchaindot \;  = -A^4\; \halfchain \tp_{\cc}  \halfchain,
\ee 
where we have assumed a Koszul ordering for the fermions which increases from left to right (see Table \ref{C2_data_table} for the origin of the phase $A^4$).
%\be
% \horizbeta \tp_\cc \horizbetadot  \sim  \horizbeta \tp_\cc \horizbeta\;.
%\ee

As we did with the string-net Hamiltonian in Section \ref{Super_pivotal_Hamiltonian}, 
we can implement these equivalence relations energetically, via an appropriately defined Hamiltonian.

Consider an interval $I$ of $\beta$ string cut into $N$ segments: $I = I_1\cup I_2\cup\dots\cup I_N$.
Each segment $I_i$ will end up mapping to a single physical site in the Kitaev chain. 
The local Hilbert space at each $I_i$ segment is generated by two basis vectors $v_e,v_o$, 
which for convenience we draw as
\begin{align} \label{vevodefn}
v_e \; = \;\LocalHilba\,,\qquad v_o \; = \; \LocalHilbb\;.
\end{align}
The upward-curved ends on each $\beta$ segment are drawn purely for aesthetic purposes, and exist solely to make drawing the Kitaev chain slightly easier. 
The local Hilbert space is then
\begin{align}
V(I_i; \beta, \beta) = \langle v_e,v_o\rangle = \cc \left[ \LocalHilba, \LocalHilbb \right].
\end{align}


The total Hilbert space of the chain is given by tensoring each local Hilbert space together:
\begin{align} 
\label{HilbInterval}
\mch_I  = V(I_1; \beta, \beta)  \tp_\cc V(I_2; \beta, \beta)  \tp_\cc \cdots \tp_\cc V(I_N; \beta, \beta).  
\end{align} 
However, this Hilbert space is too large, since we have not yet modded out by the equivalence relations \eqref{graphical_equiv_reln}. 
Instead of using the relative tensor product to do this, we can make use of the projectors
\begin{align} \label{kitaev_chain_ham_sitei}
H_i =  \frac{1}{2} \left( \Id_{\text{\; site $i$}} \;\; - \;\; \TwoLinedotdot_{\text{site $i$}} \right),
\end{align}
where the left lines are applied to right end of the $i$th strand and the right lines are applied to the left end of the $(i+1)$-th strand\footnote{Note that there are no terms that act on both the right and left ends of a single strand---the absence of such terms is because we are working in topological phase of the Kitaev chain in the completely staggered limit. Physically, this corresponds to tuning the chemical potential to zero and the 
magnitude of the superconducting gap to the hopping amplitude.}.
%\ethan{I am in favor of writing the Hamiltonian like this, with the understanding that when we draw pictures and stuff, we are modding out by fermion slides and that the spin framing is fixed so that we have the usual $C_2$ diagrammatic manipulations. We can talk about $\alpha(e)$ stuff and edge terms to implement endos etc, but I think we probably shouldn't}

The image of this projector on a pair of adjacent string endpoints is 
\begin{align} 
\quarterpiper \tp_\cc \quarterpipel + \quarterpiperdot \tp_\cc \quarterpipeldot = \halfchain \tp_{\End(\beta)} \halfchain,
\end{align}
so that using $\tp_\cc$ and projecting with $H$ is equivalent to using $\tp_{\End(\beta)}$ (in Section \ref{Super_pivotal_Hamiltonian}, the $(1-D_e)$ term played the role of $H$). 
Thus, $H_i$ is responsible for gluing together ends of $\beta$ strands. 
In terms of electronic operators, $H_i$ implements hopping and pairing between electrons in nearest neighbor sites $i$ and $i+1$.
%The Hamiltonian 

We form our Hamiltonian from a sum of $H_i$ projectors acting between each pair of strands:
\begin{align}
H = \sum_{i = 1}^{N-1} H_i.
\end{align}
This Hamiltonian describes the zero correlation length limit of the Kitaev chain.

\ethan{I think we should keep the following so that people believe that the hamiltonian really is the kitaev chain hamiltonian} 
Each $H_i$ contains two terms: the first is the identity operator, while the second is nontrivial and is identified with $i \gamma_1 \gamma_2$, 
a Majorana bilinear which acts as the fermion parity operator $(-1)^F$ on the on-site Hilbert space.
To see this, we can simply check the action of the second term on the basis vectors of the single (physical) site Hilbert space. 
Graphically, the  
second term acts by stacking diagrams onto input states, and so it acts
on single-site basis states as
%\begin{align}
%\TwoLinedotdot\; \circ \; \CupSigma \; =\;  \CupSigma \quad \text{and} \quad \TwoLinedotdot\; \circ\; %\CupSigmadot \; =\;   - \; \CupSigmadot.
%\end{align}
\begin{align}
\TwoLinedotdot\; \circ \; v_e =\; v_e \quad \text{and} \quad \TwoLinedotdot\; \circ\; v_o \; =\;   - \; v_o.
\end{align}
This is precisely the action of the Majorana bilinear $i\gamma_1\gamma_2 = (-1)^F$.

In order to show that the Hamiltonian is Hermitian, we need to define a graphical way of implementing 
Hermitian conjugation. We define the Hermitian conjugate of a diagram to be its reflection about 
the horizontal axis, with the action of reflection determined by the properties of the complex line 
bundle constructed in Appendix \ref{fermion_line_bundle} (for more details on the reflection structure, see Section \ref{reflection_ss}). One then verifies that $H^\dagger H = -\lambda^2 \unit$, where the 
minus sign is a Koszul sign, $\lambda$ is the phase picked up when removing a pair of fermions 
as in \eqref{removing_fermions}. 
As usual, the notation $H^\dagger H$ graphically corresponds to stacking $H^\dagger$ on top of $H$.
Since $\lambda^2 = -1$ as we prove in Appendix \ref{fib_appendix}, we verify that $H^\dagger H = \unit$.  

It is straightforward to find the ground states of this Hamiltonian: there are two zero-energy ground states, differing by their fermion parity:
\begin{align} \label{kitaev_wire_ground_states}
\Psi_e = \;\StaggaredGSEven \; \cdots \; \StaggaredGSEvenR\;, 
\qquad \text{and} \qquad 
\Psi_o =\; \StaggaredGSOdd \; \cdots  \; \StaggaredGSEvenR\;.
\end{align}

A nice feature of this diagrammatic notation is that the ``delocalization" of the fermion number in the topological phase of the 
Kitaev wire is made manifest.
For example, the fermion in the odd-parity state $\Psi_o$ is not localized: it is free to slide along the bottom $\beta$ line
from one end of the chain to the other (with the sliding being implemented through the odd elements in $\End(\beta)$, or in the context of \eqref{graphical_equiv_reln} by the use of the modified tensor product $\tp_{\cliff_1}$).

%At first glance ... in terms of Hilbert spaces, since each physical site carries a Hilbert space of $\cc^{1|1}$, for a chain with $n$ physical sites we have the naive Hilbert space $\mch \cong \cliff_1^{\otimes n}$, which is far bigger than the total ground state space $\cc^{1|1}$ generated by the vectors $\Psi_e$ and $\Psi_o$. 
%... we are actually using the tensor product $\tp_{\cliff_1}$, so that in fact the degrees of freedom are $\cliff_1 \tp_{\cliff_1}  \dots \tp_{\cliff_1} \cliff_1 \cong \cliff_1$. 

To better understand the wavefunctions $\Psi_e,\Psi_o$, we can perform a change of basis by applying $F$-moves to 
each cup-cap pair with the $F$-move \eqref{cupcap_fmove}. 
On a Kitaev wire with $n$ physical sites, we have
\be \Psi_e = \frac{1}{d^{n-1}} \bigotimes_{i=1}^n (-A^4)^{n_f/2} \sum_{\{v_i\}}  v_i,\ee
where the sum is over all combinations of $v_i =v_o,v_e$ such that only an even number $n_f$ of $v_o$ vectors appear 
in the tensor product, and where $v_e,v_o$ are defined as in \eqref{vevodefn}.
For the odd wavefunction $\Psi_o$, we get 
\be \Psi_e = \frac{1}{d^{n-1}} \bigotimes_{i=1}^n(-A^4)^{(n_f+1)/2} \sum_{\{v_i\}}  v_i,\ee
where the sum is now restricted so that only an odd number $n_f$ of $v_o$ vectors appear 
in the tensor product.
From the expressions for $\Psi_e,\Psi_o$ in this basis, we see that they are given by configurations 
that are coherent sums over all fermion parity even and fermion parity odd states, respectively.  

%\ethan{talk about KW duality here?}

Note that the fermion dot appearing in $\Psi_o$ has zero energy, while this is not true for fermion dots 
appearing on the interior intermediate cups (since the Hamiltonian does not act on either the beginning of the first strand in the chain or the end of the last strand). 
Physically, this implies the presence of two Majorana zero modes at the ends of the chain. 
Excited states are easily constructed by putting dots on the intermediate cups. 

If we instead consider the Kitaev chain on a circle, the wavefunctions can be found in a similar manner.
The 
ground-state wavefunctions 
have a similar form to those in \eqref{kitaev_wire_ground_states}, except the bottom $\beta$ string 
forms a closed loop below the series of $\beta$ cups. 
In this case, the fermion parity of the wavefunction is uniquely determined by the spin structure, being 
even for $S^1_B$ and odd for $S^1_N$, which can be verified using the graphical calculus for the 
$C_2$ theory. For both spin structures, the ground state is unique. 
We write the wavefunctions on the $B$ and $N$ sectors as \ethan{modify this so that the big legs on the ends are chopped off and an appropriate $(-1)^F$ operator is inserted to be the branch cut on the non-bounding wavefunction}
\begin{align} \label{kitaev_wire_circle_ground_states}
\Psi_B = \cdots\;\StaggaredGSEven \; \cdots \; \StaggaredGSEvenR\; \cdots, 
\qquad \qquad 
\Psi_N =\cdots\; \; \StaggaredGSOdd \; \cdots  \; \StaggaredGSEvenR \; \cdots,
\end{align}
where the ??? operator inserted in the $\Psi_N$ wavefunction is the branch cut needed to account for the non-bounding spin structure. 
The odd-parity $\Psi_N$ wavefunction hosts a single fermion dot on the lower circular $\beta$ line, 
and the delocalization of fermion parity is again manifest, as the exact location of the fermion dot 
is arbitrary. 

\ethan{Dave might have to change notation to find something he's happy with}
Note that the way we draw vectors in the Hilbert space of the Kitaev wire is essentially a graphical presentation of a matrix product state (MPS). 
In the MPS formalism, a site $i$ labeled by the basis vector $a_i$ is attached 
to is associated with a matrix $A^{a_i}_{xy}$, where $x,y$ are virtual indices. 
The $A^{a_i}$ matrices can be determined from the shadow world approach, and in this case are given by the structure constants of $\cliff_1$ \cite{turzillo2016}. 
Wavefunctions are then constructed by contracting the virtual indices following the 
shadow world prescription. To construct a wavefunction $|\Psi\rangle$ we need to compute 
partition functions of the form $Z(M;\emptyset,v_1)$, where $\emptyset$ denotes 
the input vacuum state. 
By using our earlier results \eqref{bounding_trace} and \eqref{nonbounding_trace} on 
partition functions, we see that 
\dave{I don't understand how this is compatible with \eqref{graphical_MPS_state} and \eqref{kitaev_wire_circle_ground_states}. 
It seems like \eqref{kitaev_wire_circle_ground_states} requires all $a_i = \beta$. }
\be |\Psi\rangle = \sum_{\{ a_i \} }\tr(A^{a_1} \cdots A^{a_N}) \ket{a_1 \cdots a_n} \ee
if the Kitaev wire is defined over an interval or on $S^1_B$, and 
\begin{align}
 |\Psi\rangle & = \sum_{\{ a_i \} }\tr((-1)^FA^{a_1} \cdots A^{a_N}) \ket{a_1 \cdots a_n} \\ 
 & =\sum_{\{ a_i \} }\str(A^{a_1} \cdots A^{a_N}) \ket{a_1 \cdots a_n} 
 \end{align}
if the wire is defined on $S^1_N$. 
This is the presentation of the wavefunctions used in the tensor network community, 
and agrees with the results of \cite{turzillo2016,bultinck2017b}. 
The graphical depiction of these states is given by the $\Psi_B,\Psi_N$ appearing in \eqref{kitaev_wire_circle_ground_states}. 

\begin{figure}
\centering
\includegraphics{box_beta_loop.pdf}
\caption{ \label{box_beta_loop} A sketch of the setup used when considering a shadow world construction which hosts a closed $\beta$ loop on its surface. Restricting the theory to the submanifold defined by this loop produces a result equivalent to the Kitaev wire. \ethan{dave may want to redo}}
\end{figure}

This $(1+1)$-dimensional example sheds some light on the $(2+1)$-dimensional examples 
considered in the majority of the paper. 
For example, in the simple case of the $C_2$ theory, it allows us to make the observation that
$\beta$ strings are equivalent to fluctuating Kitaev chains in their topological phase. 
We can see this by considering a state in the $C_2$ theory constructed 
from the shadow world approach described in the previous section with $v_0 = \emptyset$ and $v_1$ a state 
with a single closed $\beta$ loop which inherits spin structure $X$, see Figure \ref{box_beta_loop}.
The degrees of freedom in this state are the same as 
the degrees of freedom in the Kitaev chain constructed over $S^1_X$, 
where we identify each vertex Hilbert space with a pair $a_{2j}\tp a_{2j+1}$ of Majoranna
variables connecting two adjacent physical sites. 
Furthermore, the Hamiltonian defined in Section \ref{Super_pivotal_Hamiltonian} is the same as the Kitaev 
chain Hamiltonian \eqref{staggered_H} when acting on $v_1$. 
The vertex and plaquette terms are trivially satisfied by the choice of $v_1$, and so we only need to focus 
on the edge term $\lambda_e \sum_{e\in \mce} (1-D_e)$. As described in Section \ref{edge_term}, 
the operator $D_e$ changes the fermion parity of the vertices $v,v'$ at the ends of the edge $e$, provided
that $e$ is colored by a $\beta$ line. 
This is exactly the same process implemented by the Kitaev wire Hamiltonian: both Hamiltonians
 are responsible for ``sliding'' fermions along $\beta$ strings.  

To summarize, states in the $C_2$ theory are formed from superpositions of fluctuating Kitaev chains, which in our case are exactly equivalent to $\beta$ strings. 
With these remarks in mind, it 
is natural to view q-type strings in more general $(2+1)$-dimensional super fusion categories 
as different types of fluctuating Kitaev chains, which is a point of view adopted in\cite{tarantino2016,ware2016,kapustin2017}. 
There are several pieces of evidence for this: the fermion parity of a closed q-type string 
depends only on the string's homology class and is determined by the spin structure it inherits, 
and in the simplest case of the $C_2$ theory, q-type strings literally are Kitaev chains, as shown 
above. 

However, we have seen examples of theories in which q-type strings do not behave like Kitaev wires.
In such theories (like the $SO(3)_6$ and $\halfesix$ examples considered earlier), two q-type
strings can fuse to a q-type string, and two m-type strings can fuse to a q-type. 
If all open q-type strings hosted Majorana zero modes at their ends, such fusion rules would 
be disallowed, since the number of Majorana zero modes must be preserved modulo 2.

%When transforming back to the original basis, one sees that the state is actually an even superposition of all possible placements of an even number of fermions, similarly for the odd state.

 