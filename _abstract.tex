\begin{abstract}
We study fermionic topological phases using the technique of fermion condensation. 
Fermion condensation procides a systematic construction of fermionic topological phases that bootstraps off well understood bosonic topological phases that host an emergent fermion.
Our approach to fermion condensation can roughly be understood 
as coupling the parent bosonic topological phase to a phase of physical fermions, and condensing pairs of physical and emergent fermions. 
%Vortices of the physical fermions are pinned to vortices of the emergent fermions.
%Consistency of the condensed theory requires the physical vortices to bind with emergent vortices.
As a fusion theory, there are two distinct types of objects, which we call ``m-type'' and ``q-type''.
The endomorphism algebras of q-type objects are complex Clifford algebras, and they
have no analogues in bosonic theories. 
As a 2+1-dimensional theory, condensing pairs of physical and emergent fermions requires physical vortices to bind with emergent vortices.
Consequently particles carry a grading determined by whether or not they harbour a physical vortex.
We analyze the 2+1-dimensional theories using a fermionic generalization of the tube category.
We then prove a series of results relating data in fermionic theories to data in their parent bosonic theories;
for example, if $\mathcal{C}$ is a modular tensor category containing a fermion, then the tube category 
constructed from the condensed theory satisfies 
$\textbf{Tube}(\mathcal{C}/\psi) \cong \mathcal{C} \times (\mathcal{C}/\psi)$.  
We also study how modular transformations, fusion rules, and coherence relations 
are modified in the fermionic setting, 
prove a fermionic version of the Verlinde dimension formula,
construct a commuting projector lattice Hamiltonian for fermionic theories, and 
write down a fermionic version of the Turaev-Viro-Barrett-Westbury state sum.  
A large portion of this work is devoted to three detailed examples of performing fermion condensation to produce fermionic topological phases: we condense fermions 
in the Ising theory, the $SO(3)_6$ theory, and the $\frac{1}{2}\text{E}_6$ theory, and compute the 
quasiparticle excitation spectrum in each of the condensed theories. 
\end{abstract}
